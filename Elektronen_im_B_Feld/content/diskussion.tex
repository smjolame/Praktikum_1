\section{Diskussion}
\label{sec:Diskussion}
Der über den Versuch bestimmte Wert der spezifischen
Elektronenladung beträgt  \\
${\frac{e_0}{m_0}=\SI{1.30(06)e10}{\coulomb\per\kilo\g}}$. Laut Literatur \cite{spez_elektroncharge} besitz die spezifische Elektronenladung eine Größe von 
$\frac{e_0}{m_0}=\SI{17.5882e10}{\coulomb\per\kilo\g}$. Damit weicht der im Versuch bestimmte Wert um 
$\SI{92.63(34)}{\percent}$ von dem Literaturwert ab. 


Die totalintensität des Erdmagentfeld kann in dem Versuch zu 
$B_{total}=\SI{8.424e-05}{\tesla}$ bestimmt werden. Dieser Wert weicht um \SI{71.24}{\percent} vom Literaturwert \cite{mag_DO} ${B_{total}^{lit}=\SI{4.91941e-05}{\tesla}}$ ab.


Beide zu bestimmenden Größen haben jeweils recht große Abweichungen bezüglich ihres Literaturwertes. Dies kann unteranderem auf die mangelhafte Funktionsweise des verwendeteten Kompasses zurückgeführt werden. Denn allein der gemessene Inklinationswinkel von $\phi_{inkl}=\SI{58}{\degree}$ weicht bereits stark von dem Literaturwert \cite{mag_DO} von $\phi_{inkl}^{lit}=\SI{66}{\degree}$ ab. Da die Präzesion des Kompasses sowohl für die Bestimmung des spezifischen Elektronenladung als auch für die Bestimmung der totalintensität des Erdmagentfeldes wichtig ist, kann zumindest ein Teil der großen Abweichung beider Werte auf diesen Mangel zurückgeführt werden. Außerdem lässt sich der zu Messende Auftreffpunkt des Elektronenstrahls auf dem Bildschirm nur mit einer mittelmäßigen Genauigkeit bestimmen. Eine feinere Skala hätte darüberhinaus auch eine feinere Schrittweite der angelegten Ströme ermölicht, wodurch mehr Messwerte hätten aufgenommen werden können und damit der statistische Fehler weiter verringert werden können. Wobei neben der Feinheit der Skala auch die Schärfe des Leuchtpunktes nur bis zu einer gewissen Genauigkeit geregelt werden kann. Auch das hat die Schrittweite der sinnvollen Messungen vergrößert.


Unter Beachtung der oben genannten Mängel an den für den Versuch verwendeteten Geräten ließe sich ein insgesamt genaueres Ergebnis erzielen. Jedoch kann nur durch eine Wiederholung des Versuchs die Gewissheit dieser Aussage bestätigt werden.






%Kurze Zusammenfassung der Ergebnisse
%-Vergleich mit Literaturwerten
%-Vergleich mit verschiedenen Messverfahren
%-bei Abweichungen mögliche Ursachen finden