\section{Theorie}
\label{sec:Theorie}

Zwischen einem homogenen Magnetfeld und einem Elektron kommen nur dann Wechselwirkungen zustande, wenn das Elektron bewegt wird
und dessen Geschwindigkeit $\vec{v}$ eine Komponente senkrecht zum Magnetfeld $\vec{B}$ besitzt. Unter dieser Bedingung wirkt 
auf das Elektron eine Lorentz-Kraft, die durch 
\begin{equation}
    \vec{F_\text{L}} = q\cdot \vec{v} \times \vec{B}
    \label{eqn:lorentz}
\end{equation}
beschrieben werden kann. In einem kartesischen Koordinatensystem, indem sich das Elektron nur in $z$-Richtung bewegt und das
homogene $B$-Feld in $x$-Richtung verläuft, wirkt also die Lorentz-Kraft zum Zeitpunkt des Eintreffens des Elektrons in 
$y$-Richtung. Das Elektron bewegt sich folglich auf einer Kreisbahn, wobei $\vec{v} $ immer 
senkrecht zu $\vec{B} $ und $\vec{F_\text{L}}$ steht. Demnach ist
\begin{equation}
    \vec{F_\text{L}} \cdot \symup{d}\vec{s} = 0\,,
\end{equation}
mit der Konsequenz, dass die Energie des Elektrons erhalten bleibt, wobei die potentielle und die kinetische Energie jeweils 
konstant bleiben, woraus folgt, dass $v$ konstant bleibt. Über den Zusammenhang von Lorentz-Kraft und Zentrifugalkraft kann ein 
Krümmungsradius der Elektronenbahn über 
\begin{equation}
    r = \frac{m_0 v}{e_0 B}
    \label{eqn:radius1}
\end{equation}
bestimmt werden.  
Der Krümmungsradius kann durch den vorliegenden Versuch einfacher über die Gleichung 
\begin{equation}
    r = \frac{L^2 + D^2}{2D}
    \label{eqn:radius2}
\end{equation}
ermittelt werden, wobei $L$ die Länge des Einflussbereiches des Magnetfeldes ist, und $D$ die dadurch erzeugte Abweichung des 
Auftreffpunktes der Elektronen. Unter Berücksichtigung der aus dem Energiesatz folgenden Gleichung 
\begin{equation}
    v = \sqrt{\frac{2 U_B e_0}{m_0}}
\end{equation} 
lässt sich durch Gleichsetzen von Gleichung \ref{eqn:radius1} und \ref{eqn:radius2} der Zusammenhang 
\begin{equation}
\label{eqn:V}
    \frac{D}{L^2 + D^2} = \frac{1}{\sqrt{8U_B}} \sqrt{ \frac{e_0}{m_0}} B
\end{equation}
herstellen, wodurch $\frac{e_0}{m_0}$ bestimmt werden kann. 

Die magnetische Feldstärke $B$ im Zentrum eines Helmholtzspulenpaares beträgt:
\begin{equation}
    B=\mu_0 \frac{8}{\sqrt{125}}\frac{NI}{R}  
    \label{eqn:B}
\end{equation}
    (N=Windungszahl; I=Spulenstrom; R=Spulenradius; $\mu_0=4\pi 10^-7\si{\V\s\per\A\per\m}$)
