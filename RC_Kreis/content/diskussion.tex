\section{Diskussion}
\label{sec:Diskussion}

Bei den Unterschiedlichen Bestimmungen der Zeitkonstante fällt auf, dass alle drei Werte nah beieinander sind. Die jeweiligen 
Standardabweichungen sind auch relativ gering. Die Zeitkonstante, wie sie durch die frequenzabhängige Kondensatorspannungsamplitude 
bestimmt wurde, weicht etwas von den anderen beiden ab, siehe Tabelle \ref{tab:a} und \ref{tab:b}.
Die theoretische Zeitkonstante des Kondensators beläuft sich auf $RC = \SI{1.4034(559)}{\milli\s}$. So weichen die einzeln gemessenen und errechneten
Zeitkonstanten im Durchschnitt ca um $0.5132 \si{\milli\s}$ nach unten hin ab. Zwei errechnete Zeitkonstanten $RC_1= \SI{0.9964(0339)}{\milli\s}$ und $RC_3=\SI{0.9383(0586)}{\milli\s}$
liegen dabei recht nah beieinander. Sie weichen im Gegensatz zur dritten errechneten Zeitkonstanten $RC_2=\SI{0.7440(0590)}{\milli\s}$ nur um ca. $\SI{71}{\percent}(RC_1)$ und $\SI{66}{\percent}(RC_3)$ ab.
Diese ähnliche Abweichung von den bestimmten Zeitkonstanten könnte auf einen systematischen Fehler im Messvorgang zurück zu führen sein. Im Allgemeinen sind die hohen Abweichungen
unteranderem durch die verhältnismäßig niedrige Skalengenauigkeit des Oszilloskops zu begründen. 

Der zeitliche Verlauf der Kondensatorspannung, sowie der frequenzabhängige Verlauf der Kondensatorspannungsamplitude und der
Phasenverschiebung weisen jeweils keine Ungewöhnlichkeiten auf. Die jeweiligen Ausgleichskurven wären eventuell präziser
gewesen, wenn mehrere Messwerte aufgenommen worden wären. Des Weiteren stimmen die Theoriekurven in ihrem tendenziellen Verlauf mit den Ausgleichskurven der Messwerte überein. Jedoch ist bei allen Abbildungen eine Verschiebung der Theoriekurven entlang der Spannungs-Achse zu beobachten. Auch hier kann die Ursache vielfältig sein. Ein systematischer Fehler ist ebenfalls nicht ausgeschlossen, da die Abweichungen alle in Richtung höherer Spannungen vorliegen.

An den Fotos \ref{fig:sinus},\ref{fig:dreieck} und \ref{fig:rechteck} ist zu erkennen, dass die Kurve der Kondensatorspannung jeweils
das Integral der Generatorspannung ist. Der RC-Kreis kann also daher als Integrator verwendet werden. 




%Kurze Zusammenfassung der Ergebnisse
%-Vergleich mit Literaturwerten
%-Vergleich mit verschiedenen Messverfahren
%-bei Abweichungen mögliche Ursachen finden