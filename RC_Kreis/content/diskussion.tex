\section{Diskussion}
\label{sec:Diskussion}

Bei den Unterschiedlichen Bestimmungen der Zeitkonstante fällt auf, dass alle drei Werte nah beieinander sind. Die jeweiligen 
Standardabweichungen sind auch relativ gering. Die Zeitkonstante, wie sie durch die frequenzabhängige Kondensatorspannungsamplitude 
bestimmt wurde, weicht etwas von den anderen beiden ab, siehe Tabelle \ref{tab:a} und \ref{tab:b}.

Der zeitliche Verlauf der Kondensatorspannung, sowie der frequenzabhängige Verlauf der Kondensatorspannungsamplitude und der
Phasenverschiebung weisen jeweils keine Ungewöhnlichkeiten auf. Die jeweiligen Ausgleichskurven wären eventuell präziser
gewesen, wenn mehrere Messwerte aufgenommen worden wären. 

An den Fotos \ref{fig:sinus},\ref{fig:dreieck} und \ref{fig:rechteck} ist zu erkennen, dass die Kurve der Kondensatorspannung jeweils
das Integral der Generatorspannung ist. Der RC-Kreis kann also daher als Integrator verwendet werden. 




%Kurze Zusammenfassung der Ergebnisse
%-Vergleich mit Literaturwerten
%-Vergleich mit verschiedenen Messverfahren
%-bei Abweichungen mögliche Ursachen finden