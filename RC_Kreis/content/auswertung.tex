\section{Auswertung}
\label{sec:Auswertung}


\subsection{Bestimmung der Zeitkonstante RC 1} % (fold)
\label{sub:Bestimmung der Zeitkonstante RC 1}

\begin{table}
    \centering
    \caption{Messdaten der Spannung in Abhängigkeit der Zeit.}
    \begin{tabular}{S S}
    \toprule
    {$t \:/\: \si{\milli\s}$} & {$U_c \:/\: \si{\volt}$} \\
    \midrule
        0.0 & 9.2 \\
        0.2 & 7.0 \\
        0.4 & 5.4 \\
        0.6 & 4.2 \\
        0.8 & 3.4 \\
        1.0 & 2.6 \\
        1.2 & 2.2 \\
        1.4 & 1.8 \\
        1.6 & 1.5 \\
        1.8 & 1.3 \\
        2.0 & 1.1 \\
        2.2 & 1.0 \\
        2.4 & 0.8 \\
        \bottomrule
    \end{tabular}
    \label{tab:a}
\end{table}

\begin{figure}
    \centering
    \caption{Spannung in Abhängigkeit der Zeit als Messwerte und Ausgleichsgerade.}
    \includegraphics[width=\textwidth]{build/a.pdf}
    \label{fig:a}
\end{figure}

In Abbildung \ref{fig:a} sind die Messwerte aus Tabelle \ref{tab:a} aufgetragen. 
Aus dem Anstieg der Ausgleichsgerade lässt sich die Zeitkonstante aus Gleichung \eqref{eqn:RC} zu $RC= \SI{0.9964(0339)}{\milli\s}$ bestimmen.
Die Ausgleichsgerade wurde dabei mit scipy erstellt.\cite{scipy} 




% subsection Bestimmung der Zeitkonstante RC 1 (end)

\subsection{Bestimmung der Zeitkonstante RC 2} % (fold)
\label{sub:Bestimmung der Zeitkonstante RC 2}

\begin{table}
    \centering
    \caption{Messdaten der Spannung in Abhängigkeit der Frequenz.}
    \begin{tabular}{S S}
    \toprule
    {$f \:/\: \si{\hertz}$} & {$A \:/\: \si{\volt}$} \\
    \midrule

        100 & 10.00 \\
        200 & 6.10 \\
        300 & 5.40 \\
        400 & 4.00 \\
        500 & 3.15 \\
        600 & 2.90 \\
        700 & 2.70 \\
        800 & 2.60 \\
        900 & 2.55 \\
        1000 & 2.45 \\
        \bottomrule
    \end{tabular}
    \label{tab:b}
\end{table}

\begin{figure}
    \centering
    \caption{Auf $U_0$ genormte Amplitude der Spannung in Abhängigkeit der Frequenz.}
    \includegraphics[width=\textwidth]{build/b.pdf}
    \label{fig:b}
\end{figure}

Es wird mit einer anderen Herangehensweise die Zeitkonstante bestimmt. Die Ergebnisse können der Tabelle \ref{tab:b} und der
Abbildung \ref{fig:b} entnommen werden. Über eine Ausgleichskurve, welche als Vorlage die Gleichung \eqref{eqn:A} verwendet, lässt
sich die Zeitkonstante bestimmen.
Die Zeitkonstante ergibt sich zu $RC=\SI{0.7440(0590)}{\milli\s}$.

% subsection Bestimmung der Zeitkonstante RC 2 (end)

\subsection{Phasenverschiebung zwischen Generator- und Kondensatorspannung} % (fold)
\label{sub:Phasenverschiebung zwischen Generator- und Kondensatorspannung}

\begin{table}
    \centering
    \caption{Messdaten des Phasenunterschiedes zwischen Generator- und Kondensatorspannung.}
    \begin{tabular}{S S S S}
    \toprule
    {$f \:/\: \si{\hertz}$} & {$a \:/\: \si{\milli\s}$} & {$b \:/\: \si{\milli\s}$} & {$\phi \:/\: \si{\radian}$} \\
    \midrule
        100 & 0.70 & 8.80 & 0.50\\
        200 & 0.55 & 4.40 & 0.79\\
        300 & 0.48 & 2.90 & 1.04\\
        400 & 0.42 & 2.20 & 1.20\\
        500 & 0.36 & 1.84 & 1.23\\
        600 & 0.32 & 1.52 & 1.32\\
        700 & 0.28 & 1.32 & 1.33\\
        800 & 0.26 & 1.16 & 1.41\\
        900 & 0.24 & 1.00 & 1.51\\
        1000 & 0.22 & 0.92 & 1.50\\
        \bottomrule
    \end{tabular}
    \label{tab:c}
\end{table}

\begin{figure}
    \centering
    \caption{Messdaten zum Phasenunterschied in Abhängigkeit der Frequenz und zugehörige Ausgleichskurve.}
    \includegraphics[width=\textwidth]{build/c.pdf}
    \label{fig:c}
\end{figure}

Es wird mit einer dritten Methode die Zeitkonstante bestimmt. Es wird dabei wieder mit einer Ausgleichskurve, welche diesmal an die 
Gleichung \eqref{eqn:phi} angelehnt ist. Die Abbildung \ref{fig:c} stellt die so bestimmte Ausgleichskurve da. Mit den Werte der Tabelle \ref{tab:c} ergibt sich die auf diese Weise berechnete Zeitkonstante 
zu $RC=\SI{0.9383(0586)}{\milli\s}$. 



% subsection Phasenverschiebung zwischen Generator- und Kondensatorspannung (end)

\subsection{RC-Kreis als Integrator} % (fold)
\label{sub:RC-Kreis als Integrator}


\begin{figure}
    \centering
    \caption{Polarplot zum zeigen des Phasenunterschiedes zwischen Generator- und Kondensatorspannung.}
    \includegraphics[width=\textwidth]{build/d.pdf}
    \label{fig:d}
\end{figure}

Im polarplot ist dabei die Spannung radial und die Phase lateral aufgetragen.



% subsection RC-Kreis als Integrator (end)
%Messwerte: Alle gemessenen physikalischen Größen sind übersichtlich darzustellen.
%
%Auswertung:
%Berechnung der geforderten Endergebnisse
%mit allen Zwischenrechnungen und Fehlerformeln, sodass die Rechnung nachvollziehbar ist.
%Eine kurze Erläuterung der Rechnungen (z.B. verwendete Programme)
%Graphische Darstellung der Ergebnisse