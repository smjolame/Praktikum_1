\section{Diskussion}
\label{sec:Diskussion}



Bei der Berechnung der experimentell bestimmten Trägheitsmomente ist auffällig, dass sowohl die der beiden Zylinder, als auch die 
der Puppe in beiden Positionen negativ sind. Das ist physikalisch nicht sinnvoll, die einzelnen 
Trägheitsmomente weichen außerdem sehr stark von den theoretisch berechneten ab. Beim symmetrischen Zylinder ist die Abweichung etwa
250\% und beim asymmetrischen sogar über 500\%. Eine ähnliche Abweichung von den Theoriewerten ist bei der Puppe zu sehen.

Um zumindest annähernd eine Aussage über die Qualität der Messungen machen zu können, wird die Differenz zwischen den experimentellen 
und den theoretischen Werten untersucht.
Dieser Wert sollte von allen untersuchten Körpern ähnlich sein, um 
die fehlerhaften experimentellen Werte auf einen allgmein ungenauen Aufbau zurückführen zu können.

Die Differenzen ergeben sich wie folgt:
\begin{align}
T_\text{diff,sym}& = \SI{266.32399 (00015)e-5}{\kilo\gram\square\m} \\
T_\text{diff,asym}& = \SI{276.82207 (00015)e-5}{\kilo\gram\square\m} \\
T_\text{diff,pos1}& = \SI{279.63397 (00014)e-5}{\kilo\gram\square\m} \\
T_\text{diff,pos2}& = \SI{282.80247 (00013)e-5}{\kilo\gram\square\m}
\end{align}

Wie zu sehen ist, sind die Differenzen von experimentell und theoretischen Werten jeweils sehr ähnlich.
Daraus kann geschlossen werden, dass obwohl die experimentellen Werte selbst nicht sinnvoll sind, deren Verhältnis zueinander recht 
genau stimmen sollte. 

Beim Vergleich der beiden Zylinder ist das Trägheitsmoment des symmetrischen größer als das des asymmetrischen. Das Verhältnis der
beiden Trägheitsmomente stimmt dabei für Theorie und Experiment etwa überein. 

Werden die Trägheitsmomente der Puppe in den verschiedenen Positionen verglichen, ist bei Position 2, siehe Abb\ref{fig:puppe_2}, 
größer als bei Position 1, siehe Abb\ref{fig:puppe_1}. Der experimentelle Wert variiert allerdings bei der Variation der Position nur 
sehr geringfügig, im Vergleich zur Theorie. Dies ist durch das sehr hohe Eigenträgheitsmoment bedingt, was allgemein zu einer sehr hohen
Ungenauigkeit bei der Berechnung aller Trägheitsmomente führt. 

