\section{Auswertung}
\label{sec:Auswertung}

Als erstes wird die Winkelrichtgröße der Torsionsfeder berechnet.
Durch passendes Umstellen und Einsetzten der Gleichungen \eqref{eqn:T} und \eqref{eqn:M} ergibt sich die Gleichung
\begin{equation*}
    D=\frac{Fr}{\phi}.
    \end{equation*}
Mit Hilfe der Daten der Tabelle \ref{tab:1} errechnet sich die Winkelrichtgröße zu $D=\SI{0.022435(001026)}{\newton\m}$.
\begin{table}
    \centering 
    \caption{Daten zur Bestimmung der Winkelrichtgröße.}
    \label{tab:1}
    \begin{tabular}{S S}
        \toprule
        {$\phi$ in Grad} & {$F \:/\: \si{\newton}$} \\
        \midrule
        20 & 0.060 \\
        30 & 0.100 \\
        40 & 0.155 \\
        50 & 0.195 \\
        60 & 0.260 \\
        70 & 0.300 \\
        80 & 0.350 \\
        90 & 0.380 \\
        
        \bottomrule
    \end{tabular}
\end{table}


\begin{table}
    \centering 
    \caption{Daten zur Bestimmung des Eigenträgheitsmomentes.}
    \label{tab:2}
    \begin{tabular}{S S}
        \toprule
        $r \:/\: \si{\centi\meter}$ & $T \:/\: \si{\s}$ \\
        \midrule
        2.5 & 2.34  \\
        4.0 & 2.52    \\
        6.0 & 2.83    \\
        8.0 & 3.20    \\
        10.0 & 3.50   \\
        12.0 & 4.00   \\
        14.0 & 4.50   \\
        16.0 & 5.05   \\
        18.0 & 5.55   \\
        20.0 & 6.10   \\
        22.0 & 6.60   \\
        
        \bottomrule
    \end{tabular}
\end{table}

Das Eigenträgheitsmoment $I_D$ des Aufbaus lässt sich aus dem Gesamtträgheitsmoment 

    \begin{equation}
    I_{\text{ges}}=2\cdot m r^2 +I_D
    \label{eqn:Iges}
    \end{equation}

bestimmen. Dazu wird eine lineare Regression auf $T^2$ und $r^2$ aus der Tabelle \ref{tab:2} verwendet:

    \begin{equation*}
        T^2=a\cdot r^2+b
    \end{equation*}
Durch Einsetzten der Gleichung \eqref{eqn:Iges} in die quadrierte Form der Gleichung \eqref{eqn:T} folgt die Gleichung
    \begin{equation*}
        T^2=4\pi^2\frac{I}{D}=\frac{(2mr^2+I_D)4\pi^2}{D}=\frac{8m\pi^2}{D}r^2+\underbrace{4\pi^2I_D}_{= b}.
    \end{equation*}
Der Teil $b$ enthält das zu Bestimmende Eigenträgheitsmoment. Die Ausgleichsgerade aus der Abbildung [BILD]
liefert den Parameter $b$, womit sich schließlich über
    \begin{equation*}
        I_D=\frac{b}{4\pi^2}
    \end{equation*}
das Eigenträgheitsmoment des Aufbaus bestimmen lässt.


%Messwerte: Alle gemessenen physikalischen Größen sind übersichtlich darzustellen.
%
%Auswertung:
%Berechnung der geforderten Endergebnisse
%mit allen Zwischenrechnungen und Fehlerformeln, sodass die Rechnung nachvollziehbar ist.
%Eine kurze Erläuterung der Rechnungen (z.B. verwendete Programme)
%Graphische Darstellung der Ergebnisse