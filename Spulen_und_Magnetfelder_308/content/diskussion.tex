\section{Diskussion}
\label{sec:Diskussion}

\subsection{Hysteresekurve}
    In diesem Fall konnte, wie in Grafik \ref{fig:hysterese} zu 
    sehen ist, keine gut erkennbare Hysterekurve erreicht werden. 
    Grund dafür ist zumindest unter anderem, dass die durch die Hallsonde
    gemessenen Werte nicht eindeutig abzulesen waren. Der vom Messgerät angezeigte
    Wert schwankte dabei um bis zu 30\%, sodass jeweils ein grob abgeschätzter 
    Mittelwert der angezeigten Werte als endgültiger Messwert notiert werden musste.
    Der Grund für die uneindeutigen Messungen scheint ein Kalibrierungsfehler gewesen zu sein.
    %irgendwie sowas hat sie doch gesagt?
    Es scheint sich jedoch lediglich um eine sehr starke Ungenauigkeit zu handeln, denn wie 
    zu sehen ist, ist das Steigungsverhalten der Kurve grundsätzlich so, wie es vom Zusammenhang 
    zwischen $H$ und $B$ zu erwarten ist. Das $B$ Feld steigt mit steigendem $H$
    bis zu einem Sättigunswert $B_s$ \approx $1210\si{\milli\tesla}$ an. Nachdem $H$ wieder
    auf null reduziert wurde, ist der neu gemessene $B$ Wert bei $H$=0 um 80\si{\milli\tesla} 
    größer, als der ursprüngliche. Der $B$ Wert stieg allerdings ursprünglich in negative Richtung an,
    wodurch der zweite $B$ Wert bei $H$=0 durch die Restmagnetisierung kleiner als der erste sein müsste.
    Diese Abweichung von ist in diesem Fall auf die starke Messungenauigkeit zurückzuführen. Wie im Plot \ref{fig:hysterese} zu sehen ist,
    weichen auch nah am Sättigunswert die unterschiedlichen $B$ Werte ähnlich stark voneinander ab, was in der Theorie
    so nicht der Fall ist. Was in diesem Versuch also eigentlich zu zeigen ist, nämlich dass das $B$ Feld bei gleichem
    $H$ unterschiedliche Werte, die durch Restmagnetisierung beeinflusst sind, annimmt, kann in diesem Fall nicht gezeigt
    werden. Das einzige, was mit den hier vorliegenden Werten eindeutig zu erkennen ist, ist, dass die magnetische 
    Flussdichte $B$ nicht mit steigendem $H$ beliebig weiter steigen kann, sondern durch einen Sättigunswert $B_s$
    begrenzt ist. Dieser kommt dadurch zustande, dass es in einem begrenzten Volumen eines Ferromagneten nur eine begrenzten
    Anzahl Dipole gibt, die sich ausrichten können. Sind also theoretisch alle Dipole im Material in Richtung $H$
    ausgerichtet, so kann $B$ bei steigendem $H$ nur noch linear zu $H$ steigen. Diese Steigung ist
    bei der skalierung einer gewöhnlichen Hysterekurve extrem gering, da das Verhältnis von $H$ zu $B$ in der 
    Größenordnung 1:1000 liegt. %formulierung eventuell unklar??

    \subsection{einfache Spulen}
        Unter der Bedingung, dass Randeffekte zu vernachlässigen sind, besteht theoretisch im Inneren eines Solonoids ein
        homogenes Magnetfeld, welches durch \ref{eqn:bhom} beschrieben werden kann. Dies ist nur dann der Fall, wenn die Länge $l$
        der Spule groß genug ist, und der Bereich, in dem gemessen wird, weit genug von den Spulenenden entfernt ist, da sonst
        Randeffekte auftreten, was zu einem inhomogenen Feld führt. Dabei kam es bei den drei zur Verfügung stehenden Spulen zu
        foglenden Problemen. Die Länge $l$ der kurzen Spule ist zu kurz, damit Randeffekte ihre Wirkung im Inneren der Spule verlieren.
        Im Plot der kurzen Spulen ist diesbezüglich zu sehen, dass $B$ an keiner Stelle konstant ist, obweohl Messungen bis über die
        Mitte der Spule vorgenommen wurden. Der Verlauf des hier gemessenen $B$ Feldes entspricht etwa dem einer einzelnen Leiterschleife,
        bei der das Magnetfeld im Zentrum maximal ist und außerhalb in beide Richtungen abfällt. Der Abfall des $B$ Feldes kann dabei
        durch \ref{eqn:bvonx} beschieben werden. 

        Ein anderes Problem ergibt sich sowohl bei der langen, als auch bei der dicken Spule. Bei denen wäre zu erwarten, dass 
        das Problem der kurzen Spule zumindest nicht in diesem Ausmaß auftritt. Allerdings konnte hierbei, durch die Halterung begrenzt,
        die Hallsonde nicht ganz bis zur Hälfte der jeweiligen Spulenlänge in die Spule eingeführt werden. Wie im zugehörigen Plot
        zu erkennen ist, ist für keine der beiden Spulen sicher deren maximales Magnetfeld gemessen worden. Daher kann nicht sicher 
        gesagt werden, ob das Magnetfeld in einem gewissen $x$- Intervall auf der Symmetrieachse homogen ist, wie es die Theorie vorschreibt.
        Der Abfall des $B$ Feldes außerhalb der Spule ist bei beiden ähnlich, wie bei der kurzen Spule, und entspricht etwa dem Verlauf
        wie in \ref{eqn:bvonx}. 
        Im Endeffekt konnte hier nicht gezeigt werden, dass ein Magnetfeld eines Solonoids mit außreichender Länge $l$ auf dessen
        Symmetrieachse auf einem $x$- Intervall homogen ist. 

    \subsection{Helmholtzspule}
        Das Spulenpaar, dessen Abstand zueienander genau dem Spulenradius $R$= 6,25\si{\centi\meter} entspricht, sollte theoretisch innerhalb
        der beiden Spulen auf deren Symmetrieachse ein homogenes Magnetfeld aufweisen. Bei Spulenabständen, die von diesem Wert abweichen,
        ist dies in der Regel nicht der Fall. Wie in den Plots \ref{fig:H12} und \ref{fig:H18} zu sehen ist, zeigt hierbei das Magnetfeld
        im Inneren des Spulenpaares einen parabolischen Verlauf und ist damit inhomogen. 
        Bei den Messungen, bei denen der Spulenabstand $d$=$R$ gewählt wurde, scheint das Magnetfeld zwischen den Spulen homogen zu sein.
        Dies wäre besser zu sehen gewesen, wenn mehrere Messwerte hätten aufgenommen werden können, was durch den Aufbau nicht möglich war, 
        da die Halterung der Hallsonde sowie die Breite der jeweiligen Spulen den messbaren Bereich einschränkten. Wie in der Grafik \ref{fig:H6_1}
        zu sehen ist, weichen jedoch die diesbezüglich gemessenen $B$ Werte nur sehr geringfügig voneinander ab.
        Außerhalb des Helmholtzspulenpaares fällt das Magnetfeld fast unabhängig vom gewählten Spulenabstand mit steigender Entfernung ab, 
        wie es auch z.B. bei den einfachen Spulen der Fall ist.

