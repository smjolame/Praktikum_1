\section{Diskussion}
\label{sec:Diskussion}

\subsection{Hysterekurve}
    In diesem Fall konnte, wie in [referenz Plot, weiß nicht wie] zu 
    sehen ist, keine gut erkennbare Hysterekurve erreicht werden. 
    Grund dafür ist zumindest unter anderem, dass die durch die Hallsonde
    gemessenen Werte nicht eindeutig abzulesen waren. Der vom Messgerät angezeigte
    Werte schwankte dabei um bis zu 30\si{\per\cent}, sodass jeweils ein grob abgeschätzter 
    Mittelwert der angezeigten Werte als endgültiger Messwert notiert werden musste.
    Der Grund für die uneindeutigen Messungen scheint ein Kalibrierungsfehler gewesen zu sein
    %irgendwie sowas hat sie doch gesagt?
    Es scheint sich jedoch lediglich im eine sehr starke Ungenauigkeit zu handeln, denn wie 
    zu sehen ist, ist das Steigungsverhalten der Kurve grundsätzlich so, wie es vom Zusammenhang 
    zwischen $\vec H$ und $\vec B$ zu erwarten ist. Das $\vec B$ Feld steigt mit steigendem $\vec H$
    bis zu einem Sättigunswert $B_s$ \approx $-300\si{\milli\tesla}$ an. Nachdem $\vec H$ wieder
    auf null reduziert wurde, ist der neu gemessene $\vec B$ Wert bei $\vec H$=0 um 80\si{\milli\tesla} 
    größer, als der ursprüngliche. Der $\vec B$ Wert stieg allerdings ursprünglich in negative Richtung an,
    wodurch der zweite $\vec B$ Wert bei $\vec H$=0 durch die Restmagnetisierung kleiner als der erste sein müsste.
    Diese Abweichung von ist in diesem Fall auf die starke Messungenauigkeit zurückzuführen. Wie im [plot] zu sehen ist,
    weichen auch nah am Sättigunswert die unterschiedlichen $\vec B$ Werte ähnlich stark voneinander ab, was in der Theorie
    so nicht der Fall ist. Was in diesem Versuch also eigentlich zu zeigen ist, nämlich dass das $\vec B$ Feld bei gleichem
    $\vec H$ unterschiedliche Werte, die durch Restmagnetisierung beeinflusst sind, annimmt, kann in diesem Fall nicht gezeigt
    werden. Das einzige, was mit den hier vorliegenden Werten eindeutig zu erkennen ist, ist, dass die magnetische 
    Flussdichte $\vec B$ nicht mit steigendem $\vec H$ beliebig weiter steigen kann, sondern durch einen Sättigunswert $B_s$
    begrenzt ist. Dieser kommt dadurch zustande, dass es in einem begrenzten Volumen eines Ferromagneten nur eine begrenzten
    Anzahl Dipole gibt, die sich ausrichten können. Sind also theoretisch alle Dipole im Material in Richtung $\vec H$
    ausgerichtet, so kann $\vec B$ bei steigendem $\vec H$ nur noch ?proportional/linear? zu $\vec H$ steigen. Diese Steigung ist
    bei der skalierung einer gewöhnlichen Hysterekurve extrem gering, da das Verhältnis von $\vec H$ zu $\vec B$ in der 
    Größenordnung 1:1000 liegt. %formulierung eventuell unklar??
