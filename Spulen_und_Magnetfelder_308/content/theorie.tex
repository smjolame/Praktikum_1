\section{Theorie}
\label{sec:Theorie}





Durch bewegte elektrische Ladungen werden Magnetfelder induziert.
Im Allgemeinen kann ein Mgnetfeld über die beiden vektoriellen Größen 
$\vec H$ und $\vec B$ beschrieben werden. Dabei ist $\vec H$ die 
magnetische Feldstärke, die das durch den stromdurchflossenen Leiter induzierte 
Magnetfeld im Vakuum beschreibt. Dahingegen enthält die magnetische Flussdichte
$\vec B$ sowohl die Vakummpermeabilität $\mu_0$, als auch die relative Permeabilität
$\mu_r$. Der physikalische Zusammenhang zwischen $\vec H$ und $\vec B$ zeigt sich wie folgt:

\begin{equation}
    \vec B = \mu_0 \mu_r \vec H
\end{equation}

Die relative Permeabilität diamagnetischer Stoffe ist dabei $\mu_r$ < 1, für 
Paramagnetika ist $\mu_r$ > 1. Dabei weicht der Wert je nach Material teilweise 
nur sehr geringfügig vom Wert 1 ab. Bei Ferromagnetika ist $\mu_r$ $\gg$ 1. Des Weiteren
können deren Magnetfelder auch ohne den aktiven Einfluss bewegter, elektrischer Ladungen
bestehen bleiben, wie beim Teilversuch Hysteresekurve zu sehen ist. Dabei richten sich 
magnetische Dipole durch den Einfluss 
eines äußeren Magnetfelds in Richtung dessen Feldstärke $\vec H$ aus. Die dadurch erzeugte magnetische
Flussdichte $\vec B$ steigt dadurch mit steigendem $\vec H$ bis zu einem gewissen Sättigungswert $B_s$
an. Wird das äußere Magnetfeld abgeschaltet, so ist es die Eigenschaft ferromagnetischer Materialen, dass
deren Dipole teilweise ihre Ausrichtung beibehalten und so eine Restmagnetisierung $B_r$ vorhanden ist.
Dies führt dazu, dass für den gleichen Wert $\vec H$ unterschiedliche $\vec B$ Werte auftreten können, je 
nach vorheriger Ausrichtung der Dipole im Ferromagneten. 


Das Magnetfeld eines stromdurchflossenen Leiters kann allgemein durch das Biot-Savart Gesetz
beschrieben werden:

\begin{equation}
    \vec{B}(\vec r) = \frac{\mu_0 I'}{4\pi} \oint_{\Gamma'} \frac{\symup{d}\vec{l'}\times (\vec{r}-\vec{r'})}{\lvert\vec{r}-\vec{r'}\rvert^3}
\end{equation}

Einfacher ist das Magnetfeld im Inneren eines Solonoids der Länge $l$ 
mit der Windungszahl $N$. Hierbei wird das Magnetfeld im Inneren ohne
Berücksichtigung von Randeffekten als homogen angenommen und durch 

\begin{equation}
    B = \mu_0 \mu_r \frac{NI}l 
\end{equation}

beschrieben. Außerhalb der Spule verhält sich das Magnetfeld ...

Für das Magnetfeld eines Helmholtz-Spulenpaares, bei dem der Abstand der beiden 
Spulen deren Radius $R$ entspricht, gilt in dessen Zentrum:

\begin{equation}
    B = \frac{\mu_0 I R^2}{({R^2}+{x^2})^\frac{3}2}
\end{equation}


