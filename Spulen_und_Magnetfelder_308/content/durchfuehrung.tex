\section{Durchführung}
\label{sec:Durchführung}

    \subsection{Verschiedene Spulen}

        Es wird hierbei der gleiche Versuch für drei einzelne Spulen 
        unterschiedlicher Maße durchgeführt. Dafür wird ein Strom $I$=1 \si{\ampere}
        and die jeweiligen Spulen angelegt. Die lange Spule hat eine 
        Windungszahl $N_1$=300, eine Länge $l_1$=16,3 \si{\centi\meter} und 
        einen Druchmesser $d_1$=4,1 \si{\centi\meter}. Für die kurze Spule ist 
        $N_2$=100, $l_2$=5,3 \si{\centi\meter} und $d_2$=4,1 \si{\centi\meter}.
        Die Maße der dicken Spule sind $N_3$=400, $l_3$=9,2 \si{\centi\meter},
        $d_3$=[???]. Die x-Achse sei dabei die Symmetrieachse 
        der jeweiligen Spulen. Eine longitudinale Hallsonde wird entlang der x-Achse
        an eine Halterung angebracht. Die Spule wird in Richtung der Hallsonde
        entlang der x-Achse verschoben. Dabei wird alle 1 \si{\centi\meter} das von 
        der Hallsonde gemessene Magnetfeld abgelesen. Bei der ersten Messung ist das nähere 
        Ende der Spule 5 \si{\centi\meter} von der Spitze der Hallsonde entfernt und bei 
        der letzten Messung befindet sich die Hallsonde 5 \si{\centi\meter} im Inneren
        der Spule. Die Abstände werden durch ein Lineal gemessen, welches an der selben
        Halterung, wie die Hallsonde befestigt ist.

    \subsection{Helmholtz-Spule}
        
        Ein Helmholtz-Spulen-Paar besteht aus zwei identischen ?Spulen, die prallel zueinander
        in einem Abstand $d$ angeordnet sind.[werte]. Der Radius $R$ einer Spule ist 6,25 \si{\centi\meter}
        die x-Achse entspreche wieder der Symmetrieachse der beiden Spulen. Die Spulen werden dabei in
        Reihe geschaltet und es wird ein Strom $I$=5 \si{\ampere} bei einer Spannung von $U$=5,3 \si{\volt}
        angelegt. Eine transversale Hallsonde wird von oben so an eine Halterung angebracht, 
        dass sie das Magnetfeld auf 
        der x-Achse in deren Richtung misst. Der Versuch wird für drei unterschiedliche 
        Abstände $d_1$= 12,5 \si{\centi\meter} ; $d_2$= 6,25 \si{\centi\meter} ; 
        $d_3$ = 18,75 \si{\centi\meter} durchgeführt. Dabei werden Messungen innerhalb sowie außerhalb 
        des Spulenpaares vorgenommen. Für $d_2$ gibt es innerhalb der Spule weniger Messungen, was
        durch den Aufbau bedingt ist, da die Halterung der Hallsonde deren Beweglichkeit einschränkt.

    \subsection{Hysteresekurve}

        In diesem Versuch wird eine Toroidspule ($N$=595; Torosdurchmesser $d_t$=37 \si{\centi\meter}) mit einem ?Eisenkern
        verwendet. Diese hat an einer Stelle einen Luftspalt der Breite $d_l$=3 \si{\milli\meter}, um in 
        dessen Zentrum mit einer transversalen Hallsonde die magnetische Flussdichte der Spule messen zu können.
        Mit einem ??? kann dann die Stromstärke $I$ und somit die  magnetische Feldstärke der Spule 
        variiert werden und die dementsprechende Flussdichte jeweils über die Hallsonde abgelesen werden.
        Für die erste Messung der Flussdichte $\vec B$ wird $I$=0 \si{\ampere} und darüber $\vec H$=0 \si{\tesla}
        eingestellt. Dann wird in  $I$ in 1 \si{\ampere}-Schritten bis auf 10 \si{\ampere} erhöht und jeweils $\vec B$
        gemessen. In gleichbleibenden Schritten mit zugehörigen Messungen wird $I$ wieder auf 0 \si{\ampere}
        reduziert. Der an der Spule anliegende Strom wird nun umgepolt und alle Messungen werden
        wiederholt, bis am Ende wieder $I$=0 \si{\ampere} ist.

        
        

        

        

