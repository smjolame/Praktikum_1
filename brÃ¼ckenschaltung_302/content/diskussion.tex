\section{Diskussion}
\label{sec:Diskussion}

Grundsätzlich fällt auf, dass bei allen über die Messwerte berechneten Größen
eine relativ große Standardabweichung zu sehen ist. Bei der Wheaton Brücke liegt diese
Beispielsweise bei etwa 10\%. Dies ist auch die Größenordnung für die Restlichen Teilversuche,
wobei manche Größen, wie z.B. die durch die Kapazitätsmessbrücke errechnete Kapazität $C_x$,
sogar eine Standardabweichung von über 30\% haben. Allgemein sind alle hier zu bestimmenden
Werte sehr ungenau. Ein Bauteil, dessen Wert hier berechnet wurde, wäre äußerst unpraktikabel,
in einem neuen Versuchsaufbau zu verwenden, da der dann zu messende Wert durch die hohen Standardabweichungen
des Bauteils deutlich an Präzision verlieren würde. 

In jeder der Brückenschaltungen wurde, unabhänging vom speziellen Aufbau, das Potentiometer $\frac{R_3}{R_4}$,
der selbe Oszillograph und der selbe Generator verwendet. Da auffällig ist, dass bei allen Versuchen die Abweichungen
sehr groß sind, wäre es vorstellbar, dass eines der eben aufgezählten Bauteile beschädigt ist. Das Netzteil scheint an 
dieser Stelle nicht der dafür verantwortliche Faktor zu sein, da die davon ausgehende Speisespannung $U_S$ zwischendurch 
gemessen wurde und jedes Mal der eingestellten Spannung von 2,5\si{\volt} entsprach. Um dem Ursprung des Fehlers genauer 
ausmachen zu können, liegen hier nicht genug Informationen vor, ein oder mehrere Defekte oder ungenaue Komponetenten in 
den jeweiligen Aufbauten sind daher die Annahme.

Bei der Auswertung der Wien-Robinson-Brücke \ref{fig:wien} ist beim Vergleich der Messwerte mit der Theoriekurve zu sehen,
dass die Werte in der Nähe des Minimums relativ gut mit der Theorie übereinstimmen. Das gemessene Minimum weicht hier recht stark
von null ab, was unter anderem daran liegen könnte, dass in dieser Umgebung die Messwerte nicht kleinschrittig genug aufgenommen 
wurden, und somit das optimal messbare Minimum übersprungen wurde. Damit wurde auch die Klirrfaktorbestimmung ungenau.
Bei einer höheren Anzahl an Messwerten in der Nähe des Minimums hätte eine genauere, und wahrscheinlich etwas niedrigere, Minimalfreqenz
ermittelt werden können, wodurch der Klirrfaktor ebenfallss niedriger ausfallen würde.  Es fällt außerdem auf, dass Messwerte ab einem bestimmten 
Punkt rechts vom Minimum wieder abfallen, obwohl sie laut Theorie konstant bleiben müssten. Dies könnte an den Streukapazitäten der Bauteile und des Oszillographen liegen, 
welche besonders bei hohen Frequenzen an Gewicht erhalten.

%Kurze Zusammenfassung der Ergebnisse
%-Vergleich mit Literaturwerten
%-Vergleich mit verschiedenen Messverfahren
%-bei Abweichungen mögliche Ursachen finden