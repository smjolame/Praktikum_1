\section{Diskussion}
\label{sec:Diskussion}

Die Abweichungen der einzelnen Materialien belaufen sich auf $\SI{12.84(642)}{\percent}$ für Messing, $\SI{15.00(688)}{\percent}$ für Edelstahl und $\SI{1.46(537)}{\percent}$ für Aluminium \cite{tool}. Es ist zu erkennen, dass die Wärmeleitfähigkeiten für Messing und Edelstahl relativ stark von den Theoriewerten abweichen. Dies kann an der geringen Anzahl an gemessenen Perioden liegen, da auf Grund eines Messfehlers innerhalb des Versuches weniger aussagekräftigen Wellenmaxima gemessenen werden konnten. 

Generell liegen die gemessenen Werte aber recht nah an den Theoriewerten. Vor allem die Wärmeleitfähigkeit von Aluminium konnte ziemlich genau bestimmt werden. Bei akribischerem Durchführen des Versuches, z.B durch die Messung über eine größere Periodenanzahl, könnten die Abweichungen weiter vermindert werden. 

\begin{table}[H]
    \centering
    \caption{Vergeleich zwischen Theorie- und Praxiswerten.}
    \begin{tabular}{c|S S S }
        \toprule
        \text{Material} & {$\kappa_\text{exp} \:/\: \si{\watt\per\meter\kelvin}$} & {$\kappa_\text{th} \:/\: \si{\watt\per\meter\kelvin}$} & {\text{Abweichung}} \\
        \midrule
        \text{Messing}   & \num{95(7)   } &  109   &   \SI{12.84(642)}{\percent}  \\
        \text{Edelstahl} & \num{13.6(11) }    &  16    &  \SI{15.00(688)}{\percent}  \\
        \text{Aluminium} & \num{202(11) }   &  205   &     \SI{1.46(537)}{\percent}  \\
        \bottomrule 
    \end{tabular}
\end{table}


%Kurze Zusammenfassung der Ergebnisse
%-Vergleich mit Literaturwerten
%-Vergleich mit verschiedenen Messverfahren
%-bei Abweichungen mögliche Ursachen finden