\section{Auswertung}
\label{sec:Auswertung}


\subsection{Statische Methode}
\begin{align*}
T_1(t=385) =& 46.50 \si{\celsius}\\
T_4(t=385) =& 44.13 \si{\celsius}\\
T_5(t=385) =& 49.71 \si{\celsius}\\
T_8(t=385) =& 33.94 \si{\celsius}
\end{align*}

Es ist zu sehen, dass die Temperatur des Aluminiumstabes am höchsten ist. Da alle Thermometer den gleichen Abstand
zum Heizkörper haben und deren anfängliche Temperaturen bis auf geringe Abweichungen gleich sind, kann daraus 
geschlossen werden, dass Aluminium die höchste Wärmeleitfähigkeit hat.





\subsection{Dynamische Methode}

%Messwerte: Alle gemessenen physikalischen Größen sind übersichtlich darzustellen.
%
%Auswertung:
%Berechnung der geforderten Endergebnisse
%mit allen Zwischenrechnungen und Fehlerformeln, sodass die Rechnung nachvollziehbar ist.
%Eine kurze Erläuterung der Rechnungen (z.B. verwendete Programme)
%Graphische Darstellung der Ergebnisse