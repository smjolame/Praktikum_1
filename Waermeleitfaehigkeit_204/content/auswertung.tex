\section{Auswertung}
\label{sec:Auswertung}


\subsection{Statische Methode}
In den folgenden Plots werden jeweils zwei Temperaturverläufe verglichen. In Abbildung \ref{fig:t1t4} wird der breite und der schmale 
Messingstab untersucht, in Plot\ref{fig:t5t8} werden Aluminium und Edelstahl auf ihre Temperaturen untersucht. Die Temperaturen werden hier jeweils
an den vom Thermoelement weiter entfernten Thermometern gemessen.

\begin{figure}[H]
    \centering
    \includegraphics[width=\textwidth]{build/T1T4.pdf}
    \caption{Temperaturverläufe der Messingstäbe.}
    \label{fig:t1t4}
\end{figure}

\begin{figure}[H]
    \centering
    \includegraphics[width=\textwidth]{build/T5T8.pdf}
    \caption{Temperaturverläufe von Aluminium und Edelstahl.}
    \label{fig:t5t8}
\end{figure}

In der Auflistung stehen die in den Plots gezeigten Temperaturen zu Ende der statischen Messung:
\begin{align*}
T_1(t=385) =& 46.50 \si{\celsius}\\
T_4(t=385) =& 44.13 \si{\celsius}\\
T_5(t=385) =& 49.71 \si{\celsius}\\
T_8(t=385) =& 33.94 \si{\celsius}
\end{align*}

%Es ist zu sehen, dass die Temperatur des Aluminiumstabes am höchsten ist. Da alle Thermometer den gleichen Abstand
%zum Heizkörper haben und deren anfängliche Temperaturen bis auf geringe Abweichungen gleich sind, kann daraus 
%geschlossen werden, dass Aluminium die höchste Wärmeleitfähigkeit hat.

Zur Berechnung des Wärmestroms wird die Querschnittsfläche $A$ der einzelnen Stäbe aus der Versuchsanleitung entnommen.
Die Wärmeleitfähigkeiten $\kappa$ sind aus der Literatur\cite{V204} übernommen und die Entfernung $\Delta x$ der jeweiligen $T$ 
werden abgemessen.

\begin{align*}
\kappa _\text{Messing}& = 109\si{\watt\per\meter\kelvin} \\
A_\text{Messing,breit}& = 48\cdot 10^{-6}\si{\meter\squared} \\
\Delta x_\text{Messing,breit}& = 0,03\si{\meter} \\
\kappa _\text{Edelstahl}& = 16\si{\watt\per\meter\kelvin} \\
A_\text{Edelstahl}& = 48\cdot 10^{-6}\si{\meter\squared} \\
\Delta x_\text{Edelstahl}& = 0,03\si{\meter} \\
\kappa_\text{Aluminium}& = 205 \si{\watt\per\meter\kelvin}
\end{align*}

\begin{table}[H]
    \centering
    \caption{Wäremestrom von Messing und Edelstahl.}
    \label{tab:deltaq}
    \begin{tabular}{S S S S S}
        \toprule
        {$t\:/\:\si{s}$} & {$\Delta T_{21}\:/\:\si{\kelvin}$} & {$\frac{\text{d}Q_{21}}{\text{d}t}\:/\:\si{\watt}$} & 
        {$\Delta T_{78}\:/\:\si{\kelvin}$} & {$\frac{\text{d}Q_{78}}{\text{d}t}\:/\:\si{\watt}$} \\
        \midrule
        75 & 7.11 & 1.23998 & 12.65 & 0.32384 \\
        150 & 4.80 & 0.83712 & 13.52 & 0.34611 \\
        225 & 3.55 & 0.61912 & 12.57 & 0.32179 \\
        300 & 2.93 & 0.51099 & 11.77 & 0.30131 \\
        375 & 2.61 & 0.45518 & 11.21 & 0.28698 \\
        \bottomrule 
    \end{tabular}
\end{table}
Der folgende Plot vergleicht die Temperaturunterschiede der nahen und fernen Thermometer des jeweils gleichen Stabs. Dieser 
Temperaturunterschied ist hier für Messing und Edelstahl gezeigt.

\begin{figure}[H]
    \centering
    \includegraphics[width=\textwidth]{build/deltaT.pdf}
    \caption{Temperaturunterschied innerhalb eines Stabes für Messing und Edelstahl.}
    \label{fig:deltat}
\end{figure}


\subsection{Dynamische Methode}

Im Folgenden wird anhand der Temperaturamplituden $A_i$ und der Phasendifferenzen $\Delta t$ die Wärmeleitfähigkeit $\kappa$
der verschiedenen Metalle berechnet. Für Messing und Aluminium wird dafür die Messung mit 80$\si{\s}$ Periodendauer 
und für Edelstahl die Messung mit 200$\si{\s}$ Periodendauer verwendet.
Die Amplituden, sowie die Phasendifferenzen können aus den Messwerten abgelesen werden. Damit kann nach Gleichung\ref{eqn:kappa}
jeweils $\kappa$ berechnet werden.

\begin{table}[H]
    \centering
    \caption{Amplituden und Phasendifferenz der Temperaturverläufe von Messing.}
    \label{tab:amp_m}
    \begin{tabular}{S S S}
        \toprule
        {$A_1\:/\:\si{\kelvin}$} & {$A_2\:/\:\si{\kelvin}$} & {$\Delta t\:/\:\si{\s}$} \\
        \midrule
        4.585 & 12.890 & 16 \\
        4.245 & 12.705 & 14 \\
        4.070 & 12.555 & 13 \\
        \bottomrule
    \end{tabular}
\end{table}

\begin{figure}[H]
    \centering
    \includegraphics[width=\textwidth]{build/T1T2.pdf}
    \caption{Temperaturunterschied innerhalb des Messingstabes.}
    \label{fig:t1t2}
\end{figure}

Die Konstanten für Messing sind:
\begin{align*}
    \rho& = 8520\:\si{\kilo\gram\per\meter\cubed} \\
    c& = 385\:\si{\joule\per\kilo\gram\kelvin}
\end{align*}

Daraus ergibt sich mit der Gleichung\ref{eqn:kappa} für die Wärmeleitfähigkeit:
\begin{equation*}
    \kappa_{\text{Messing}} = \SI{95(7)}{\watt\per\meter\kelvin}
\end{equation*}   

\begin{table}[H]
    \centering
    \caption{Amplituden und Phasendifferenz der Temperaturverläufe von Aluminium.}
    \label{tab:amp_a}
    \begin{tabular}{S S S}
        \toprule
        {$A_5\:/\:\si{\kelvin}$} & {$A_6\:/\:\si{\kelvin}$} & {$\Delta t\:/\:\si{\s}$} \\
        \midrule
        8.060 & 15.955 & 8 \\
        7.635 & 15.600 & 7 \\
        7.500 & 15.515 & 7 \\
        \bottomrule
    \end{tabular}
\end{table}

\begin{figure}[H]
    \centering
    \includegraphics[width=\textwidth]{build/T5T6.pdf}
    \caption{Temperaturunterschied innerhalb des Aluminiumstabes.}
    \label{fig:t5t6}
\end{figure}

Die Konstanten für Aluminium sind:
\begin{align*}
    \rho& = 2800\:\si{\kilo\gram\per\meter\cubed} \\
    c& = 850\:\si{\joule\per\kilo\gram\kelvin}
\end{align*}

Daraus ergibt sich für die Wärmeleitfähigkeit:
\begin{equation*}
    \kappa_{\text{Aluminium}} = \SI{202(11)}{\watt\per\meter\kelvin}
\end{equation*}

\begin{table}[H]
    \centering
    \caption{Amplituden und Phasendifferenz der Temperaturverläufe von Edelstahl.}
    \label{tab:amp_e}
    \begin{tabular}{S S S }
        \toprule
        {$A_7\:/\:\si{\kelvin}$} & {$A_8\:/\:\si{\kelvin}$} & {$\Delta t\:/\:\si{\s}$} \\
        \midrule
        19.575 & 3.110 & 66 \\
        19.410 & 2.920 & 60 \\
        19.515 & 2.955 & 46 \\
        19.040 & 2.610 & 51 \\
        \bottomrule
    \end{tabular}
\end{table}

\begin{figure}[H]
    \centering
    \includegraphics[width=\textwidth]{build/T7T8.pdf}
    \caption{Temperaturunterschied innerhalb des Edelstahlstabes.}
    \label{fig:t7t8}
\end{figure}

Die konstanten für Edelstahl sind:
\begin{align*}
    \rho& = 8000\:\si{\kilo\gram\per\meter\cubed} \\
    c& = 400\:\si{\joule\per\kilo\gram\kelvin}
\end{align*}

Daraus ergibt sich für die Wärmeleitfähigkeit:
\begin{equation*}
    \kappa = \SI{13.6(11)}{\watt\per\meter\kelvin}
\end{equation*}
