\section{Theorie}
\label{sec:Theorie}

Bei einem Temperaturunterschied innerhalb eines Materials strebt das System einen thermischen Gleichgewichtszustand an.
Dabei kann die Wärme auf drei verschiedenen Arten übertragen werden:
\begin{itemize}
    \item Wärmestrahlung
    \item Konvektion
    \item Wärmeleitung
\end{itemize}

In diesem Versuch wird  nur die Wärmeleitung betrachtet. 
Die Wärmemenge d$Q$,die in einem ungleichmäßig temperierten Stab in der Zeit d$t$ durch die Querschnittsfläche $A$ fließt lässt sich mit der Wämrmestromdichte 
\begin{equation}
    j_{\omega}=-\kappa \frac{\partial T}{\partial x},
\end{equation}
über die Gleichung
\begin{equation}
    \symup{d}Q=j_{\omega}A\symup{d}t
\end{equation}
bestimmen.
Dabei ist $\kappa$ die materialabhängige Wärmeleitfähigkeit und nach Konvention, da Wärme immer von höherer zu tieferer Temperatur fließt, mit einem negativen Vorzeichen versehen.


Da die Gesamtenergie in einem abgeschlossenem System stets konstant bleiben muss, lässt sich mit Hilfe dieser Kontinuitätsbedingung die, für diesen Versuch eindimensionale, Wärmeleitungsgleichung  
\begin{equation}
        \frac{\partial T}{\partial t}=\underbrace{\frac{\kappa}{\rho c}}_{\sigma_T}\frac{\partial^2 T}{\partial x^2}
\end{equation}
aufstellen.
$\rho$ stellt hierbei die Massendichte und $c$ die spezifische Wärmekapazität des Materials dar.
$\sigma_T$ wird Temperaturleitfähigkeit genannt. 

Bei einem sehr langem Stab treten bei periodischem Erhitzen und Abkühlen Temperaturwellen auf, welche sich über die Gleichung
\begin{equation}
    T(x,t)=T_{\text{max}}e^{-\sqrt{\frac{\omega \rho c}{2 \kappa}}x}\cos \left(\omega t-\sqrt{\frac{\omega \rho c}{2 \kappa}}x\right)
\end{equation}
darstellen lassen.
Die Phasengeschwindigkeit bestimmt sich somit zu
\begin{equation}
    v=\frac{\omega}{k}=\omega / \sqrt{\frac{\omega \rho c}{2 \kappa}}=\sqrt{\frac{2 \kappa \omega}{\rho c}}.
\end{equation}

Letztendlich kann die Wärmeleitfähigkeit $\kappa$, bei Beachtung des Amplitudenverhältnisses zwischen $A_{\text{nah}}$ und $A_{\text{fern}}$ der Welle an den Orten $x_{\text{nah}}$ und $x_{\text{fern}}$, durch die Gleichung
\begin{equation}
    \kappa =\frac{\rho c (\Delta x)^2}{2 \Delta t \ln \left(A_{\text{nah}}/A_{\text{fern}}\right)}
\label{eqn:kappa}
\end{equation}
errechnet werden.
$\Delta x$ ist der Abstand zwischen den beiden Messstellen und $\Delta t$ bezeichnet die Phasendifferenz der Temperaturwellen zwischen den Messstellen.
%In knapper Form sind die physikalischen Grundlagen des Versuches, des Messverfahrens, sowie sämtliche für die Auswertung erforderlichen Gleichungen darzustellen. (Keine Herleitung)

%(eventuell die Aufgaben)

%Der Versuchsaufbau: Beschreibung des Versuchs und der Funktionsweise (mit Skizze/Bild/Foto)
