\section{Diskussion}
\label{sec:Diskussion}

\subsection{Einseitige Einspannung}

Bei der einseitigen Einspannung lässt sich für den Kupferstab eine experimentelle Abweichung bezüglich des Theoriewertes von $\SI{3.28(28)}{\percent}$ und beim Aluminiumstab eine Abweichung von $\SI{0.55(28)}{\percent}$ feststellen. Somit weichen beide Werte innerhalb der Fehlertoleranz vom theoretischen Wert ab. Die Abweichungen können eventuell auf die fehlende Genauigkeit der Messuhr zurückgeführt werden. Bei dieser
war während der Durchführung auffällig, dass ein leichtes Klopfen auf die Standfüße des Aufbaus schon eine relevante
Veränderung des auf der Messuhr angezeigten Wert bewirkt. 
Dennoch sind die Abweichung beider bestimmter Elastizitätsmodule relativ nah am Theoriewert, wodurch trotz der anfälligen Messuhren, ein für die einseitige Einspannung aussagekräftiges Ergebnis enstanden ist.







\subsection{Beidseitige Auflage}

Die Abweichung zu den Theoriewerten belaufen sich bei der beidseitigen Auflage auf $\SI{4.1(2)}{\percent}$ für die Kupferstange und $\SI{9.9(25)}{\percent}$ für die Aluminiumstange. Die Abweichung ist erheblich größer als bei der Aluminiumstange bei einseitiger Einspannung. Wieder liegen die Theoriewerte außerhalb der Fehlerbereiche. Auch hier spielt die vorher erwähnte Ungenauigkeit der Messuhren eine Rolle. Jedoch wird die große Abweichung eher an dem Umstand liegen, jener sich die Stangen trotz höherem Gewichtes, weniger durchbiegen als bei der einseitigen Einspannung. Dadurch wird der Fehler der Messuhren noch signifikanter. Außerdem führen Krümmungen der nicht perfekt geraden Stangen bei geringer Biegung ebenfalls zu größeren Fehlern. Dadurch sind auch die Unterschiede zwischen den Elastizitätsmodulen der jeweils linken und rechten Messung und Berechnung zu erklären. 


Bei gesamter Betrachtung beider Methoden führen die gemessenen Werte in Richtung der Theoriewerte. Die einmal hohe und einmal niedrige Abweichung des Elastizitätsmodules der Aluminiumstange, jedoch relativ konstante Abweichung des Elastizitätsmodules der Kupferstange, lässt auf einen eher statistischen, als systematsichen Fehler schließen. Bei mehr Wiederholungen des Versuchs sollten dadurch im Mittel weniger Abweichende Elastizitätsmodule bestimmt werden können.





