\section{Diskussion}
\label{sec:Diskussion}

\subsection{Einseitige Einspannung}

Wie in der Auswertung im plot\ref{fig:1mw} zu sehen ist, stimmen die Messwerte fast exakt mit der Theoriekurve überein.
Die minimale Abweichung kann eventuell auf die fehlende Genauigkeit der Messuhr zurückgeführt werden. Bei dieser
war während der Durchführung auffällig, dass ein leichtes Klopfen auf die Standfüße des Aufbaus schon eine relevante
Veränderung des auf der Messuhr angezeigten Wert bewirkt. Insgesamt scheint die Messung für beide Metallstäbe
bei einseitiger Einspannung aber sehr genau gewesen zu sein, da keine relevante Abweichung von der Theorie zu sehen ist.

\subsection{Beidseitige Auflage}

Hierbei ist auffällig, dass die Werte rechts und links vom Zentrum etwas voneinander abweichen (maximal etwa 0,1$\si{\milli\meter}$), 
wobei die Theoriekurve 
natürlich auf beiden Seiten identisch ist. Die vorher Beschriebene Ungenauigkeit der Messuhr kann auch hier teilweise für die Abweichung
von der THeorie verantwortlich sein.
Darüber hinaus ist auch zu beobachten, dass der Unterschied der rechten und linken Messwerte zueienander bei der Kupferstange größer ist, als bei der 
Alustange. Eine Krümmung oder ein Knick in der Stange kann zu solchen Unterschieden führen. 
Allerdings zeigt die Theoriekurve einen sehr ähnlichen Verlauf, auch wenn die Werte jeweils leicht abweichen.
Bei der Alustange sind die Messwerte rechts und links nur in der Nähe des Zentrums recht stark verschieden. Nach außen hin 
stimmt die Auslenkung rechts jedoch sehr gut mit der auf der linken Seite überein. Die Theoriekurve zeigt auch einen fast identischen
Verlauf, allerdings mit etwas geringeren Auslenkungen an jeder Stelle. Die maximale Abweichung vom Theoriewert beträgt dabei jedoch 
nur etwa 0,1$\si{\milli\meter}$. 

Insgesamt führten die Messungen zu sehr genau berechneten Elastizitätsmodulen, siehe Tabelle\ref{tab:zsmfassung}. Der zugehörige
Theoriewert weicht jeweils nur um weniger als 1$\si{\giga\pascal}$ vom experimentell errechneten Wert ab.



