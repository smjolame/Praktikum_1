\section{Auswertung}
\label{sec:Auswertung}

Bei dem Versuch wurden zwei verschiedene Stäbe auf ihr jeweiliges Elastizitätsmodul untersucht.
Beim vermessen und wiegen der beiden Stäbe konnten folgende Dichten ermittelt werden.
Der quaderförmige Stab hat eine Dichte von ca. $\rho_Q = 2.9350 \si{\gram\per\cubic\centi\meter}$
und der zylindrische Stab eine Dichte von $\rho_Z = 8.8554 \si{\gram\per\cubic\centi\meter}$.
Beim Vergleich mit Dichten von verschiedenen Metallen kann man mit relativer Exaktheit auf das Material
der Stäbe schließen. Dabei ist es nie sicher,ob es sich wirklich um das Vermutete Material handelt, 
da die Probestäbe Legierungen und inhomogone Massenverteilung aufweisen könnten,
welche in diesem Versuch nicht untersucht werden konnten. So wird unter diesen Annahmen und Annäherungen der 
quaderförmige Stab als aus Aluminium (Dichte Aluminium: 2.70 \si{\gram\per\cubic\centi\meter}\cite{taschenbuch_physik}) bestehend
 und der zylindrische Stab als aus Kupfer (Dichte Kupfer: 8.93 \si{\gram\per\cubic\centi\meter}\cite{taschenbuch_physik}) bestehend angenommen und im weiteren Verlauf des Protokolls auch so identifiziert und unterschieden.



Die Durchbiegung $D$ ist die Differenz zwischen der Durchbiegung $D_0$ im unbelasteten Zustand und der
Durchbiegung $D_M$ im belasteten Zustand.



\subsection{Biegung bei einseitiger Einspannung}
\begin{table}
    \centering 
    \caption{Durchbiegung des runden Kupferstabes bei einseitiger Einspannung}
    \sisetup{table-format=2.2}
    \label{tab:1}
    \begin{tabular}{S[table-format=2] S S S}
        \toprule
        {$x\:/\: \si{\centi\m}$} & {$D_0\:/\: \si{\milli\m}$} & {$D_M\:/\: \si{\milli\m}$} & {$D\:/\: \si{\milli\m}$ }\\
        \midrule
        3.00 & 10.16 & 10.10 & 0.06 \\
        6.00 & 10.07 & 9.87 & 0.20 \\
        9.00 & 9.92 & 9.53 & 0.39 \\
        12.00 & 9.70 & 9.05 & 0.65 \\
        15.00 & 9.47 & 8.49 & 0.98 \\
        18.00 & 9.23 & 7.86 & 1.37 \\
        21.00 & 9.06 & 7.26 & 1.80 \\
        24.00 & 8.95 & 6.67 & 2.28 \\
        27.00 & 8.84 & 6.06 & 2.78 \\
        30.00 & 8.76 & 5.42 & 3.34 \\
        33.00 & 8.65 & 4.72 & 3.93 \\
        36.00 & 8.54 & 4.00 & 4.54 \\
        39.00 & 8.41 & 3.25 & 5.16 \\
        42.00 & 8.23 & 2.46 & 5.77 \\
        45.00 & 8.05 & 1.64 & 6.41 \\
        
        \bottomrule
    \end{tabular}
\end{table}

\begin{figure}
    \centering
    \includegraphics[width=\textwidth]{build/ein_k.pdf}
    \caption{Ausgleichskurve und Messdaten des einseitig eingespannten Kupferstabes}
\end{figure}

Bei einseitiger Einspannung lässt sich über eine Ausgleichsrechnung auf Basis der Gleichung [GLEICHUNG] und den Daten der Tabelle \ref{tab:1}
das Elastizitätsmodul bestimmen.
Dabei ergibt sich für den Kupferstab ein Elastizitätsmodul von \SI{125.4255(3936)}{\giga\pascal}.

\begin{table}
    \centering 
    \caption{Durchbiegung des eckigen Aluminiumstabes bei einseitiger Einspannung}
    \sisetup{table-format=2.2}
    \label{tab:2}
    \begin{tabular}{S[table-format=2] S S S}
        \toprule
        {$x\:/\: \si{\centi\m}$} & {$D_0\:/\: \si{\milli\m}$} & {$D_M\:/\: \si{\milli\m}$} & {$D\:/\: \si{\milli\m}$ }\\
        \midrule
        3.00 & 9.99 & 9.86 & 0.13 \\
        6.00 & 9.94 & 9.72 & 0.22 \\
        9.00 & 9.81 & 9.40 & 0.41 \\
        12.00 & 9.65 & 8.95 & 0.70 \\
        15.00 & 9.48 & 8.43 & 1.05 \\
        18.00 & 9.27 & 7.81 & 1.46 \\
        21.00 & 9.06 & 7.15 & 1.91 \\
        24.00 & 8.85 & 6.45 & 2.40 \\
        27.00 & 8.65 & 5.69 & 2.96 \\
        30.00 & 8.46 & 4.92 & 3.54 \\
        33.00 & 8.25 & 4.11 & 4.14 \\
        36.00 & 8.06 & 3.28 & 4.78 \\
        39.00 & 7.83 & 2.42 & 5.41 \\
        42.00 & 7.61 & 1.54 & 6.07 \\
        45.00 & 7.39 & 0.63 & 6.76 \\
        
        \bottomrule
    \end{tabular}
\end{table}

\begin{figure}
    \centering
    \includegraphics[width=\textwidth]{build/ein_a.pdf}
    \caption{Ausgleichskurve und Messdaten des einseitig eingespannten Aluminiumstabes}
\end{figure}

Mit Hilfe des gleichen Verfahrens wie beim Kupferstab und der Tabelle \ref{tab:2} ergibt sich das Elastizitätsmodul des Aluminiumstabes zu \SI{70.1001(2918)}{\giga\pascal}.


\subsection{Biegung bei beidseitiger Auflage}

\begin{table}
    \centering 
    \caption{Durchbiegung des runden Kupferstabes bei beidseitiger Auflage}
    \sisetup{table-format=2.2}
    \label{tab:3}
    \begin{tabular}{S[table-format=2.2] S S S}
        \toprule
        {$x\:/\: \si{\centi\m}$} & {$D_0\:/\: \si{\milli\m}$} & {$D_M\:/\: \si{\milli\m}$} & {$D\:/\: \si{\milli\m}$ }\\
        \midrule
        15.50 & 8.86 & 6.84 & 2.02 \\
        16.50 & 8.78 & 6.66 & 2.12 \\
        17.50 & 8.70 & 6.50 & 2.20 \\
        18.50 & 8.62 & 6.32 & 2.30 \\
        19.50 & 8.56 & 6.16 & 2.40 \\
        20.50 & 8.48 & 6.00 & 2.48 \\
        21.50 & 8.39 & 5.86 & 2.53 \\
        22.50 & 8.32 & 5.74 & 2.58 \\
        23.50 & 8.23 & 5.63 & 2.60 \\
        24.50 & 8.18 & 5.54 & 2.64 \\
        30.50 & 8.10 & 5.50 & 2.60 \\
        31.50 & 8.17 & 5.66 & 2.51 \\
        32.50 & 8.25 & 5.78 & 2.47 \\
        33.50 & 8.33 & 5.92 & 2.41 \\
        34.50 & 8.42 & 6.06 & 2.36 \\
        35.50 & 8.50 & 6.23 & 2.27 \\
        36.50 & 8.59 & 6.40 & 2.19 \\
        37.50 & 8.68 & 6.56 & 2.12 \\
        38.50 & 8.79 & 6.73 & 2.06 \\
        39.50 & 8.85 & 6.91 & 1.94 \\

        
        \bottomrule
    \end{tabular}
\end{table}

\begin{figure}
    \centering
    \includegraphics[width=\textwidth]{build/bei_k.pdf}
    \caption{Ausgleichskurve und Messdaten des beidseitig aufliegenden Kupferstabes}
\end{figure}

Mit Hilfe der Gleichung [GLEICHUNG] für die linke Hälfte des Stabes und der Gleichung [GLEICHUNG] für die rechte Hälfte des Stabes
kann man mit den Werten der Tabelle \ref{tab:3} das Elastizitätsmodul des Kupferstabes bestimmen. Dieser ergibt sich 
auf der linken Hälfte zu \SI{121.3312(2693)}{\giga\pascal} und auf der rechten Hälfte zu \SI{126.2855(4137)}{\giga\pascal}.
Im Mittel ergibt sich also ein Wert von \SI{123.8083(3415)}{\giga\pascal}.





\begin{table}
    \centering 
    \caption{Durchbiegung des eckigen Aluminiumstabes bei beidseitiger Auflage}
    \sisetup{table-format=2.2}
    \label{tab:4}
    \begin{tabular}{S[table-format=2.2] S S S}
        \toprule
        {$x\:/\: \si{\centi\m}$} & {$D_0\:/\: \si{\milli\m}$} & {$D_M\:/\: \si{\milli\m}$} & {$D\:/\: \si{\milli\m}$ }\\
        \midrule
        15.50 & 9.40 & 7.28 & 2.12 \\
        16.50 & 9.38 & 7.18 & 2.20 \\
        17.50 & 9.36 & 7.07 & 2.29 \\
        18.50 & 9.35 & 6.96 & 2.39 \\
        19.50 & 9.32 & 6.88 & 2.44 \\
        20.50 & 9.31 & 6.79 & 2.52 \\
        21.50 & 9.30 & 6.71 & 2.59 \\
        22.50 & 9.26 & 6.66 & 2.60 \\
        23.50 & 9.25 & 6.62 & 2.63 \\
        24.50 & 9.20 & 6.57 & 2.63 \\
        30.50 & 9.07 & 6.39 & 2.68 \\
        31.50 & 9.07 & 6.42 & 2.65 \\
        32.50 & 9.05 & 6.45 & 2.60 \\
        33.50 & 9.06 & 6.51 & 2.55 \\
        34.50 & 9.07 & 6.57 & 2.50 \\
        35.50 & 9.08 & 6.65 & 2.43 \\
        36.50 & 9.10 & 6.74 & 2.36 \\
        37.50 & 9.11 & 6.85 & 2.26 \\
        38.50 & 9.14 & 6.95 & 2.19 \\
        39.50 & 9.19 & 7.08 & 2.11 \\
        
        \bottomrule
    \end{tabular}
\end{table}

\begin{figure}
    \centering
    \includegraphics[width=\textwidth]{build/bei_a.pdf}
    \caption{Ausgleichskurve und Messdaten des beidseitig aufliegenden Aluminiumstabes}
\end{figure}


Das Verfahren ist das Gleiche wie bei dem Kupferstab. Mit den Daten der Tabelle \ref{tab:4} ergibt sich aus der linken Hälfte ein Elastizitätsmodul von
\SI{69.9711(3238)}{\giga\pascal} und aus der rechten Hälfte ein Elastizitätsmodul von \SI{70.1704(1507)}{\giga\pascal}. Im Mittel lässt sich das 
Elastizitätsmodul des Aluminiumstabes zu \SI{70.0708(2373)}{\giga\pascal} bestimmen.





%Messwerte: Alle gemessenen physikalischen Größen sind übersichtlich darzustellen.
%
%Auswertung:
%Berechnung der geforderten Endergebnisse
%mit allen Zwischenrechnungen und Fehlerformeln &   sodass die Rechnung nachvollziehbar ist.
%Eine kurze Erläuterung der Rechnungen (z.B. verwendete Programme)
%Graphische Darstellung der Ergebnisse