\section{Auswertung}
\label{sec:Auswertung}




\begin{table}
    \centering 
    \caption{Durchbiegung des runden Kupferstabes bei einseitiger Einspannung}
    \sisetup{table-format=2.2}
    \begin{tabular}{S[table-format=2] S S S}
        \toprule
        {$x\:/\: \si{\centi\m}$} & {$D_0\:/\: \si{\milli\m}$} & {$D_M\:/\: \si{\milli\m}$} & {$D\:/\: \si{\milli\m}$ }\\
        \midrule
        3.00 & 10.16 & 10.10 & 0.06 \\
        6.00 & 10.07 & 9.87 & 0.20 \\
        9.00 & 9.92 & 9.53 & 0.39 \\
        12.00 & 9.70 & 9.05 & 0.65 \\
        15.00 & 9.47 & 8.49 & 0.98 \\
        18.00 & 9.23 & 7.86 & 1.37 \\
        21.00 & 9.06 & 7.26 & 1.80 \\
        24.00 & 8.95 & 6.67 & 2.28 \\
        27.00 & 8.84 & 6.06 & 2.78 \\
        30.00 & 8.76 & 5.42 & 3.34 \\
        33.00 & 8.65 & 4.72 & 3.93 \\
        36.00 & 8.54 & 4.00 & 4.54 \\
        39.00 & 8.41 & 3.25 & 5.16 \\
        42.00 & 8.23 & 2.46 & 5.77 \\
        45.00 & 8.05 & 1.64 & 6.41 \\
        
        \bottomrule
    \end{tabular}
\end{table}


\begin{table}
    \centering 
    \caption{Durchbiegung des eckigen Aluminiumstabes bei einseitiger Einspannung}
    \sisetup{table-format=2.2}
    \begin{tabular}{S[table-format=2] S S S}
        \toprule
        {$x\:/\: \si{\centi\m}$} & {$D_0\:/\: \si{\milli\m}$} & {$D_M\:/\: \si{\milli\m}$} & {$D\:/\: \si{\milli\m}$ }\\
        \midrule
        3.00 & 9.99 & 9.86 & 0.13 \\
        6.00 & 9.94 & 9.72 & 0.22 \\
        9.00 & 9.81 & 9.40 & 0.41 \\
        12.00 & 9.65 & 8.95 & 0.70 \\
        15.00 & 9.48 & 8.43 & 1.05 \\
        18.00 & 9.27 & 7.81 & 1.46 \\
        21.00 & 9.06 & 7.15 & 1.91 \\
        24.00 & 8.85 & 6.45 & 2.40 \\
        27.00 & 8.65 & 5.69 & 2.96 \\
        30.00 & 8.46 & 4.92 & 3.54 \\
        33.00 & 8.25 & 4.11 & 4.14 \\
        36.00 & 8.06 & 3.28 & 4.78 \\
        39.00 & 7.83 & 2.42 & 5.41 \\
        42.00 & 7.61 & 1.54 & 6.07 \\
        45.00 & 7.39 & 0.63 & 6.76 \\
        
        \bottomrule
    \end{tabular}
\end{table}


\begin{table}
    \centering 
    \caption{Durchbiegung des runden Kupferstabes bei beidseitiger Einspannung}
    \sisetup{table-format=2.2}
    \begin{tabular}{S[table-format=2.2] S S S}
        \toprule
        {$x\:/\: \si{\centi\m}$} & {$D_0\:/\: \si{\milli\m}$} & {$D_M\:/\: \si{\milli\m}$} & {$D\:/\: \si{\milli\m}$ }\\
        \midrule
        15.50 & 8.86 & 6.84 & 2.64 \\
        16.50 & 8.78 & 6.66 & 2.60 \\
        17.50 & 8.70 & 6.50 & 2.58 \\
        18.50 & 8.62 & 6.32 & 2.53 \\
        19.50 & 8.56 & 6.16 & 2.48 \\
        20.50 & 8.48 & 6.00 & 2.40 \\
        21.50 & 8.39 & 5.86 & 2.30 \\
        22.50 & 8.32 & 5.74 & 2.20 \\
        23.50 & 8.23 & 5.63 & 2.12 \\
        24.50 & 8.18 & 5.54 & 2.02 \\
        30.50 & 8.10 & 5.50 & 2.60 \\
        31.50 & 8.17 & 5.66 & 2.51 \\
        32.50 & 8.25 & 5.78 & 2.47 \\
        33.50 & 8.33 & 5.92 & 2.41 \\
        34.50 & 8.42 & 6.06 & 2.36 \\
        35.50 & 8.50 & 6.23 & 2.27 \\
        36.50 & 8.59 & 6.40 & 2.19 \\
        37.50 & 8.68 & 6.56 & 2.12 \\
        38.50 & 8.79 & 6.73 & 2.06 \\
        39.50 & 8.85 & 6.91 & 1.94 \\
        
        \bottomrule
    \end{tabular}
\end{table}


\begin{table}
    \centering 
    \caption{Durchbiegung des eckigen Aluminiumstabes bei beidseitiger Einspannung}
    \sisetup{table-format=2.2}
    \begin{tabular}{S[table-format=2.2] S S S}
        \toprule
        {$x\:/\: \si{\centi\m}$} & {$D_0\:/\: \si{\milli\m}$} & {$D_M\:/\: \si{\milli\m}$} & {$D\:/\: \si{\milli\m}$ }\\
        \midrule
        15.50 & 9.40 & 7.28 & 2.63 \\
        16.50 & 9.38 & 7.18 & 2.63 \\
        17.50 & 9.36 & 7.07 & 2.60 \\
        18.50 & 9.35 & 6.96 & 2.59 \\
        19.50 & 9.32 & 6.88 & 2.52 \\
        20.50 & 9.31 & 6.79 & 2.44 \\
        21.50 & 9.30 & 6.71 & 2.39 \\
        22.50 & 9.26 & 6.66 & 2.29 \\
        23.50 & 9.25 & 6.62 & 2.20 \\
        24.50 & 9.20 & 6.57 & 2.12 \\
        30.50 & 9.07 & 6.39 & 2.68 \\
        31.50 & 9.07 & 6.42 & 2.65 \\
        32.50 & 9.05 & 6.45 & 2.60 \\
        33.50 & 9.06 & 6.51 & 2.55 \\
        34.50 & 9.07 & 6.57 & 2.50 \\
        35.50 & 9.08 & 6.65 & 2.43 \\
        36.50 & 9.10 & 6.74 & 2.36 \\
        37.50 & 9.11 & 6.85 & 2.26 \\
        38.50 & 9.14 & 6.95 & 2.19 \\
        39.50 & 9.19 & 7.08 & 2.11 \\
        
        \bottomrule
    \end{tabular}
\end{table}
%Messwerte: Alle gemessenen physikalischen Größen sind übersichtlich darzustellen.
%
%Auswertung:
%Berechnung der geforderten Endergebnisse
%mit allen Zwischenrechnungen und Fehlerformeln,  sodass die Rechnung nachvollziehbar ist.
%Eine kurze Erläuterung der Rechnungen (z.B. verwendete Programme)
%Graphische Darstellung der Ergebnisse