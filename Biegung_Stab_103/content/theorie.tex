\section{Theorie}
\label{sec:Theorie}

Im Allgemeinen können Kräfte, die auf einen Körper einwirken, dessen Form und Volumen beeinflussen. 
Eine pro Fläche wirkende Kraft wird dabei als Spannung bezeichnet, wobei die Normalspannung $\sigma$, die auch als Druck bezeichnet wird,
diejenige Kraft pro Fläche ist, die senkrecht zur Oberfläche des Körpers auf diesen einwirkt. Für eine eindimensionale Änderung der
Körperform gilt dabei der Zusammenhang \begin{equation}
    \sigma = E \frac{\Delta L}L
\end{equation}

, der das Hookesche Gesetz beschreibt. Dabei ist $L$ die Dimension des Körpers ohne Krafteinwirkung und $\Delta L$ die eindimensionale
Verformung, die durch die wirkende Kraft zustande kommt. Der Faktor $E$ gibt den Elastizitätsmodul an. Ist diese bekannt, kann das zu 
untersuchende Material bestimmt werden. 

Die Bestimmung eines Elastizitätsmoduls erfolgt in dieser Versuchsreihe über eine Biegung, also eine spezifische Verformung des zu 
untersuchenden Materials. 

\subsection{Einseitige Einspannung}

abb[]

Die gestrichelte Linie zeigt die sogenannte neutrale Faser, die sich bei der Biegung nicht in der Länge verändert. 
Über dieser kommt es zur Streckung, und darunter zur Stauchung des Materials. Am Querschnitt $Q$ greifen damit Zug- und Druckspannungen
an, die bei gleichem Betrag entgegengesetzt wirken, wodurch ein Drehmoment $M_{\sigma}$ entsteht. Dieses lässt sich durch \begin{equation}
    M_{\sigma} = \int_Q y\sigma(y)\text{d}q
\end{equation}

darstellen, wobei $y$ der Abstand von $\text{d}q$ zur neutralen Faser ist.

Bei konstanter Kraft gleicht sich das durch sie wirkende Drehmoment $M_F$ mit dem eben beschrieben Drehmoment aus. Daraus folgt \begin{equation}
    \int_Q y\sigma(y)\text{d}q = F (L - x) 
\end{equation}

Mit \begin{equation}
    \sigma (y) = E y \frac{\text{d}^2 D}{\text{d} x^2}
\end{equation}

lässt sich dann durch Integration über $Q$ von gl[] \begin{equation}
    D(x) = \frac{F}{2EI} \Bigl( L x^2 - \frac{x^3}3 \Bigr)
\end{equation}

Dabei ist $I$ das Flächenträgheitsmoment \begin{equation}
    I := \int_Q y^2 \text{d}q
\end{equation}

Damit ist der Elastizitätsmodul nur durch $x$ und die davon abhängige Durchbiegung $D$ zu bestimmen.

\subsection{Beidseitige Auflage}
abb[]
Da hier der Stab beidseitig aufliegt und eine Auslenkung in der Mitte der der beiden Auflagepunkte gemessen wird, ist hierbei
vom voherigen Aufbau abweichend das Drehmoment \begin{align}
M_F &= - \frac{F}2 x       &   0 \leq x \leq L/2 \\
M_F &= - \frac{F}2 (L - x) & L/2 \leq x \leq L
\end{align}

Dadurch kommt die Funktion für die $x$- abhängige Auslenkung folgendermaßen zustande: \begin{equation}
   D(x) = \frac{F}{48EI} \bigl( 3L^2 x -4x^3 \bigr)
\end{equation} für $0 \leq x \leq L/2$
und für $L/2 \leq x \leq L$ gilt entsprechend \begin{equation}
    D(x) = \frac{F}{48EI} \bigl( 4x^3 - 12Lx^2 + 9L^2 x - L^3 \bigr)
\end{equation}

