\section{Durchführung}
\label{sec:Durchführung}

Die Klemme wird entlang des Lineals an verschiedenen Orten befestigt. Die Position auf dem Lineal und 
die vom Newtonmeter angezeigte Kraft wird zusammen abgetragen.
Die jeweiligen Tupel werden in der Tabelle \ref{tab:Messdaten} dargestellt.
  

    \subsection{Messdaten}
    \label{sec:Messdaten}

     \begin{table}
        \centering
        \caption{Messdaten des Versuchs}
        \label{tab:Messdaten} 
        \sisetup{table-format=1.2}
        \begin{tabular}{S S S}
            \toprule
            {$\Delta x \:/\: \si{\meter}$} & {$F \:/\: \si{\newton}$} & {$D \:/\: \frac{\si{\newton}}{\si{\meter}}$} \\
            \midrule
            0.05   & 0.15& 3.00  \\
            0.10  & 0.29 &  2.90\\
            0.15  & 0.44 &  2.93\\
            0.20  & 0.59 &  2.95\\
            0.25  & 0.74 &  2.96\\
            0.30  & 0.89 &  2.97\\
            0.35  & 1.04 &  2.97\\
            0.40  & 1.19 &  2.98\\
            0.45  & 1.34 &  2.98\\   
            0.50  & 1.49 &  2.98\\
            \bottomrule
        \end{tabular}
    \end{table}

