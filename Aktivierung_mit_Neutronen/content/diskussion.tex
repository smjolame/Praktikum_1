\section{Diskussion}
\label{sec:Diskussion}

Zusammengefasst ergeben sich die Halbwertszeiten zu $T_V=\SI{215(7)}{\s}$, $T_{R_l}=\SI{260(40)}{\s}$ und $T_{R_s}=\SI{43.0(11)}{\s}$. Laut Literatur \cite{JLab} betragen die Halbwertszeiten der untersuchten Elemente:
\begin{align*}
    T_{\ce{^{51}V}}&=224.58  \\
    T_{\ce{^{104}Rh}}&=42.3  \\
    T_{\ce{^{104i}Rh}}&=260.4
\end{align*}

Somit weicht die bestimmte Halbwertszeit von $\ce{^{51}V}$ um $\SI{4.3(32)}{\percent}$, von $\ce{^{104}Rh}$ um $\SI{1.8(25)}{\percent}$ und von $\ce{^{104i}Rh}$ um $\SI{1(14)}{\percent}$ von dem jeweiligen Literaturwert ab. Diese recht geringen Abweichungen kommen zustande, da in der Auswertung die jeweiligen Regressionsbereiche mit einiger Willkür ausgewählt werden. So wird zum einen keine erneute Regression zum Bestimmen der Halbwertszeit von Vanadium \ref{Vanadium} durchgeführt, bei der nur die Zeiten bis zur doppelten Halbwertszeit beachtet werden, da bei dieser Rechnung eine größere Abweichungen zum Literaturwert folgt. Ebenso wird, zum Bestimmen der Halbwertszeit von Rhodium \ref{Rhodium} der mittlere Bereich $t=\SI{210}{\s}-\SI{270}{\s}$ aus der Ausgleichsrechnung entfernt. Auch dieser Bereich wird mit einiger Willkür gewählt. Dadurch lässt sich ein literaturwertnahes Ergebnis erzielen. Ohne diese Möglichkeit, den Literaturwert im Vorhinein zu kennen, wären die Abweichungen deutlich höher, da es nur eingschränkt möglich ist, mit genügend Exaktheit die Ausgleichsgeraden zu platzieren.  



%Kurze Zusammenfassung der Ergebnisse
%-Vergleich mit Literaturwerten
%-Vergleich mit verschiedenen Messverfahren
%-bei Abweichungen mögliche Ursachen finden