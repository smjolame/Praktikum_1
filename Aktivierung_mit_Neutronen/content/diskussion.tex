\section{Diskussion}
\label{sec:Diskussion}

Zusammengefasst ergeben sich die Halbwertszeiten zu $T_V=\SI{215(7)}{\s}$, $T_{R_l}=\SI{260(40)}{\s}$ und $T_{R_s}=\SI{43.0(11)}{\s}$. Laut Literatur \cite{JLab} betragen die Halbwertszeiten der untersuchten Elemente:
\begin{align*}
    T_{\ce{^{51}V}}&=224.58  \\
    T_{\ce{^{104}Rh}}&=42.3  \\
    T_{\ce{^{104i}Rh}}&=260.4
\end{align*}

Somit weicht die bestimmte Halbwertszeit von $\ce{^{51}V}$ um $\SI{4.3(32)}{\percent}$, von $\ce{^{104}Rh}$ um $\SI{1.8(25)}{\percent}$ und von $\ce{^{104i}Rh}$ um $\SI{1(14)}{\percent}$ von dem jeweiligen Literaturwert ab.



%Kurze Zusammenfassung der Ergebnisse
%-Vergleich mit Literaturwerten
%-Vergleich mit verschiedenen Messverfahren
%-bei Abweichungen mögliche Ursachen finden