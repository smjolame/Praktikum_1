\section{Durchführung}
\label{sec:Durchführung}

Die Erzeugung von Neutronen erfolgt, indem $\ce{^{9}_{4}Be}$-Kerne mit $\alpha$-Teilchen beschossen werden, 
wobei die Reaktion 
\begin{equation}
    \ce{^{9}_{4}Be} + \ce{^{4}_{2}\alpha} \longrightarrow \ce{^{12}_{6}Be} + \ce{^{1}_{0}n}
\end{equation}
erfolgt. Die Geschwindigkeit dieser wird dann gesenkt, indem ein Paraffinmantel die Neutronenquelle umgibt. Beim durchdringen 
dieser Schicht geben die Neutronen durch Stoßprozesse einen Teil ihrer Energie ab, wodurch sie an Geschwindigkeit verlieren bis 
diese etwa $\SI{2.2}{\kilo\metre\per\s} $ bei $T=\SI{290}{\kelvin} $ beträgt. Ein solches Neutron wird dann als thermisches 
Neutron bezeichnet. 
\\
Der Zerfall der Atomkerne, der durch den Beschuss mit Neutronen zustande kommt, siehe Gleichung\ref{sec:Theorie}, wird durch ein 
Geiger-Müller-Zählrohr registriert. Dieser leitet das Signal an zwei alternierende Zähler weiter,die jeden Impuls abwechselnd messen.
So kann die Größe $N_{\Delta t}(t) $ bei festgelegtem Zeitintervall gemessen werden. Das Geiger-Müller-Zählrohr befindet sich dabei 
samt Probe in einem Bleimantel, um äußere Strahlungseinwirkung zu minimieren. Um diese Strahlungseinwirkung trotzdem mit 
einbeziehen zu können, wird der Nulleffekt $N_U$ gemessen, indem das Geiger-Müller-Zählrohr insgesamt sieben mal über ein 
Zeitintervall von $\Delta t =\SI{300}{\s} $ eine Messung ohne Preparat die auftreffenden Impulse misst. Dies kann in der Auswertung der Messergebnisse 
berücksichtigt werden.
\\
Um die Halbwertszeit von Vanadium zu untersuchen, wird die Vanadiumprobe in den Aufbau integriert. Dies erfolgt möglichst schnell,
nachdem die das Preparat durch den Neutronenbeschuss aktiviert wurde. Die Messzeit für das zählen der Zerfälle beträgt hierbei 
$\Delta t= \SI{30}{\s} $ und die Messung wird insgesamt 42 mal wiederholt. Für die Untersuchung von Rhodium ist die 
Vorgehensweise identisch, nur dass diesmal das Zeitintervall der Messung $\Delta t= \SI{30}{\s} $ beträgt. Diese Messung wird 
ebenfalls 42 mal wiederholt. 