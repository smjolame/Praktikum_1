\section{Auswertung}
\label{sec:Auswertung}
Alle Berechnungen werden mit dem Programm \glqq Numpy" \cite{numpy}, die Unsicherheiten mit dem Modul \glqq Uncertainties" \cite{uncertainties}, die Ausgleichsrechnungen mit dem Modul \glqq Scipy" \cite{scipy} durchgeführt und die grafischen Darstellungen über das Modul \glqq Matplotlib" \cite{matplotlib} erstellt.

\subsection{Bestimmung der Brennweite über Relation von Bildweite und Gegenstandsweite}

Die gemessenen Werte für die Bildweite $b$, die Gegenstandsweite $g$ und die über Gleichung \ref{eqn:linsengl} 
berechnete Brennweite $f$ sind für die beiden unterschiedlichen Linsen in den Tabellen \ref{tab:f150} und \ref{tab:f300} zu sehen.

\begin{table}
    \centering
    \caption{Werte von $g$, $b$ und $f$ für Linse 1 $(f_\text{Angabe}=\SI{150}{\mm})$}
    \begin{tabular}{c c c}
    \toprule
    $g\,/\,cm$ & $b\,/\,cm$ & $f\,/\,cm$ \\
    \midrule
    20 & 23.1 & 10.71 \\
    21 & 22.6 & 10.90 \\
    22 & 22.2 & 11.05 \\
    23 & 21.9 & 11.22 \\
    24 & 21.5 & 11.34 \\
    25 & 21.6 & 11.57 \\
    26 & 21.1 & 11.65 \\
    27 & 20.8 & 11.75 \\
    28 & 20.7 & 11.90 \\
    29 & 20.5 & 12.01 \\
    30 & 20.3 & 12.11 \\
    \bottomrule
    \end{tabular}
    \label{tab:f150}
\end{table}

\begin{table}
    \centering
    \caption{Werte von $g$, $b$ und $f$ für Linse 1 $(f_\text{Angabe}=\SI{300}{\mm})$}
    \begin{tabular}{c c c}
    \toprule
    $g\,/\,cm$ & $b\,/\,cm$ & $f\,/\,cm$ \\
    \midrule
    50 & 52.5 & 25.61 \\
    51 & 52.4 & 25.85 \\
    52 & 51.5 & 25.87 \\
    53 & 51.2 & 26.06 \\
    54 & 50.6 & 26.12 \\
    55 & 50.0 & 26.19 \\
    56 & 49.9 & 26.39 \\
    57 & 49.4 & 26.45 \\
    58 & 48.9 & 26.53 \\
    59 & 48.3 & 26.56 \\
    60 & 48.0 & 26.68 \\
    \bottomrule
    \end{tabular}
    \label{tab:f300}
\end{table}

Aus den errechneten Brennweiten wird nun für die jeweiligen Linsen der Mittelwert samt Fehler berechnet, wodurch sich die 
Brennweiten zu
\begin{align*}
    f_{150} &= \SI{11.47+-0.14}{\centi\metre} \\
    f_{300} &= \SI{26.21+-0.10}{\centi\metre} \, .
\end{align*}

Eine weitere Methode, um aus den gemessenen Daten die Brennweite zu bestimmen, ist es, Wertepaare der jeweils zusammengehörenden
Bildweiten und Gegenstandsweiten zu bilden. Zwischen diesen Wertepaaren werden jeweils Geraden gezogen und es werden Schnittpunkte
der Geraden untereinander betrachtet. Theoretisch gäbe es einen Einzigen Schnittpunkt, indem sich alle Geraden schneiden, 
allerdings ist dies nur bei unendlich genauer Messung der Fall, was in der Realität nicht zu erreichen ist. Daher wird auch ein 
Mittelwert der Schnittpunkte, bzw. jeweils ein Mittelwert für die $x-$ und $y-$Werte der Schnittpunkte bestimmt. 
Beide dieser Werte sollten die Brennweite der jeweiligen Linse angeben. In den Plots \ref{fig:150} und \ref{fig:300} sind für 
beide Linsen die Schnittpunkte der Geraden sichtbar. 

\begin{figure}
    \centering
    \includegraphics[width=\textwidth]{build/plot_brennweite150.pdf}
    \caption{Schnittpunkte der Geraden zwischen zusammengehörenden Brennweiten und Gegenstandsweiten für \SI{150}{\mm} Linse }
    \label{fig:150}
\end{figure}

\begin{figure}
    \centering
    \includegraphics[width=\textwidth]{build/plot_brennweite300.pdf}
    \caption{Schnittpunkte der Geraden zwischen zusammengehörenden Brennweiten und Gegenstandsweiten für \SI{300}{\mm} Linse }
    \label{fig:300}
\end{figure}

Daraus ergeben sich für die Brennweiten die Werte 
\begin{align*}
    f_{150,x} &= \SI{5.70+-0.18}{\centi\metre} \\
    f_{150,y} &= \SI{16.48+-0.15}{\centi\metre} \\
    f_{300,x} &= \SI{18.2+-0.5}{\centi\metre} \\
    f_{300,y} &= \SI{33.6+-0.5}{\centi\metre} \, .
\end{align*}

\subsection{Methode von Bessel}

In den Tabellen \ref{tab:besselrot}, \ref{tab:besselblau} und \ref{tab:bessel1} sind die jeweils aufgenommenen Messwerte 
zu sehen. Über den Abstand 
\begin{equation}
    d = g2-g1
\end{equation} kann mit der Gleichung \ref{eqn:besself} die Brennweite $f$ bestimmt werden. 

\begin{table}
    \centering
    \caption{Messwerte für die Messung mit Bessel-Methode mit weißem Licht}
    \begin{tabular}{c c c c c}
        \toprule
        $ e\,/\,cm$ & $g_1\,/\,cm$ & $g_2\,/\,cm$ & $d\,/\,cm$ & $f\,/\,cm$ \\
        \midrule
        102 & 37.0 & 47.2 & 10.2 & 25.24 \\
        104 & 33.8 & 52.7 & 18.9 & 25.14 \\
        106 & 31.6 & 56.5 & 24.9 & 25.03 \\
        108 & 30.8 & 59.9 & 29.1 & 25.03 \\
        110 & 29.1 & 63.1 & 34.0 & 24.86 \\
        112 & 27.9 & 65.7 & 37.8 & 24.82 \\
        114 & 27.2 & 68.9 & 41.7 & 24.70 \\
        116 & 26.6 & 71.7 & 45.1 & 24.62 \\
        118 & 25.7 & 74.0 & 48.3 & 24.56 \\
        120 & 25.4 & 76.1 & 50.7 & 24.64 \\
        \bottomrule
    \end{tabular}
    \label{tab:bessel1}
\end{table}

\begin{table}
    \centering
    \caption{Messwerte für die Messung mit Bessel-Methode mit blauem Licht}
    \begin{tabular}{c c c c c}
        \toprule
        $ e\,/\,cm$ & $g_1\,/\,cm$ & $g_2\,/\,cm$ & $d\,/\,cm$ & $f\,/\,cm$ \\
        \midrule
        104 & 33.1 & 53.7 & 20.6 & 24.98 \\
        106 & 31.7 & 57.7 & 26.0 & 24.91 \\
        108 & 29.6 & 60.9 & 31.3 & 24.73 \\
        110 & 28.9 & 64.0 & 35.1 & 24.70 \\
        112 & 27.9 & 65.9 & 38.0 & 24.78 \\
        \bottomrule
    \end{tabular}
    \label{tab:besselblau}
\end{table}

\begin{table}
    \centering
    \caption{Messwerte für die Messung mit Bessel-Methode mit rotem Licht}
    \begin{tabular}{c c c c c}
        \toprule
        $ e\,/\,cm$ & $g_1\,/\,cm$ & $g_2\,/\,cm$ & $d\,/\,cm$ & $f\,/\,cm$ \\
        \midrule
        104 & 34.7 & 52.3 & 17.6 & 25.26 \\
        106 & 32.4 & 55.8 & 23.4 & 25.21 \\
        108 & 30.6 & 59.4 & 28.8 & 25.08 \\
        110 & 29.6 & 62.9 & 33.3 & 24.98 \\
        112 & 28.9 & 66.0 & 37.1 & 24.93 \\
        \bottomrule
    \end{tabular}
    \label{tab:besselrot}
\end{table}

Auch hier werden wieder aus den einzelnen Tabellen für $f$ die Mittelwerte gebildet. Es ergeben sich für die Brennweiten
die Werte 
\begin{align*}
    f_{300,weiß} &= \SI{24.86+-0.08}{\centi\metre} \\
    f_{300,blau} &= \SI{24.82+-0.05}{\centi\metre} \\
    f_{300,rot} &= \SI{25.09+-0.06}{\centi\metre} \, .
\end{align*}

