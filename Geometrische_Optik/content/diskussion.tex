\section{Diskussion}
\label{sec:Diskussion}

Die erste Bestimmung der Brennweite erfolgte über den Vergleich von Bildweite und Gegenstandsweite. Es wurde zwei Linsen, eine mit 
$\SI{15}{\cm} $ und eine mit $\SI{30}{\cm} $ angegebener Brennweite. Die über die Gleichung \ref{eqn:linsengl} direkt errechneten 
Werte 
\begin{align}
    f_{150} &= \SI{11.47+-0.14}{\centi\metre} \\
    f_{300} &= \SI{26.21+-0.10}{\centi\metre} 
\end{align}
weichen um $\SI{23.5}{\percent} $ bei der $\SI{15}{\cm}$-Linse und um $\SI{12.6}{\percent} $ bei der $\SI{30}{\cm}$-Linse ab.
Die Werte liegen jeweils in der richtigen Größenordnung, aber können nicht als Bestätigung für den Angabewert der Linsenbrennweite 
angesehen werden. Für die gleiche Messreihe wurden die Werte noch mit einer grafischen Methode, wie sie in der Auswertung beschrieben
wird, berechnet. Diese Werte sind 
\begin{align*}
    f_{150,x} &= \SI{5.70+-0.18}{\centi\metre} \\
    f_{150,y} &= \SI{16.48+-0.15}{\centi\metre} \\
    f_{300,x} &= \SI{18.2+-0.5}{\centi\metre} \\
    f_{300,y} &= \SI{33.6+-0.5}{\centi\metre} \, .
\end{align*}
Nach der Theorie sollten jeweils die $x-Werte$ und $y-Werte$ gleich, bzw. sehr nahe beieinander sein, da sie beide die Brennweite
angeben sollten. Hier ist auffällig, dass der zwar die $y-Wert$ durchaus in der näheren Umgebung der angegebenen Brennweiten
liegen, die $x-Werte$ allerdings große Abweichungen zeigen. Ein Grund dafür könnte eine kontinuierlich fehlerhafte Aufnahme der 
Messwerte sein. Es ist auch zu beachten, dass der Fehler für die jeweiligen Größen verhältnismäßig klein ist, was eine ungenaue 
Messung eher ausschließt. Es scheint daher, als wären mit hoher Genauigkeit Werte aufgenommen worden, die nicht die gewollte 
Größe aussagen. In der Durchführung ist erwähnt, an welchen Stellen jeweils die Werte notiert wurden. Dies könnte aus genanntem 
Grund ein systematischer Fehler gewesen sein. 

Eine weitere Messreihe wurde die Bestimmung der Brennweite nach der Methode von Bessel aufgenommen.
Die dadurch erhaltenen Werte sind 
\begin{align*}
    f_{300,weiß} &= \SI{24.86+-0.08}{\centi\metre} \\
    f_{300,blau} &= \SI{24.82+-0.05}{\centi\metre} \\
    f_{300,rot} &= \SI{25.09+-0.06}{\centi\metre} \, .
\end{align*}
Die tatsächliche Brennweite sollte dabei bei der Messung mit weißem Licht bestimmt worden sein. Es ist eine Abweichung von 
$\SI{17.3}{\percent} $ zu sehen, was eine ähnlich Genauigkeit wie zuerst bestimmte Brennweite zeigt. Auch geht die Abweichung
in die gleiche Richtung, die beiden experimentell bestimmten Werte weichen also um nur $\SI{5.4}{\percent} $ voneinander ab. 
Es ist auch eine Möglichkeit, dass der angegebene Wert der Brennweite bei der $\SI{30}{\cm} $-Linse etwas ungenau ist. Diese 
Überlegung ist allerdings für eine Aussage über die Genauigkeit der experimentell bestimmten Werte problematisch, da der 
angegebene Wert der einzige Richtwert ist, an dem sich orientiert werden kann. \\

Weiterhin fällt auf, dass sich bei der Verwendung von blauem Licht die Brennweite verringert und bei rotem Licht vergrößert. 
Dies entspricht auch der Theorie. Da kein genauer Wert über die jeweiligen Wellenlängen des blauen und roten Lichts bekannt ist, 
kann an dieser Stelle keine Aussage über die Genauigkeit der Messung für die $\lambda$-abhängige Veränderung der Brennweite 
gemacht werden. 
