\section{Auswertung}
\label{sec:Auswertung}


\subsection{Bestimmung der Wärmekapazität des Dewargefäßes}

Damit im folgenden die einzelnen spezifischen Wärmekapazitäten der verschiedenen Stoffe bestimmt werden können, muss zuerst die Wärmekapazität des Kalorimeters ermittelt werden. Dazu wird über die Gleichung \eqref{eqn:c_g} und den Werten
\begin{align*}
    c_w= & \SI{4.18}{\joule\per\g\per\kelvin} \\
    m_1= & \SI{276.60}{\g}         \\
    m_2= & \SI{276.05}{\g}         \\
    T_1= & \SI{20.7}{\celsius}    \\
    T_2= & \SI{73.3}{\celsius}    \\
    T_m= & \SI{44.9}{\celsius}  
\end{align*}
die Wärmekapazität zu $c_gm_g=\SI{334.23}{\joule\per\g\per\kelvin}$ bestimmt.

\subsection{Bestimmung der spezifische Wärmekapazität verschiedener Stoffe}
 
Zum Bestimmen der Molwärme $C_V$ wird vorerst die spezifische Wärmekapazität $c_k$ berechnet. Mit Kenntnis von $c_k$ kann dann über Gleichung \eqref{eqn:C_V} die Molwärme $C_V$ bestimmt werden. Die Materialeigenschaften der jeweiligen Stoffe können in der Versuchsanleitung \cite{V201} eingesehen werden.


Die spezifische Wärmekapazität der Stoffe wird im folgenden bestimmt.

\subsubsection{Kupfer}

\begin{table}
    \centering
    \caption{Messdaten Kupfer.}
    \label{tab:k}
    \begin{tabular}{c| S S S S }
        \toprule
        & $m_\text{D} \:/\: \si{\g}$ & $T_\text{Cu} \:/\: \si{\celsius}$ & $T_\text{D} \:/\: \si{\celsius}$ &  $T_\text{M}\:/\: \si{\celsius}$  \\
        \midrule
        Messung 1 & 592.95 & 87.2 & 21.7 & 23.8 \\
        Messung 2 & 576.26 & 81.6 & 21.6 & 24.0 \\
        Messung 3 & 568.06 & 84.3 & 21.3 & 23.3 \\
        \bottomrule 
    \end{tabular}
\end{table}

Über die Gleichung \eqref{eqn:c_k} kann mit Hilfe der Werte der Tabelle \ref{tab:k} die spezifische Wärmekapazität von Kupfer berechnet werden. Diese ergeben sich in den drei Messungen zu:
\begin{align*}
\text{Messung 1:} \: c_k = & \SI{0.391}{\joule\per\g\per\kelvin}  \\  
\text{Messung 2:} \: c_k = & \SI{0.480}{\joule\per\g\per\kelvin}  \\
\text{Messung 3:} \: c_k = & \SI{0.373}{\joule\per\g\per\kelvin}
\end{align*}
Über die Gleichungen [REF] und [REF] kann der Mittelwert samt Fehler berechnet werden.
Daraus ergibt sich eine spezifische Wärmekapazität für Kupfer von $c_k=\SI{0.415(033)
}{\joule\per\g\per\kelvin}$.

\subsubsection{Graphit}

Da für Graphit nur die Daten einer Messung vorliegen, kann die spezifische Wärmekapazität direkt über Gleichung \eqref{eqn:c_k} bestimmt werden. Die Daten der Messung sind in Tabelle \ref{tab:g} aufgeführt. Damit ergibt sich ein Wert von $c_k=\SI{0.509}{\joule\per\g\per\kelvin}$.

\begin{table}
    \centering
    \caption{Messdaten Graphit.}
    \label{tab:g}
    \begin{tabular}{S S S S }
        \toprule
         $m_\text{D} \:/\: \si{\g}$ & $T_\text{Gr} \:/\: \si{\celsius}$ & $T_\text{D} \:/\: \si{\celsius}$ &  $T_\text{M}\:/\: \si{\celsius}$  \\
        \midrule
         531.39 & 82.2 & 21.7 & 22.2 \\

        \bottomrule 
    \end{tabular}
\end{table}

\subsubsection{Zinn}

Das Verfahren zur Bestimmung der spezifischen Wärmekapazität läuft analog zu den anderen Stoffen ab. Die benötigten Messdaten sind in Tabelle \ref{tab:z} dargestellt.
\begin{table}
    \centering
    \caption{Messdaten Zinn.}
    \label{tab:z}
    \begin{tabular}{c| S S S S }
        \toprule
        & $m_\text{D} \:/\: \si{\g}$ & $T_\text{Sn} \:/\: \si{\celsius}$ & $T_\text{D} \:/\: \si{\celsius}$ &  $T_\text{M}\:/\: \si{\celsius}$  \\
        \midrule
        Messung 1 & 545.59 & 81.2 & 21.7 & 22.2 \\
        Messung 2 & 575.85 & 82.0 & 21.3 & 21.8 \\
        Messung 3 & 578.78 & 80.8 & 21.2 & 21.6 \\
        \bottomrule 
    \end{tabular}
\end{table}

Es ergeben sich durch die drei Messdaten die Wärmekapazitäten:
\begin{align*}
\text{Messung 1:} \: c_k = & \SI{0.094}{\joule\per\g\per\kelvin}   \\ 
\text{Messung 2:} \: c_k = & \SI{0.098}{\joule\per\g\per\kelvin}   \\
\text{Messung 3:} \: c_k = & \SI{0.080}{\joule\per\g\per\kelvin}
\end{align*}

Im Mittel ergibt sich eine spezifische Wärmekapazität für Zinn von $c_k=\SI{0.091(005)}{\joule\per\g\per\kelvin}$

\subsection{Bestimmung der Molwärme}
Nach Berechnung aller ,für die Bestimmen der Molwärme, notwendiger Werte kann nun über die Gleichung \eqref{eqn:C_V} die Molwärme der einzelnen Stoffe betimmt werden. 




\begin{table}
    \centering
    \caption{Molwärme verschiedener Stoffe.}
    \label{tab:m}
    \begin{tabular}{c| S S S }
        \toprule
        &\multicolumn{3}{S}{$C_V \:/\: \si{\joule\per\kelvin\per\mol}$} \\
        & {Kupfer} & {Zinn} & {Graphit} \\
        \midrule
        Messung 1 & 24.8 & 11.1 & 6.1 \\
        Messung 2 & 30.5 & 11.5  &  \\
        Messung 3 & 23.7 & 9.4 & \\
        \bottomrule 
    \end{tabular}
\end{table}

Mit den bekannten Gleichungen zum Bestimmen des Mittelwertes und des Fehlers ergeben sich die Molwärmen der Stoffe zu:
\begin{align*}
\text{Kupfer: }\: C_V=&\SI{26.3(21)}{\joule\per\kelvin\per\mol}  \\
\text{Graphit:}\: C_V=&\SI{6.1}{\joule\per\kelvin\per\mol}  \\
\text{Zinn: }  \: C_V=&\SI{10.6(6)}{\joule\per\kelvin\per\mol}
\end{align*}


%Messwerte: Alle gemessenen physikalischen Größen sind übersichtlich darzustellen.
%
%Auswertung:
%Berechnung der geforderten Endergebnisse
%mit allen Zwischenrechnungen und Fehlerformeln& sodass die Rechnung nachvollziehbar ist.
%Eine kurze Erläuterung der Rechnungen (z.B. verwendete Programme)
%Graphische Darstellung der Ergebnisse