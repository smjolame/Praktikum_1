\section{Theorie}
\label{sec:Theorie}


Die Molwärme $C$ beschreibt diejenige Wärmemenge $\symup dQ$, die benötigt wird, um ein Mol eines bestimmten Elements um 
$\symup dT$ zu erwärmen. Dabei ist zu unterscheiden, ob der Wärmeaustausch bei konstantem Volumen $C_V$ oder konstantem Druck $C_p$
passiert. Für ein konstantes Volumen gilt:
\begin{equation}
    C_V = \frac{\symup dQ}{\symup dT} 
\end{equation}

Im durchzuführenden Versuch ist es nicht möglich, ein konstantes Volumen beizubehalten. Die Beziehung zur Molwärme $C_p$ besteht wie 
folgt:
\begin{equation}
    C_V = C_p - 9 \alpha^2 \kappa V_0 T_m 
\end{equation}

Dabei ist $\alpha$ der lineare Ausdehnungskoeffizient, $\kappa$ der Kompressionsmodul und $V_0$ das Molvolumen.

Die Molwärme kann auch über die spezifische Wärmekapazität $c_k$ und die molare Masse $M$ beschrieben werden. Es gilt 
\begin{equation}
    C_p = c_k M
\end{equation},
wobei $c_k$ der Proportionalitätsfaktor für die Temperaturänderung $\Delta T$ bei einer zugeführten Wärmemenge $\Delta Q$ ist. 
Dieser Zusammenhang wird durch 
\begin{equation}
    \Delta Q = m c_k \Delta T
\end{equation}
beschrieben.
Alle Werte eingesetz ergeben die Gleichung:
\begin{equation}
   C_V = c_k M - 9 \alpha^2 \kappa \frac{M}{\rho} T_m 
   \label{eqn:C_V}
\end{equation}


Das Dulong-Petit'sche Gesetz sagt aus, dass die Atomwärme, unabhängig von den Eigenschaften des Elements, immer den
konsanten Wert $3\cdot R$ hat, wobei $R$ die allgemeine Gaskonsante ist. 

Der gemittelte Wert für die Gesamtenergie eines Atoms beträgt 
\begin{equation}
    \langle E \rangle = \langle E_\text{kin} \rangle + \langle E_\text{pot} \rangle = 2\cdot \langle E_\text{kin} \rangle
\end{equation}

Die kinetische Energie eines Atoms kann dabei durch 
\begin{equation}
    \langle E_\text{kin} \rangle = \frac{1}2 k T
\end{equation}
beschrieben werden, wobei $k$ die Boltzmannkonstante ist.

Damit gilt für ein Mol mit $N_L$ Atomen 
\begin{equation}
    \langle E \rangle = N_L k T = R T 
\end{equation}

Ein Atom eines Festkörpers hat drei Freiheitsgrade, wodurch sich für dessen mittlere Gesamtenergie 
\begin{equation}
    \langle E \rangle = 3 R T 
\end{equation},
und analog dazu 
\begin{equation}
    C_V = 3 R
\end{equation}
ergibt.
Wird ein Probekörper der Tempertatur $T_k$ in ein Dewargefäß mit Wasser niedrigerer Tempertatur $T_w$ gelegt, gibt er 
dabei die Wärmemenge 
\begin{equation}
    Q_1 = c_k m_k (T_k - T_m)
\end{equation}
ab, während das Wasser die Wärmemenge 
\begin{equation}
    Q_2 = c_w m_w + c_g m_g (T_m - T_w)
\end{equation}
aufnimmt.
Dabei ist $c_g m_g$ die Wärmekapazität des Dewargefäßes.

Für die spezifische Wärmekapazität des Probekörpers $c_k$ gilt damit 
\begin{equation}
    c_k = \frac{c_w m_w + c_g m_g (T_m - T_w)}{m_k (T_k - T_m)}
    \label{eqn:c_k}
\end{equation}

Bei unbekannter Wärmekapazität des Dewargefäßes lässt sich diese durch die Mischung zweier Wassermengen der Masse $m_1$ und $m_2$
der Tempertaturen $T_1$ und $T_2$ bestimmen. Die Mischtemperatur dieser beiden Tempertaturen ist als $T_m'$ angegeben.
Analog zur vorherigen Berechnung gilt dann
\begin{equation}
    c_g m_g = \frac{c_w m_2 (T_2 - T_m') - c_w m_1 (T_m' - T_1)}{T_m' - T_1} 
    \label{eqn:c_g}
\end{equation}


