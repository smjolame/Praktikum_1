\section{Diskussion}
\label{sec:Diskussion}

Im Folgenden sind experimentell bestimmte Werte und Literaturwerte[] der spezifischen Wärmekapazitäten zusammengefasst.

\begin{align*}
    c_{k_{ex},Cu} =& \:\SI{0.415(033)}{\joule\per\g\per\kelvin}&   c_{k_{th},Cu} =& \:\SI{0.385}{\joule\per\g\per\kelvin} \\
    c_{k_{ex},C} =& \:\SI{0.509}{\joule\per\g\per\kelvin}&   c_{k_{th},C} =& \:\SI{0,720}{\joule\per\g\per\kelvin} \\
    c_{k_{ex},Sn} =& \:\SI{0.091(005)}{\joule\per\g\per\kelvin}&   c_{k_{th},Sn} =& \:\SI{0.221}{\joule\per\g\per\kelvin} 
\end{align*}

Der experimentell bestimmte Wert für Kupfer ist sehr nah an der Literaturangabe, mit einer Abweichung von nur $7,79\si{\percent}$.
Bei Graphit beläuft sich die Abweichung auf $29,31\si{\percent}$, und bei Zinn $58,82\si{\percent}$. 

Die aus den experimentell bestimmten $c_k$ berechneten Werte für die Molwärme $C_V$ sind wie folgt aufgelistet:

\begin{align*}
    C_{V,Cu}=&\:\SI{26.3(21)}{\joule\per\kelvin\per\mol} \\
    C_{V,C}=&\:\SI{6.1}{\joule\per\kelvin\per\mol} \\
    C_{V,Sn}=&\:\SI{10.6(6)}{\joule\per\kelvin\per\mol}
\end{align*}

Der Literaturwert für die allgemeine Gaskonstante beträgt 
\begin{equation}
    R = 8,314\:\si{\joule\per\mol\kelvin}.
\end{equation}

In der Theorie müsste daher die Molwärme 
\begin{equation}
    C_V \approx 24,942\:\si{\joule\per\mol\kelvin}
\end{equation}
sein. 

Die Molwärme, die über das $c_{k_{ex},Cu}$ berechnet wurde, stimmt verhältnismäßig gut mit dem zu erwartenden Wert überein und weicht
nur um $5,44\si{\percent}$ von $3R$ ab. Durch ein sehr stark vom Theoriewert abweichendes $c_{k_{ex}}$ von Zinn und Graphit kommt auch
für deren Molwärme eine starke Abweichung von $3R$ zustande. Für $C_{V,C}$ ist die Abweichung $75,54\si{\percent}$ und für $C_{V,Sn}$
$57,47\si{\percent}$. 

Urpsrunglich zu erwarten gewesen wäre, dass die durch $C_{V,Sn}$ ähnlich nah wie $C_{V,Cu}$ an dem Wert $3R$ ist, da deren Atomgewicht
relativ ähnlich ist. Graphit hat ein sehr geringes Atomgewicht, wodurch das $\omega$ aus der Gleichung \ref{eqn:e_qm} groß ist.
Damit diese Gleichung als $3R$ angenähert werden kann, muss $k T \gg \hbar \omega$ sein, was höhere Temperaturen erfordert, je höher
$\omega$ ist. Die große Abweichung von $C_{V,C}$ wäre also dadurch zu erklären. Warum $C_{V,Sn}$ so weit vom erwarteten Wert abweicht,
ist nicht klar, es könnte aber auf systematische Fehler bei der Versuchsdurchführung zurückgeführt werden. Beispielsweise funktionierte
der Rührfisch nicht, weshalb ein manuelle umrühren des Wassers im Dewargefäß notwendig war. Dies wurde eventuell nicht stark genug 
gemacht. Wie eben angedeutet, ist allerdings eine eindeutige Begründung für die hohe Abweichung von $C_{V,Sn}$ nicht möglich.
