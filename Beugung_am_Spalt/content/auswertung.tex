\section{Auswertung}
\label{sec:Auswertung}
Alle Berechnungen werden mit dem Programm \glqq Numpy" \cite{numpy}, die Unsicherheiten mit dem Modul \glqq Uncertainties" \cite{uncertainties}, die Ausgleichsrechnungen mit dem Modul \glqq Scipy" \cite{scipy} durchgeführt und die grafischen Darstellungen über das Modul \glqq Matplotlib" \cite{matplotlib} erstellt.


Die gemessenen Ströme sind in Abhängigkeit des Abstandes $x$ in der Tabelle \ref{tab:1} aufgeführt. 
Da die Photokathode auch bei Dunkelheit einen Strom misst, welcher in diesem Fall einen Wert von $I_{dunkel} = \SI{64}{\nano\ampere}$ aufweist, müssen die Messwerte über das Abziehen des Dunkelstroms bereinigt werden. Also werden über
\begin{equation}
    I_{bereinigt}=I-I_{dunkel}
\end{equation}
alle gemessenen Ströme korregiert. 
Um über die Messwerte die Spaltbreite bestimmen zu können, wird anhand der Messwerte eine Ausgleichsrechnung durchgeführt. Dazu wird als Ausgleichskurve die Funktion aus Gleichung \eqref{eqn:aus} verwendet. Um diese durchführen zu können, muss der Messwert bei $x=0$ bei der Rechnung missachtet werden, da sonst im Nenner eine $0$ auftauchen würde. Außerdem wird der Abweichwinkel über 
\begin{equation}
    \phi=\arctan \left(\frac{x-x_0}{L}\right )
\end{equation}
berechnet. Dabei ist $x_0$ der Ort, bei dem das Licht keine Beugung erfahren hat und $L$ der Abstand zwischen Spalt und Photoelement. Für die auf die vorliegende Weise aufgenommenen Messwerte lässt sich $x_0 = \SI{27.5}{\milli\meter}$ festlegen. 

So ergibt sich nach Durführung der Ausgleichsrechnung eine Spaltbreite von 
\begin{equation*}
    b = \SI{0.1090(0014)}{\milli\meter} \..
\end{equation*}

Die entsprechende Ausgleichskurve und die Messwerte sind in Abbildung \ref{fig:graph} dargestellt.




\begin{table}
    \centering 
    \caption{Messwerte der Lichtintensitäsverteilung am Einzelspalt.}
    \sisetup{table-format=1.3}
    \label{tab:1}
    \begin{tabular}[t]{S[table-format=1.1] S}
        \toprule
        {$x\:/\: \si{\milli\m}$} & {$I \:/\: \si{\micro\ampere}$} \\
        \midrule
0.5    &   1.44     \\
0      & 1.5      \\
-0.5   &    1.44     \\
-1     &  1.26     \\
-1.5   &    1.08     \\
-2     &  0.78     \\
-2.5   &    0.58     \\
-3     &  0.4      \\
-3.5   &    0.264    \\
-4     &  0.183    \\
-4.5   &    0.141    \\
-5     &  0.126   \\
-5.5   &    0.12     \\
-6     &  0.114    \\
-6.5   &    0.105    \\
-7     &  0.096    \\
-7.5   &    0.088    \\
-8     &  0.08     \\
-8.5   &    0.078    \\
-9     &  0.08     \\
-9.5   &    0.082    \\
-10    &   0.086    \\
-10.5  &      0.088    \\        
-11    &    0.088   \\
-11.5  &      0.086    \\
-12    &    0.082    \\
-12.5  &      0.078    \\
-13    &    0.075    \\
        \bottomrule
    \end{tabular}
    \begin{tabular}[t]{S[table-format=1.1] S}
        \toprule
        {$x\:/\: \si{\milli\m}$} & {$I \:/\: \si{\micro\ampere}$} \\
        \midrule
-13.5  &      0.073    \\
-14    &    0.072    \\
-14.5  &      0.071    \\
-15    &    0.07     \\
-15.5  &      0.07    \\
-16    &    0.07     \\
-16.5  &      0.071    \\
1      &  1.38   \\
1.5    &    1.14     \\
2      &  0.84     \\
2.5    &    0.59     \\
3      &  0.37     \\
3.5    &    0.222    \\
4      &  0.15     \\
4.5    &    0.129    \\
5      &  0.144    \\
5.5    &    0.168    \\
6      &  0.186    \\
6.5    &    0.186    \\
7      &  0.168   \\
7.5    &    0.147    \\
8      &  0.123    \\
8.5    &    0.108    \\
9      &  0.096    \\
9.5    &    0.096    \\
10     &   0.099    \\
10.5   &     0.102    \\
11     &   0.102   \\

        
        \bottomrule
    \end{tabular}
\end{table}

\begin{figure}
\centering
\caption{Ausgleichskurve durch die Messwerte der Licht-Intensitätsverteilung am Einzelspalt.}
\includegraphics{build/graph.pdf}
\label{fig:graph}
\end{figure}
%Messwerte: Alle gemessenen physikalischen Größen sind übersichtlich darzustellen.
%
%Auswertung:
%Berechnung der geforderten Endergebnisse
%mit allen Zwischenrechnungen und Fehlerformeln, sodass die Rechnung nachvollziehbar ist.
%Eine kurze Erläuterung der Rechnungen (z.B. verwendete Programme)
%Graphische Darstellung der Ergebnisse