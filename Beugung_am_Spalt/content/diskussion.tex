\section{Diskussion}
\label{sec:Diskussion}
Der aus der Ausgleichsrechnung erhaltene Wert für die Spaltbreite $b = \SI{0.1090(0014)}{\milli\meter}$ weicht um \SI{27.3(9)}{\percent} von der vom Hersteller angegebenen Spaltbreite von $b = \SI{0.15}{\milli\meter}$ ab. Diese Abweichungen lässt sich als Produkt einiger Ungenauigkeiten des Versuches erklären. 

Zum Einen entspricht der Verlauf der Messwerte nicht dem zu erwartenen Ergebnis einer symmetrischen Verteilung. Diese Asymmetrie ist in der grafischen Darstellung \ref{fig:graph} besonders gut zu erkennen. Die Position und Höhe der beiden ersten Nebenmaxima sollten im Idealfall identisch sein. Außerdem kann bei der Ausgleichsrechnung die Position des Hauptmaximums nur geschätz werden, indem der Ort der höchsten Intensität als Position des Hauptmaximums festgelegt wird. Jedoch handelt es sich um ein diskretes Messverfahren, wodurch die bestimmte Position des Hauptmaximums von der tatsächlichen abweichen kann. Diese Abweichungen ist zwar gering, da die Messintervalle schon so gering wie möglich gewählt sind, jedoch hätten feinere Geräte eine höhere Genauigkeit erzeugt. 

Bei der Ausgleichsrechnung handelt es sich um eine recht komplizierte Ausgleichsfunktion, wodurch sich die nummerische Annäherung nur erschwert umsetzten lässt. 

Darüberhinaus findet der Versuch in einer gestörten Umgebung statt. Ein in kompletter Dunkelheit und isolierter Umgebung durchgeführter Versuch hätte exaktere Messwerte ergeben können.


%Kurze Zusammenfassung der Ergebnisse
%-Vergleich mit Literaturwerten
%-Vergleich mit verschiedenen Messverfahren
%-bei Abweichungen mögliche Ursachen finden