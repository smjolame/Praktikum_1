\section{Durchführung}
\label{sec:Durchführung}

Für die Durchführung des Versuches wird in die beiden Wasserbehälter jeweils genau drei Liter Wasser gefüllt. Zu Anfang 
sind dabei deren Temperaturen gleich, bei etwa 21$\si{\celsius}$.
Diese dienen als Reservoir 1 und 2 und werden wie in der Grafik[] platziert. In beiden Reservoiren ist ein Thermometer
angebracht, welches die Wassertemperatur misst. Der Druck in beiden Reservoiren kann außerdem jeweils über ein Manometer
abgelesen werden. Nach anschalten des Kompressors wird im Minutentakt Temperatur und Druck beider Reservoire, sowie 
die Kompressorleistung abgelesen und notiert, bis $T_1$ $50\si{\celsius}$ erreicht. 