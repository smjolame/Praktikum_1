\section{Theorie}
\label{sec:Theorie}

Beim Kontakt zwischen zwei Wärmereservoiren unterschiedlicher Temperatur wird nach dem zweiten Hauptsatz der Thermodynamik
im Normalfall immer Wärme vom wärmeren zum kälteren Reservoir übertragen. Ziel der Wärmepumpe ist es, den Wärmeaustausch umzukehren,
sodass Wärme vom kälteren Reservoir in das Wärmere übertragen wird. Dazu muss allerding zusätzliche Energie in Form von mechanischer
Arbeit aufgewendet werden.

Dabei gibt die Güteziffer $\nu$ Aufschluss über die Effizienz des Wärmetransports, indem die pro Arbeit $A$ abgegebene 
Wärmemenge $Q_1$ beschrieben wird:
\begin{equation}
    \nu = \frac{Q_1}A 
\end{equation}
mit
\begin{equation}
    Q_1 = Q_2 + A 
\end{equation}

Dabei ist $Q_2$ die vom kälteren Reservoir entnommene Wärmemenge.

Für den Idealfall, dass die Wärmeübertragung komplett reversibel ist, gilt außerdem 
\begin{equation}
    \frac{Q_1}{T_1} - \frac{Q_2}{T_2} = 0,
\end{equation}
was in der Realität nicht der Fall ist. Dieser Wert ist damit immer größer als null, wodurch auch die Beziehung
\begin{equation}
    \nu_{\text{id}} = \frac{T_1}{T_1 - T_2}
\end{equation}
zu 
\begin{equation}
    \nu_{\text{real}} < \frac{T_1}{T_1 - T_2}
\end{equation}
wird.

Die genaue Berechnung der realen Güteziffer berechnet sich über die Gleichung 
\begin{equation}
    \nu_\text{real} = \frac{\text{d}Q_1}{\text{d}tN} = (m_1c_w + m_kc_k)\frac{\text{d}T_1}{\text{d}t}\cdot\frac{1}N,
    \label{eqn:nureal}
\end{equation}
wobei $N$ die gemittelte Leistungsaufnahme des Kompressors ist. $m_1c_w$ ist die Wärmekapazität des Wassers in Reservoir 1 und 
$m_kc_k$ ist die Wärmekapazität der Kupferschlange und des Wasserbehälters. Die genannten Komponenten sind in der folgenden Grafik
zu sehen. []

Analog zu Gleichung\ref{eqn:nureal} lässt sich auch die pro Zeit abgegebene Wärmemenge durch 
\begin{equation}
    \frac{\text{d}Q_2}{\text{d}t} = (m_2c_w + m_kc_k)\frac{\text{d}T_2}{\text{d}t} 
\end{equation}
beschreiben. Umgeformt wird dies zu 
\begin{equation}
    \frac{\text{d}Q_2}{\text{d}t} = L\frac{\text{d}m}{\text{d}t},
    \label{eqn:mass}
\end{equation}
wodurch sich mit bekannter Verdampfungswärme $L$ der Massendurchsatz $\frac{\text{d}m}{\text{d}t}$ berechnen lässt.

Verringert ein Kompressor das Gasvolumenn $V_a$ auf $V_b$, so gilt für die dafür verrichtete Arbeit
\begin{equation}
    A_m = - \int_{V_a}^{V_b} p\text{d}V
\end{equation}

Der Zusammenhang zwischen Druck und Volumen lässt sich dabei durch die Poisson-Gleichung 
\begin{equation}
    p_aV_a^\kappa = p_bV_b^\kappa = pV^\kappa
\end{equation}
beschreiben. Damit gilt für die Kompressorleistung
\begin{equation}
    N_{\text{mech}} = \frac{\text{d}A_m}{\text{d}t} = \frac{1}{\kappa -1} \biggl(p_b \sqrt[\kappa]{\frac{p_a}{p_b}}-p_a \biggr)\frac{\text{d}V_a}{\text{d}t}
    = \frac{1}{\kappa -1} \biggl(p_b \sqrt[\kappa]{\frac{p_a}{p_b}}-p_a \biggr)\frac{1}{\rho}\frac{\text{d}m}{\text{d}t},
    \label{eqn:N_mech}
\end{equation}
mit dem Druck $p_a$ und $p_b$ und der Gasdichte $\rho$.

Der Aufbau und die Funktionsweise der Komponenten der Wärmepumpe werden im Folgenden erläutert.

Das Transportgas wird, nachdem es durch $D$ in das Reservoir 2 gelangt, gasförmig, was über den Druck $p_a$ realisiert wird.
Dabei nimmt das Gas Wärme auf, die dem Wasser aus Reservoir 2 entzogen wird. Das Gas gelangt dann über den Kompressor $K$ in das 
Reservoir 1, in dem es sich durch den höheren Druck wieder verflüssigt und damit Wärme an das Wasser abgibt. Das Gas verdampft 
wieder, sobald es durch $D$ in Reservoir 2 gelangt und der Vorang wiederholt sich. So wird kontinuierlich Wärme von Reservoir 2
and Reservoir 1 übertragen.



