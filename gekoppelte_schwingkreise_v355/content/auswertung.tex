\section{Auswertung}
\label{sec:Auswertung}

Es wurde für die Versuchsreihe die Schaltung 2 verwendet. Deren Komponenten haben folgende Werte:
\begin{align*}
& L & = 23,954 \si{\milli\henry} \\
& C_\text{kon} & = 0,793 \si{\nano\farad} \\
& C_\text{sp} & = 0,028 \si{\nano\farad} \\
& R & = 48 \si{\ohm} \\
\end{align*}
Die Kapazitäten $C_k$ der Koppelkondensatoren sind hierbei $2,19 \si{\nano\farad}$; $2,86 \si{\nano\farad}$;
$4,74 \si{\nano\farad}$; $6,86 \si{\nano\farad}$; $8,18 \si{\nano\farad}$; $9,99 \si{\nano\farad}$ und
$12 \si{\nano\farad}$.

\subsection{Frequenzverhältnis}
Das Frequenzverhältnis zwischen Schwingung und Schwebung lässt sich durch die Anzahl der Maxima der Schwingung innerhalb
einer Schwebung ausdrücken. Es wurde im Experiment jedoch die Anzahl der Amplituden gemessen, wodurch die Anzahl der 
Maxima nur näherungsweise als $N/2$ bestimmt werden kann.  

\begin{table}
    \centering
    \caption{Frequenzverhältnis von Schwingung und Schwebung}
    \begin{tabular}{c c c c}
        \toprule
        {$\nu / \si{\hertz}$} & {$C_k / \si{\nano\farad}$} & {$N_\text{amplituden}$} & $n_\text{max}$ \\
        \midrule
        710 &  2.19 &  7 &  3.5 \\  
        710 &  2.86 & 10 &  5   \\
        710 &  4.74 & 13 &  6.5 \\
        710 &  6.86 & 19 &  9.5 \\
        710 &  8.18 & 24 & 12   \\
        710 &  9.99 & 26 & 13   \\
        317 & 12.00 & 33 & 16.5 \\
        \bottomrule
        \label{tab:a}
    \end{tabular}
\end{table}

\FloatBarrier
\subsection{Fundamentalschwingungen}
Für die unterschiedlichen Kopplungskapazitäten sind im Folgenden die Frequenzen der jeweiligen
Fundamentalschwingungen $\nu^+$ und $\nu^-$, wie auch die dabei jeweils auftretenden maximalen
Spannungsamplituden $U^+$ und $U^-$.
\begin{table}
    \centering
    \caption{Frequenzen der Fundamentalschwingungen}
    \begin{tabular}{c c c c c}
    \toprule
    {$C_k / \si{\nano\farad}$} & {$\nu^- / \si{\kilo\hertz}$} & {$U^-$} & {$\nu^+ / \si{\kilo\hertz}$} & {$U^+$} \\
    \midrule
    12.00 & 37.9 & 4.3 & 35.6 & 4.7 \\
     9.99 & 38.4 & 4.4 & 35.6 & 4.7 \\
     8.18 & 38.9 & 4.4 & 35.6 & 4.7 \\
     6.86 & 39.5 & 4.4 & 35.6 & 4.7 \\
     4.74 & 41.1 & 4.3 & 35.6 & 4.7 \\
     2.86 & 44.2 & 4.1 & 35.6 & 4.7 \\
     2.19 & 46.5 & 4.0 & 35.6 & 4.7 \\
     \bottomrule
     \label{tab:b}
    \end{tabular}
\end{table}

\subsection{Vergleich mit Theorie}
Es werden die theoretischen Werte für $\nu^+$ und $\nu^-$, sowie für die Anzahl der Maxima innerhalb einer Schwebung $n_t$
berechnet und in der folgenden Tabelle mit den experimentell bestimmten Werten verglichen. Dabei geben die Indizes $e$ die jeweiligen
experimentellen Werte, und $t$ die theoretischen Werte an. 
In der Theorie lässt sich $n_t$ über die Formel \begin{equation}
    n_t = \frac{\nu_{t}^{+} + \nu_{t}^{-}}{2(\nu_{t}^{-} - \nu_{t}^{+})}
\end{equation}
angeben. $\nu_{t}^{+}$ und $\nu_{t}^{-}$ sind dabei mit den Formeln \ref{eqn:frq1} und \ref{eqn:frq2} zu berechnen.
Da $\nu^+$ keine $C_k$ Abhängiigkeit hat, und es somit nur jeweils einen theoretischen und experimentellen Wert gibt, wird
dieser Vergleich unterhalb der Tabelle nur einmal aufgeführt.
\begin{table}
    \centering
    \caption{Vergleich mit Theoriewerten}
    \label{tab:theo}
    \begin{tabular}{c c c c c}
    \toprule
    $C_k / \si{\ohm}$ & $\nu_{e}^- / \si{\kilo\hertz}$ & $\nu_{t}^- / \si{\kilo\hertz}$ & $n_e$ & $n_t$ \\
    \midrule
     2.19 & 46.5 & 47.6 &  3.5 &  3.6 \\
     2.86 & 44.2 & 45.2 &  5.0 &  4.5 \\
     4.74 & 41.1 & 41.8 &  6.5 &  6.8 \\
     6.86 & 39.5 & 40.1 &  9.5 &  9.4 \\
     8.18 & 38.9 & 39.5 & 12.0 & 11.1 \\
     9.99 & 38.4 & 38.9 & 13.0 & 13.3 \\
    12.00 & 37.9 & 38.5 & 16.5 & 15.8 \\
    \bottomrule
    \end{tabular}
\end{table}

$\nu_e^+$ = $35.6 \si{\kilo\hertz}$

$\nu_t^+$ = $36.1 \si{\kilo\hertz}$

Die Theoriewerte die Stromstärkeamplituden $I_t^+$ und $I_t^-$ sind in der folgenden Tabelle als Vergleichswerte
neben den experimentell bestimmten Werten $I_e^+$ und $I_e^-$ eingetragen: 

\begin{table}
    \centering
    \caption{Stromstärkeamplituden}
    \label{tab:I}
    \begin{tabular}{c c c c c}
    \toprule
    $C_k / \si{\nano\farad}$ & $I_e^+ / \si{\ampere}$ & $I_t^+ / \si{\ampere}$ & $I_e^- / \si{\ampere}$ & $I_t^- / \si{\ampere}$ \\
    \midrule
     2.19 & 0.098 & 0.104 & 0.083 & 0.076 \\
     2.86 & 0.098 & 0.104 & 0.085 & 0.081 \\
     4.74 & 0.098 & 0.104 & 0.089 & 0.089 \\
     6.86 & 0.098 & 0.104 & 0.092 & 0.093 \\
     8.18 & 0.098 & 0.104 & 0.092 & 0.095 \\
     9.99 & 0.098 & 0.104 & 0.092 & 0.096 \\
    12.00 & 0.098 & 0.104 & 0.089 & 0.097 \\
    \bottomrule
    \end{tabular}
\end{table}




