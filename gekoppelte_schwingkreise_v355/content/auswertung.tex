\section{Auswertung}
\label{sec:Auswertung}

Es wurde für die Versuchsreihe die Schaltung 2 verwendet. Deren Komponenten haben folgende Werte:

$L$ = $23,954 \si{\milli\henry}$

$C$ = $0,793 \si{\nano\farad}$

$C_\text{spule}$ = $0,028 \si{\nano\farad}$

$R$ = $48 \si{\ohm}$

Die Kapazitäten $C_k$ der Koppelkondensatoren sind hierbei $2,19 \si{\nano\farad}$, $2,86 \si{\nano\farad}$,
$4,74 \si{\nano\farad}$, $6,86 \si{\nano\farad}$, $8,18 \si{\nano\farad}$, $9,99 \si{\nano\farad}$ und
$12 \si{\nano\farad}$.


Das Frequenzverhältnis $n$ lässt sich über die Anzahl der Maxima der Schwingungen innerhalb einer Schwebung 
ausdrücken. Im Folgenden werden die experimentell ermittelten Verhältnisse mit den theoretisch errechneten verglichen.
In der Theorie lässt sich $n_t$ über die Formel \begin{equation}
    n_t = \frac{\nu_{t}^{+} + \nu_{t}^{-}}{2(\nu_{t}^{+} - \nu_{t}^{-})}
\end{equation}
angeben. $\nu_{t}^{+}$ und $\nu_{t}^{-}$ sind dabei mit den Formeln \ref{eqn:frq1} und \ref{eqn:frq2} zu berechnen.


\begin{table}
    \centering
    \caption{aufgabe a}
    \label{tab:a}
    \begin{tabular}{c c c}
        \toprule
        {$\omega / \si{\hertz}$} & {$C_k / \si{\nano\farad}$} & {$n_\text{amplituden}$} \\
        \midrule
        710 &  2.19 &  7 \\  
        710 &  2.86 & 10 \\
        710 &  4.74 & 13 \\
        710 &  6.86 & 19 \\
        710 &  8.18 & 24 \\
        710 &  9.99 & 26 \\
        317 & 12.00 & 33 \\
        \bottomrule
    \end{tabular}
\end{table}


Für die unterschiedlichen Kopplungskapazitäten sind im Folgenden die Frequenzen der jeweiligen
Fundamentalschwingungen $\nu^+$ und $\nu^-$, wie auch die dabei jeweils auftretenden maximalen
Spannungsamplituden $U^+$ und $U^-$.

\begin{table}
    \centering
    \caption{Frequenzen der Fundamentalschwingungen}
    \label{tab:b}
    \begin{tabular}{c c c c c}
    \toprule
    {$C_k / \si{\nano\farad}$} & {$\nu^- / \si{\kilo\hertz}$} & {$U^-$} & {$\nu^+ / \si{\kilo\hertz}$} & {$U^+$} \\
    \midrule
    12.00 & 37.9 & 4.3 & 35.6 & 4.7 \\
     9.99 & 38.4 & 4.4 & 35.6 & 4.7 \\
     8.18 & 38.9 & 4.4 & 35.6 & 4.7 \\
     6.86 & 39.5 & 4.4 & 35.6 & 4.7 \\
     4.74 & 41.1 & 4.3 & 35.6 & 4.7 \\
     2.86 & 44.2 & 4.1 & 35.6 & 4.7 \\
     2.19 & 46.5 & 4.0 & 35.6 & 4.7 \\
     \bottomrule
    \end{tabular}
\end{table}

Es werden die theoretischen Werte für $\nu^+$ und $\nu^-$, sowie für die Anzahl der Maxima innerhalb einer Schwebung $n_t$
berechnet und in der folgenden Tabelle mit den experimentell bestimmten Werten verglichen. Dabei geben die Indizes $e$ die jeweiligen
experimentellen Werte, und $t$ die theoretischen Werte an. 
Es wurden im Experiment nicht die Maxima, sondern alle Amplituden gezählt, wodurch als Näherung nur ${n_\text{amplituden}}/2$ = $n_e$
angenommen werden kann.
Da $\nu^+$ keine $C_k$ Abhängiigkeit hat, und es somit nur jeweils einen theoretischen und experimentellen Wert gibt, wird
dieser Vergleich unterhalb der Tabelle nur einmal aufgeführt.

\begin{table}
    \centering
    \caption{Vergleich mit Theoriewerten}
    \label{tab:theo}
    \begin{tabular}{c c c c c}
    \toprule
    $C_k$ & $\nu_{e}^-$ & $\nu_{t}^-$ & $n_e$ & $n_t$ \\
    \midrule
     2.19 & 46.5 & 47.6 &  3.5 &  3.6 \\
     2.86 & 44.2 & 45.2 &  5.0 &  4.5 \\
     4.74 & 41.1 & 41.8 &  6.5 &  6.8 \\
     6.86 & 39.5 & 40.1 &  9.5 &  9.4 \\
     8.18 & 38.9 & 39.5 & 12.0 & 11.1 \\
     9.99 & 38.4 & 38.9 & 13.0 & 13.3 \\
    12.00 & 37.9 & 38.5 & 16.5 & 15.8 \\
    \bottomrule
    \end{tabular}
\end{table}



