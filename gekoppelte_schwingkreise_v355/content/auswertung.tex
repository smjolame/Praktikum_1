\section{Auswertung}
\label{sec:Auswertung}

Es wurde für die Versuchsreihe die Schaltung 2 verwendet. Deren Komponenten haben folgende Werte:

$L$ = $23,954 \si{\milli\henry}$

$C$ = $0,793 \si{\nano\farad}$

$R$ = $48 \si{\ohm}$

Die Kapazitäten $C_k$ der Koppelkondensatoren sind hierbei $2,19 \si{\nano\farad}$, $2,86 \si{\nano\farad}$,
$4,74 \si{\nano\farad}$, $6,86 \si{\nano\farad}$, $8,18 \si{\nano\farad}$, $9,99 \si{\nano\farad}$ und
$12 \si{\nano\farad}$.

\subsection{Frequenzverhältnis zwischen Schwingung und Schwebung}

Das Frequenzverhältnis $n$ lässt sich über die Anzahl der Amplituden der Schwingungen innerhalb einer Schwebung 
ausdrücken. Im Folgenden werden die experimentell ermittelten Verhältnisse mit den theoretisch errechneten verglichen.
In der Theorie lässt sich $n_t$ über die Formel \begin{equation}
    n_t = \frac{\nu_{t}^{+} + \nu_{t}^{-}}{2(\nu_{t}^{+} - \nu_{t}^{-})}
\end{equation}
angeben. $\nu_{t}^{+}$ und $\nu_{t}^{-}$ sind dabei mit den Formeln \ref{eqn:frq1} und \ref{eqn:frq2} zu berechnen.


\begin{table}
    \centering
    \caption{aufgabe a}
    \label{tab:a}
    \begin{tabular}{c c c}
        \toprule
        {$\omega / \si{\hertz}$} & {$C_k / \si{\nano\farad}$} & {$n_\text{amplituden}$} \\
        \midrule
        710 &  2.19 &  7 \\  
        710 &  2.86 & 10 \\
        710 &  4.74 & 13 \\
        710 &  6.86 & 19 \\
        710 &  8.18 & 24 \\
        710 &  9.99 & 26 \\
        317 & 12.00 & 33 \\
        \bottomrule
    \end{tabular}
\end{table}

[plot, der $n_t$ und $n_e$ vergleicht]




\begin{table}
    \centering
    \caption{werte b}
    \label{tab:b}
    \begin{tabular}{c c c c c}
    \toprule
    {$C_k / \si{\nano\farad}$} & {$\omega_{\pi} / \si{\kilo\hertz}$} & {$U_{\pi}$} & {$\omega_0 / \si{\kilo\hertz}$} & {$U_0$} \\
    \midrule
    12.00 & 37.9 & 4.3 & 35.6 & 4.7 \\
     9.99 & 38.4 & 4.4 & 35.6 & 4.7 \\
     8.18 & 38.9 & 4.4 & 35.6 & 4.7 \\
     6.86 & 39.5 & 4.4 & 35.6 & 4.7 \\
     4.74 & 41.1 & 4.3 & 35.6 & 4.7 \\
     2.86 & 44.2 & 4.1 & 35.6 & 4.7 \\
     2.19 & 46.5 & 4.0 & 35.6 & 4.7 \\
     \bottomrule
    \end{tabular}
\end{table}
