\section{Theorie}
\label{sec:Theorie}

Bei einem gekoppeltem Schwingkreis handelt es sich um zwei Schwingkreise, bei denen im Laufe der Schwingung
die Energie von einen in den anderen Schwingkreis übertagen werden kann. Ein einfachen Beispiel für ein gekoppeltes
System sind zwei Pendel, welche durch eine Feder miteinander verbunden sind. [Bild] Dabei dient die Feder als Medium für die hin- und herübertragung der Energie der einzelnen Pendel.
Auf diese Weise würde das das System, bei Vernachlässigung von dissipativen Effekten, endlos weiter Schwingen.
In dem Versuch werden kapazitiv gekoppelte LC-Schwingkreise betrachtet, welche im Prinzip wie in Abbildung [Bild]
aufgebaut sind. Dabei sind die Kirchhoffschen Regeln wichtige Eigenschaften von elektrischen Schaltkreisen. 

1.  Die Summe der zufließenden Ströme in einem Knotenpunkt eines Stromkreises entspricht
    der Summe der abfließenden Ströme.

2.  Die Summe aller Spannungen innerhalb einer Masche eines geschlossenen Stromkreises
    ist null.

Aus diesem System [Bild] lassen sich mit Hilfe der Kirchhoffschen Regeln und den Zusammenhängen
\begin{equation*}
    U_C=\frac{1}{C}\int I \symup{dt}
\end{equation*}
und
\begin{equation*}
    U_L=L\dot{I}
\end{equation*}
folgende Differentzialgleichungen aufstellen.
\begin{equation}
    L\dv[2]{}{t}(T_1+I_2)+\frac{1}{C}(I_1+I_2)=0
\end{equation}
\begin{equation}
    L\dv[2]{}{t}(T_1-I_2)+(\frac{1}{C}+\frac{1}{C})(I_1-I_2)=0
\end{equation}

Durch Einführung neuer Variablen und Addieren und Subtrahieren der beiden Gleichungen erhält man



%In knapper Form sind die physikalischen Grundlagen des Versuches, des Messverfahrens, sowie sämtliche für die Auswertung erforderlichen Gleichungen darzustellen. (Keine Herleitung)

%(eventuell die Aufgaben)

%Der Versuchsaufbau: Beschreibung des Versuchs und der Funktionsweise (mit Skizze/Bild/Foto)
