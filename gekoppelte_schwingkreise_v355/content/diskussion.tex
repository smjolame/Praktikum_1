\section{Diskussion}
\label{sec:Diskussion}

Beim Untersuchen des Frequenzverhältnisses $n$ gab es, wie in der Auswertung bereits erwähnt, die Problematik, dass nicht die Anzhal der 
Maxima innerhalb einer Schwebung, sondern alle Amplituden abgezählt wurden. Es wurde daher in allen Berechnungen angenommen, dass die 
Anzahl der Maxima genau die Hälfte der Anzahl der Amplituden ist, was nicht ganz der Realität entspricht. Der Vergleich mit der 
Theoriekurve zeigt allerdings trotzdem eine recht gute Übereinstimmung, wie in [plot] zu sehen ist.

Die Stromstärkeamplituden $I^+$ und $I^-$ weichen, wie im [plot2] dargestellt, relativ gering von den zugehörigen Theoriwerten ab.
Etwas auffällig ist hier nur der Verlauf der Theoriekurve für $I^-$, der konstant steigt, während $I^+$ den Wert $0,92\si{\ampere}$
nicht überschreitet.
%Kurze Zusammenfassung der Ergebnisse
%-Vergleich mit Literaturwerten
%-Vergleich mit verschiedenen Messverfahren
%-bei Abweichungen mögliche Ursachen finden