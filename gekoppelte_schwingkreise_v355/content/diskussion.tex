\section{Diskussion}
\label{sec:Diskussion}

Beim Untersuchen des Frequenzverhältnisses $n$ gab es, wie in der Auswertung bereits erwähnt, die Problematik, dass nicht die Anzhal der 
Maxima innerhalb einer Schwebung, sondern alle Amplituden abgezählt wurden. Es wurde daher in allen Berechnungen angenommen, dass die 
Anzahl der Maxima genau die Hälfte der Anzahl der Amplituden ist, was nicht ganz der Realität entspricht. Der Vergleich mit der 
Theoriekurve zeigt allerdings trotzdem eine recht gute Übereinstimmung, wie in Plot\ref{fig:amplituden} zu sehen ist.

Die $C_k$ abhängigen $\nu^-$ stimmen mit sehr geringen Abweichungen mit den entsprechenden Theoriewerten überein.

Die Werte $I_2$ sind experimentell nur für die Frequenzen $\nu^+$ und $\nu^-$ entsprechend als $I_2^+$ und $I_2^-$ bestimmt worden.
Die Theoriekurve gibt allerdings einen kontinuierlichen, frequenzabhängigen Verlauf von $I_2$ an. Ein Abgleich der Messwerte mit den 
Theoriewerten kann also hierbei nur an den Stellen $\nu^+$ und $\nu^-$ gemacht werden. Hierbei fällt auf, dass die Theoriewerte im 
Plot stark von denen der Tabelle abweichen. Der Grund dafür konnte nicht festgestellt werden. Die Stellen $\nu^+$ und $\nu^-$, an denen die Maxima von 
$I_2$ auftreten, stimmen allerdings recht gut mit denen der Messwerte überein, wie in Plot\ref{fig:strom} zu sehen ist.

%Kurze Zusammenfassung der Ergebnisse
%-Vergleich mit Literaturwerten
%-Vergleich mit verschiedenen Messverfahren
%-bei Abweichungen mögliche Ursachen finden