\section{Diskussion}
\label{sec:Diskussion}

Bei der Untersuchung des Photostroms in Abhängigkeit der Beschleunigungsspannung ist auffällig, dass der Zusammenhang zwischen $\sqrt{I} $
und $U$ nicht linear ist, weshalb die Ausgleichsgerade nur durch den näherungsweise linear zusammenhängenden Teil der Messwerte berechnet 
wird, wie in der Auswertung zu sehen ist. In diesem Bereich sind allerdings nur wenige Messwerte vorhanden, die Genauigkeit der Berechnung 
der jeweiligen $U_g$ hätte also durch kleinere Abstände bei der Variation der Beschleunigungsspannung erhöht werden können. 

Auch an den Messwerten des Plots \ref{fig:rip} ist zu sehen, dass die jeweiligen $U_g$ nicht sehr genau bestimmt worden sein können, da 
$U_g$ für gelbes Licht theoretisch größer als für grünes Licht sein müsste. Beim Vergleich des dadurch berechneten Verhältnisses $\frac{h}{e_0}$
mit dem Theoriewert fällt allerdings auf, dass der experimentell bestimmte Wert relativ nah am Theoriewert liegt. Es 
\begin{align*}
    \frac{h}{e_0}_\text{theo} &= \SI{-4.14e-15}{\frac{\joule}{\ampere}} \\
    \frac{h}{e_0}_\text{exp}  &= \SI{-4.5(22)e-15}{\frac{\joule}{\ampere}}
\end{align*}
lässt sich eine Abweichung von etwa $\SI{8}{\percent} $ feststellen. 
\\ Weitere Gründe für die sehr hohen Ungenauigkeiten könnten in der Messung, bzw. im Aufbau des Veruschs liegen. Die Messung muss im 
Optimalfall bei völliger Dunkelheit durchgeführt werden, da der Aufbau nicht nach außen hin abgeschirmt ist. Allerdings ist eine hinreichend
intensive Lichtquelle zum Ablesen der Werte notwenig, wobei festzustellen war, dass die angezeigten Werte des Picoamperemeters stark 
schwankten, was die Präzision des Ablesens der Werte beeinträchtigt hat. 