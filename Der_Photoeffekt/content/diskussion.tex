\section{Diskussion}
\label{sec:Diskussion}

Bei der Untersuchung des Photostroms in Abhängigkeit der Beschleunigungsspannung ist auffällig, dass der Zusammenhang zwischen $\sqrt{I} $
und $U$ nicht linear ist, wodurch die Ausgleichsgeraden und die dadurch berechneten Grenzspannungen $U_g$  der drei Wellenlängen
ungenau werden. Der Grund dafür könnte eine Messung bei zu hoher Beschleunigungsspannung sein, es wäre eventuell sinnvoller gewesen in 
kleineren Abständen mit kleineren Spannungsbeträgen zu arbeiten. 

Folglich lässt sich bei der Untersuchung der Grenzspannungen in Abhängigkeit der Lichtfrequenzen erkennen, dass die Ausgleichsgerade der 
drei Messwerte quasi keine Aussage gibt, da die drei Messwerte in Plot \ref{fig:rip} nicht ansatzweise einen linearen Zusammenhang zeigen. 
Dadurch werden auch für die über die Ausgleichsgerade zu bestimmenden Werte für $\frac{h}{e_0} $ und $A_K $ keine zuverlässigen Werte 
erhalten. Bei einem Vergleich des experimentell bestimmten Verhältnis $\frac{h}{e_0} $ mit dem Theoriewert 
\begin{align*}
    \frac{h}{e_0}_\text{theo} &= \SI{-4.14e-15}{\frac{\joule}{\ampere}} \\
    \frac{h}{e_0}_\text{exp}  &= \SI{0.2(21)e-14}{\frac{\joule}{\ampere}}
\end{align*}
lässt sich eine Abweichung von etwa $\SI{200}{\percent} $ feststellen. Außerdem ist eine verhältnismäßig große Standardabweichung sowohl bei 
$\frac{h}{e_0} $, wie auch bei $A_K $ zu sehen. Diese Werte haben, wie auch sichtlich an der Gestalt der Ausgleichsgerade zu erkennen, 
keine Aussage. 
\\ Weitere Gründe für die sehr hohen Ungenauigkeiten könnten in der Messung, bzw. im Aufbau des Veruschs liegen. Die Messung muss im 
Optimalfall bei völliger Dunkelheit durchgeführt werden, da der Aufbau nicht nach außen hin abgeschirmt ist. Allerdings ist eine hinreichend
intensive Lichtquelle zum Ablesen der Werte notwenig, wobei festzustellen war, dass die angezeigten Werte des Picoamperemeters stark 
schwankten, was die Präzision des Ablesens der Werte beeinträchtigt hat. 