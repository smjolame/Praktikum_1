\section{Theorie}
\label{sec:Theorie}
Wird eine Metalloberfläche mit Licht bestrahlt, treten Elektronen aus ihr aus. Dieses Phänomen wird Photoeffekt genannt und wird innerhalb dieses Versuches näher betrachtet.

Generell gibt es zwei Modelle, welche das Wesen des Lichts versuchen darzustellen. Trifft Licht zum Beispiel auf einen schmalen Spalt, treten Interferenzeffekte ein. Da diese Effekte nur bei Wellen bekannt sind, geht ein Modell von einer Welle als wesentliche Eigenschaft des Lichts aus.
Jedoch wurden im Rahmen des Photoeffekts Beobachtungen gemacht, welche mit dem Modell der Welle nicht vereinbar sind. Zum einen ist die Zahl der pro Zeiteinheit ausgelösten Elektronen proportional zur Lichtintensität. Zum anderen ist die Energie proportional zur Lichtfrequenz und es gibt eine Grenzfrequenz, bei welcher keine Elektronen mehr aus dem Metall austreten. Aus den Beobachtungen geht eher hervor, dass Licht aus einer Art Teilchen besteht, dessen Energien sich auf kleine Raumbereiche aufteilen. Diese kleinen Energiebereiche werden Photonen genannt und quanteln das Licht auf. Nach Einstein beschreiben diese Photonen das gleiche wie die von Planck beschriebenden Energiequanten. 

Laut Planck besteht monochromatisches Licht aus eine Menge von Photonen, welche sich geradlinig, mit einer Energie von $h\nu$, durch den Raum bewegen. Dabei ist $h$ das Plancksche Wirkungsquantum. Wenn ein Photon ein Elektron aus der Metalloberfläche lösen soll, so muss es mindestens eine Energie haben, welche höher als die Bindungsenergie zwischen Elektron und Festkörper ist. Diese zum lösen notwendige Energie wird Austrittsarbeit $A_k$ genannt. Unter der Energieerhaltung muss also die Gleichung
\begin{equation}
    h \nu = E_{kin} + A_k
\end{equation}
gelten. Dabei kann ein Photon nur jeweils ein Elektron lösen. Diese Art der Beschreibung stimmt mit den im Rahmen dieses Experiments gemachten Beobachtungen überein und erklärt diese.


%In knapper Form sind die physikalischen Grundlagen des Versuches, des Messverfahrens, sowie sämtliche für die Auswertung erforderlichen Gleichungen darzustellen. (Keine Herleitung)

%(eventuell die Aufgaben)

%Der Versuchsaufbau: Beschreibung des Versuchs und der Funktionsweise (mit Skizze/Bild/Foto)
