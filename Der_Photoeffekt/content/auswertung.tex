\section{Auswertung}
\label{sec:Auswertung}
Alle Berechnungen werden mit dem Programm \glqq Numpy" \cite{numpy}, die Unsicherheiten mit dem Modul \glqq Uncertainties" \cite{uncertainties}, die Ausgleichsrechnungen mit dem Modul \glqq Scipy" \cite{scipy} durchgeführt und die grafischen Darstellungen über das Modul \glqq Matplotlib" \cite{matplotlib} erstellt.
\subsection{Untersuchung des Photostroms bei variabler Spannung für drei Wellenlängen}

In Tabelle \ref{tab:1} sind die gemessenen Photoströme abhängig von der angelegten Beschleunigungsspannung zu sehen. 
\begin{table}
    \centering
    \caption{Photoströme bei unterschiedlichen Wellenlängen in Abhängigkeit der Beschleunigungsspannung.}
    \begin{tabular}{S S}
        \toprule
        $U_\text{gelb}\,/\,\si{\volt}$ & $I\,/\,\si{\nano\ampere}$ \\
        \midrule
           20 & 0.70 \\
           19 & 0.62 \\
           18 & 0.72 \\
           17 & 0.65 \\
           16 & 0.61 \\
           15 & 0.60 \\
           14 & 0.60 \\
           13 & 0.60 \\
           12 & 0.58 \\
           11 & 0.53 \\
           10 & 0.51 \\
           09 & 0.50 \\
           08 & 0.49 \\
           07 & 0.45 \\
           06 & 0.41 \\
           05 & 0.39 \\
           04 & 0.32 \\
           03 & 0.24 \\
           02 & 0.18 \\
           01 & 0.11 \\
           00 & 0.01 \\
        -0.24 & 0.00 \\
        \bottomrule
    \end{tabular}
    \bigskip
    \begin{tabular}{S S}
        \toprule
        $U_\text{grün}\,/\,\si{\volt}$ & $I\,/\,\si{\nano\ampere}$ \\
        \midrule
            8 & 0.68 \\
            7 & 0.70 \\
            6 & 0.65 \\
            5 & 0.56 \\
            4 & 0.42 \\
            3 & 0.34 \\
            2 & 0.24 \\
            1 & 0.12 \\
            0 & 0.02 \\
        -0.22 & 0.00 \\
              &      \\    
              &      \\     
              &      \\     
              &      \\     
              &      \\     
              &      \\     
              &      \\    
              &      \\     
              &      \\     
              &      \\     
              &      \\     
              &      \\     
        \bottomrule        
    \end{tabular}
    \bigskip
    \begin{tabular}{S S}
        \toprule
        $U_\text{violet}\,/\,\si{\volt}$ & $I\,/\,\si{\nano\ampere}$ \\
        \midrule
            8 & 2.10 \\
            7 & 1.95 \\
            6 & 1.86 \\
            5 & 1.62 \\
            4 & 1.20 \\
            3 & 0.84 \\
            2 & 0.62 \\
            1 & 0.36 \\
            0 & 0.20 \\
        -0.94 & 0.00 \\
              &      \\    
              &      \\     
              &      \\     
              &      \\     
              &      \\     
              &      \\     
              &      \\    
              &      \\     
              &      \\     
              &      \\     
              &      \\     
              &      \\ 
        \bottomrule
    \end{tabular}
    \label{tab:1}
\end{table}
\FloatBarrier

Die Grenzspannung $U_g$ wird anhand einer linearen Ausgleichsrechnung 
\begin{equation}
    \sqrt{I} = a\cdot U_g + b
\end{equation}
berechnet. Die Ausgleichsgeraden werden dabei bei gelbem Licht durch die ersten fünf, bei grünem und violetem Licht durch die ersten vier 
Messwerte gelegt, da für diese Intervalle jeweils ein näherungsweise linearer Zusammenhang vorliegt. 
Die Parameter der Ausgleichsrechnungen für die einzelnen Wellenlängen und die dadurch berechneten Grenzspannungen $U_g$ sind
in der Tabelle \ref{tab:Ug} dargestellt.

    \begin{table}
        \centering
        \caption{Per Ausgleichsrechnung berechnete Grenzspannungen für die einzelnen Wellenlängen.}
        \scalebox{0.7}{
        \begin{tabular}{SSSS}
            \toprule
             $\text{Wellenlänge} \,/\,\si{\nano\m}$ & $a\,/\,\si{\frac{\sqrt{\ampere}}{\volt}} $ & $b\,/\,\si{\sqrt{\ampere}} $ & $U_g\,/\,\si{\volt} $ \\
             \midrule
             578&4.7(08)e-06&3.10(14)e-06 &-0.66(031)\\
             546&6.6(10)e-06&3.20(12)e-06 &-0.48(019)\\
             434&8.1(16)e-06&1.04(019)e-05&-1.28(035)\\
        \end{tabular}
        }
        \label{tab:Ug}
    \end{table}

\FloatBarrier

In den Plots \ref{fig:gelb}, \ref{fig:gruen} und \ref{fig:violet} sind die Wurzeln der jeweiligen Photoströme in Abhängigkeit von der 
Beschleunigungsspannung $U_g$ und jeweils eine Ausgleichsgerade grafisch dargestellt.

\begin{figure}
  \centering
  \includegraphics[width=\textwidth]{build/yellow.pdf}
  \caption{Abhängigkeit der Wurzel des Photostroms von der angelegten Beschleunigungsspannung $U_g$ für gelbes Licht.}
  \label{fig:gelb}
\end{figure}
\begin{figure}
  \centering
  \includegraphics[width=\textwidth]{build/green.pdf}
  \caption{Abhängigkeit der Wurzel des Photostroms von der angelegten Beschleunigungsspannung $U_g$ für grünes Licht.}
  \label{fig:gruen}
\end{figure}
\begin{figure}
  \centering
  \includegraphics[width=\textwidth]{build/violet.pdf}
  \caption{Abhängigkeit der Wurzel des Photostroms von der angelegten Beschleunigungsspannung $U_g$ für violetes Licht.}
  \label{fig:violet}
\end{figure}
\FloatBarrier

\subsection{Untersuchung des Zusammenhangs von Grenzspannung und }

Im Folgenden zu untersuchen ist der Zusammenhang von den bereits bestimmten Grenzspannungen und den dazugehörigen Lichtfrequenzen $\nu$.
Dafür wird in einem Plot die Grenzspannung gegen die Lichtfrequenz aufgetragen und eine Ausgleichsgerade berechnet. Die Frequenz des Lichts
lässt sich dabei durch die bekannten Größen
\begin{equation}
    \nu = \frac{c}{\lambda}
\end{equation}
beschreiben.
Dies ist im Plot \ref{fig:rip} dargestellt.
\begin{figure}
    \centering
    \includegraphics[width=\textwidth]{build/grenz.pdf}
    \caption{Grenzspannung $U_g$ in Abhängigkeit der Lichtfrequenz mit Ausgleichsgerade.}
    \label{fig:rip}
\end{figure}
\FloatBarrier

Die lineare Ausgleichsrechnung hat dabei die Form 
\begin{equation}
    U_g = \nu\cdot \frac{h}{e_0} - \frac{A_K}{e_0}
\end{equation}

Dadurch ergeben sich die Parameter 
\begin{align*}
    \frac{h}{e_0} &= \SI{-4.5(22)e-15}{\frac{\joule}{\ampere}} \\
    A_K           &= \SI{1.9(13)}{\electronvolt}\,.
\end{align*}
