\section{Auswertung}
\label{sec:Auswertung}
Alle Berechnungen werden mit dem Programm \glqq Numpy" \cite{numpy}, die Unsicherheiten mit dem Modul \glqq Uncertainties" \cite{uncertainties}, die Ausgleichsrechnungen mit dem Modul \glqq Scipy" \cite{scipy} durchgeführt und die grafischen Darstellungen über das Modul \glqq Matplotlib" \cite{matplotlib} erstellt.


\begin{table}
\centering
\caption{Messwerte bei maximalen Heizstrom $I_f=\SI{2.3}{\A}$.}
\begin{tabular}[t]{S S}
\toprule
{$U_A \:/\: \si{\V} $} & {$I_A \:/\: \si{\A}$}  \\
\midrule
0 & 0.000   \\
2 & 0.004   \\
4 & 0.010   \\
6 & 0.015   \\
8 & 0.020   \\
10 & 0.027  \\
12 & 0.035  \\
14 & 0.042  \\
16 & 0.050  \\
18 & 0.060  \\
20 & 0.070  \\
22 & 0.081  \\
24 & 0.092  \\
26 & 0.103  \\
\bottomrule
\end{tabular}
\begin{tabular}[t]{S S}
\toprule
{$U_A \:/\: \si{\V} $} & {$I_A \:/\: \si{\A}$}  \\
\midrule
28 & 0.114  \\
30 & 0.122  \\
40 & 0.189  \\
50 & 0.263  \\
60 & 0.336  \\
70 & 0.408  \\
80 & 0.479  \\
90 & 0.536  \\
100 & 0.664 \\
110 & 0.684 \\
120 & 0.765 \\
130 & 0.821 \\
140 & 0.869 \\
150 & 0.906 \\ 
\bottomrule
\end{tabular}
\end{table}



\begin{table}
\centering
\caption{Messwerte bei Heizstrom $I_f=\SI{1.8}{\A}$.}
\begin{tabular}[t]{S S}
\toprule
{$U_A \:/\: \si{\V} $} & {$I_A \:/\: \si{\A}$}  \\
\midrule
0 & 0.000    \\
2 & 0.000    \\
4 & 0.002    \\
6 & 0.004    \\
8 & 0.006    \\
10 & 0.008       \\
12 & 0.010       \\
14 & 0.011       \\
16 & 0.012       \\
18 & 0.013       \\
20 & 0.015       \\
\bottomrule
\end{tabular}
\begin{tabular}[t]{S S}
\toprule
{$U_A \:/\: \si{\V} $} & {$I_A \:/\: \si{\A}$}  \\
\midrule
22 & 0.016       \\
24 & 0.016       \\
26 & 0.017       \\
28 & 0.018       \\
30 & 0.018       \\
40 & 0.020       \\
50 & 0.021       \\
60 & 0.022       \\
70 & 0.022       \\
80 & 0.021       \\
\bottomrule
\end{tabular}
\end{table}
%Messwerte: Alle gemessenen physikalischen Größen sind übersichtlich darzustellen.
%
%Auswertung:
%Berechnung der geforderten Endergebnisse
%mit allen Zwischenrechnungen und Fehlerformeln, sodass die Rechnung nachvollziehbar ist.
%Eine kurze Erläuterung der Rechnungen (z.B. verwendete Programme)
%Graphische Darstellung der Ergebnisse