\section{Durchführung}
\label{sec:Durchführung}

Für den Aufbau werden lediglich die in der Theorie bereits beschriebene Hochvakuumdiode und zwei Netzteile benötigt. Eines der Netzteile wird an 
den Wolframdraht angeschlossen, um die für die Erzeugung freier Elektronen benötigte Heizspannung zu erzeugen. Mit dem anderen Netzteil wird eine 
Spannung zwischen Kathode und Anode aufgebaut, wobei die Anode positiv geladen sein muss, damit die Elektronen in deren Richtung beschleunigt 
werden.\\
Es werden zwei Messreihen aufgenommen, bei denen jeweils eine konstante Heizspannung besteht und die Anodenspannung variiert wird. 
Dabei wird von $\SI{0}{\volt} $ bis $\SI{30}{\volt} $ die Spannung immer in Zweierschritten erhöht. von $\SI{30}{\volt} $ bis 
$\SI{150}{\volt} $ wird die Saugspannung dann in Zehnerschritten erhöht. Für die erste Messreihe wird eine Heizspannung von 
$\SI{4}{\volt} $ bei $\SI{1.8}{\ampere} $ verwendet und für die zweite Messreihe $\SI{5}{\volt} $ bei $\SI{2.3}{\ampere} $.