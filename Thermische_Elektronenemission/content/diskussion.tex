\section{Diskussion}
\label{sec:Diskussion}

Der in \ref{sub:Langmuir-Schottkysches Raumladungsgesetz} berechnete Exponent von $n=\num{1.376(012)}$ weicht um $\SI{8.2(8)}{\percent}$ von dem Wert des Langmuir-Schottkyschen Raumladungsgesetz $n=\num{1.5}$ ab. Die Abweichung ist gering genug, sodass die Gültigkeit des Gesetztes als gegeben angenommen werden kann.
Der Literaturwert der Austrittsarbeit für Wolfram beträgt $W_a^{lit}=\SI{4.54}{\eV}$. Der experimentell bestimmte Wert von $W_a=\SI{3.901(021)}{\eV}$ weicht damit um $\SI{14.1(5)}{\percent}$ von dem Literaturwert ab. 
Die bestimmten Kathodentemperaturen lassen sich nicht vergleichen, da keine Temperaturmessungen vorgenommen worden sind. Die Kathodentemperatur hätte auch über eine Analyse des Anlaufstromgebietes erfolgen können, wodurch die Ergebnisse der einzelenen Temperaturermittlungen hätten verglichen werden können. Auch über die Korrektheit der Sättignugsströme kann keine Aussage getroffen werden, da keine Vergleichswerte vorliegen. Trotz der relativ geringen Abweichungen lässt sich sagen, dass über eine größere
Anzahl an Kennlinienaufnahmen die Ergebnisse eine höhere Aussagekraft erlangt hätten. 
%Kurze Zusammenfassung der Ergebnisse
%-Vergleich mit Literaturwerten
%-Vergleich mit verschiedenen Messverfahren
%-bei Abweichungen mögliche Ursachen findens