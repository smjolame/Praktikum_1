\section{Diskussion}
\label{sec:Diskussion}


Die beiden in dem Versuch untersuchten Größen sind die Schallgeschwindigkeit in Acryl und der spezifische Schwächungskoeffizient 
des Acryls. Die unterschiedlichen Messungen der Schallgeschwindigkeit weichen zueinander nur um $\SI{0.2}{\percent} $ ab. 
Allerdings musste ein Messwert $t$ bei der Messung zum Echo-Impuls-Verfahren vernachlässigt werden, da bei diesem der falsche 
Impuls gemessen wurde. Daher erstreckt sich auch die Ausgleichsgerade bei diesem Verfahren über nur drei Messwerte, während 
beim Durchschallungsverfahren vier Werte in Betracht gezogen werden konnten. Die sehr ähnlichen Ergebnisse für die Schallgeschwindigkeit
lassen aber darauf schließen, dass dies keinen großen Fehler verursacht. 

Bei der Messung des Schwächungskoeffizienten ist ein Fehler von etwa $\SI{32}{\percent} $ zu beobachten. Über die Richtigkeit der 
Messung kann aufgrund eines Fehlenden Theoriewertes keine Aussage gemaht werden. Allerdings wäre eine präzisere Messung vermutlich 
mit mehr als drei Messwerten sinnvoll gewesen. 
%Kurze Zusammenfassung der Ergebnisse
%-Vergleich mit Literaturwerten
%-Vergleich mit verschiedenen Messverfahren
%-bei Abweichungen mögliche Ursachen finden