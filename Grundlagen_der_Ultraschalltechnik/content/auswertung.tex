\section{Auswertung}
\label{sec:Auswertung}
Alle Berechnungen werden mit dem Programm \glqq Numpy" \cite{numpy}, die Unsicherheiten mit dem Modul \glqq Uncertainties" \cite{uncertainties}, die Ausgleichsrechnungen mit dem Modul \glqq Scipy" \cite{scipy} durchgeführt und die grafischen Darstellungen über das Modul \glqq Matplotlib" \cite{matplotlib} erstellt.

In den foglenden Tabellen \ref{tab:impulsecho} und \ref{tab:durchschallung} sind die Messwerte des Echo-Impuls-Verfahrens und 
des Durchschallungsverfahrens dargestellt. Die Indices $1$ und $2$ kennzeichnen in zeitlicher Reihenfolge die unterschiedlichen Peaks.

\begin{table}
    \centering
    \caption{Messergebnisse des Echo-Impuls-Verfahrens.}
    \begin{tabular}{c c c c c}
        \toprule
        $l\,/\,\si{\milli\metre}$ & $U_1\,/\,\si{\volt}$ & $t_1\,/\,\si{\micro\second}$ & $U_2\,/\,\si{\volt}$ & $t_2\,/\,\si{\micro\second}$  \\
        \midrule
          40.65 & 0.455 &  1.4 & 0.019 & 30.5 \\
          80.45 & 1.195 &  1.4 & 0.063 & 59.6 \\
         120.06 & 1.188 &  1.4 & 0.145 & 88.7 \\
        \bottomrule
    \end{tabular}
    \label{tab:impulsecho}
\end{table}

\begin{table}
    \centering
    \caption{Messergebnisse des Durchschallungsverfahrens.}
    \begin{tabular}{c c}
        \toprule
        l\,/\,\si{\milli\metre} & t\,/\,\si{\micro\second} \\
        \midrule
         40,65 & 15.8 \\
         80.45 & 30.4 \\
        120.06 & 45.0 \\
        121.10 & 45.1 \\
        \bottomrule
    \end{tabular}
    \label{tab:durchschallung}
\end{table}

Anhand dieser Messwerte kann für beide Verfahren mit einer Ausgleichsrechnung die Schallgeschwindigkeit $c_\text{acryl}$ in Acryl bestimmt werden.
Die gemessene Zeit $t$ beim Impuls-Echo-Verfahren muss bei der Rechnung halbiert werden, da der Impuls mit dieser Variante erst 
nach zweifachem Durchlaufen der Länge des Zylinders registriert wird.

\begin{figure}
    \centering
    \includegraphics[width=\textwidth]{build/impulsecho.pdf}
    \caption{Messwerte des Impuls-Echo-Verfahrens mit Ausgleichsgerade.}
    \label{fig:imp}
\end{figure}

\begin{figure}
    \centering
    \includegraphics[width=\textwidth]{build/durchschallung.pdf}
    \caption{Messwerte des Durchschallungsverfahrens mit Ausgleichsgerade.}
    \label{fig:durch}
\end{figure}

Die Funktion der Ausgleichsgerade hat dabei die Form 
\begin{equation}
    f(x) = a\cdot x + b 
\end{equation}

Damit gilt für das Durchschallungsverfahren 
\begin{equation}
    t = \frac{1}{c}\cdot s + b\,,
\end{equation}
und für das Impuls-Echo-Verfahren
\begin{equation}
    \frac{t}{2} = \frac{1}{c}\cdot s + b
\end{equation}

Die Parameter der Ausgleichsrechnung ergeben sich damit für beide Verfahren zu 
\begin{align}
    a_\text{D} &= \SI{365.8(21)e-6}{\s\per\m} \\
    b_\text{D} &= \SI{0.94(21)e-6}{\s\per\m} \\
    a_\text{I} &= \SI{366.5(05)e-6}{\s\per\m} \\
    b_\text{I} &= \SI{0.34(04)e-6}{\s\per\m} 
\end{align}

Die Schallgeschwindigkeit lässt sich dann durch 
\begin{equation}
    c_\text{acryl} = \frac{1}{a}\cdot 10^6
\end{equation}
berechnen. Der Parameter $b$ gibt einen systematischen Fehler an, der im Folgenden vernachlässigt werden kann. 

Die Schallgeschwindigkeit ergibt sich nach den beiden Verfahren zu 
\begin{align}
    c_\text{D} &= \SI{2734(16)}{\m\per\s} \\
    c_\text{I} &= \SI{2729(4)}{\m\per\s}
\end{align}

Desweiteren lässt sich aus den vorliegenden Messwerten der spezifische Schwächungskoeffizient $\alpha$ des Acryls ermittelt 
werden. Dieser lässt sich über den Zusammenhang 
\begin{equation}
    \ln{\bigl(\frac{I(l)}{I_0}\bigr)} = - \alpha \cdot l
\end{equation}
berechnen. Dabei beschreibt $I(x)$ die Intensität des Impulses an der Stelle $x$. Für die Bestimmung wird wieder eine Ausgleichsrechnung
durchgeführt, dessen Funktion wieder die Form 
\begin{equation}
    f(x) = a\cdot x + b 
\end{equation}  
hat. Diese Berechnung wird nur mit den Daten des Echo-Impuls-Verfahren durchgeführt.

\begin{figure}
    \centering
    \includegraphics[width=\textwidth]{build/alpha.pdf}
    \caption{Ausgleichsrechnung zur Berechnung der Dämpfungskonstante $\alpha$.}
    \label{fig:alpha}
\end{figure}

Die Parameter der Ausgleichsgerade ergeben sich dann zu 
\begin{align}
    a &= \SI{13.5(44)}{\per\metre} \\
    b &= \SI{-3.8(04)}{\per\metre}\,.
\end{align}
Der Parameter $a$ entspricht genau der Dämpfungskonstante $\alpha$, welche damit 
\begin{equation}
    \alpha = \SI{13.5(44)}{\per\m}
\end{equation}
entspricht. 
