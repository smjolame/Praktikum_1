\section{Theorie}
\label{sec:Theorie}
Schallwellen in einem Frequenzbereich von ca. $\SI{20}{\kilo\hertz}$ bis ca. $\SI{1}{\giga\hertz}$ werden als Ultraschall bezeichnet. Dabei ist Schall eine longitudinale Welle und lässt sich als Druckschwankung über
\begin{equation}
 p(x,t)=p_0+v_0Z\cos(wt-kx)
\end{equation}
ausdrücken. $Z=c\cdot \rho$ beschreibt dabei die akustische Impedanz, welche von der Dichte $\rho$ des durschallten Materials und der Schallgeschwindigkeit $c$ in dem Material abhängt.
Die Schallgeschwindigkeit hängt dabei stark von den Eigenschaften des Materials ab. Durchläuft der Schall z.B. eine Flüssigkeit, so beträgt seine Geschwindigkeit
\begin{equation}
    c_{Fl}=\sqrt{\frac{1}{\kappa \rho}} .
\end{equation}
$\kappa$ beschreibt dabei die Kompressibilität und $\rho$ die Dichte der Flüssigkeit.
Wird ein Festkörper durchschallt, so treten neben den longitudinalen Anteilen auch transversale Wellen auf, wodurch sich die Schallgeschwindigkeit im Material, abhängig von dem Elastzitätsmodul $E$, über
\begin{equation}
    c_{Fe}=\sqrt{\frac{E}{\rho}}
\end{equation}
berechnen lässt.
Es ist zu beachten, dass die Schallgeschwindigkeiten in Festkörpern generell richtungsabhängig sind. 

Beim Eintreten des Schalls geht durch Absorption ein Teil der Energie verloren. Über 
\begin{equation}
    I(x)=I_0e^{-\alpha x}
    \label{eqn:abs}
\end{equation}
lässt sich die Abnahme der anfänglichen Intensität $I_0$ beim tiefer dringen in das Material bestimmen. Dabei beschreibt $x$ die Strecke, die der Schall in dem Festkörper zurückgelegt hat und $\alpha$ den Absorptionskoeffizienten der Schallamplitude. Dieser Absorptionskoeffizient ist bei Luft recht hoch. Desshalb wird zwischen Schallquelle und dem zu untersuchenden Material oft ein Kontaktmittel gegeben, sodass möglichst wenig Schall vor Erreichen des Materials absorbiert wird. 

Erreicht der Schall eine Grenzfläche, so wird ein Teil des auftreffenden Schalls reflektiert. Das Verhältnis $R$ zwischen der einfallenden und reflektierter Schallintensität lässt sich abhängig von der akustischen Impedanz $Z$ über
\begin{equation}
    R=\left(\frac{Z_1-Z_2}{Z_1+Z_2}\right)^2
\end{equation}
bestimmen. Das Verhältnis $T$ zwischen eingehender und transmittieten Schallintensität kann dann über $T=1-R$ berechnet werden.

Bei der Erzeugung von Ultraschall können piezoelektrische Kristalle verwendet werden. Dabei wird der piezo-elektrische Effekt ausgenutzt. Dieser sorgt dafür, dass sich ein in einem elektrischen Wechselfeld befindlicher Kristall, zu Schwingungen anregen lässt. Dabei erzeugt der Kristall Ultraschallwellen. Wird dabei die Eigenfrequenz des Kristalls und die Frequenz des Wechselfeldes synchronisiert, tritt Resonanz auf und es können hohe Schallamplituden generiert werden. Neben der Erzeugung von Ultraschall kann dieser Effekt auch zum Messen von Schall verwendet werden. Trifft Schall auf den Kristall, so wird er zum Schwingen angeregt. Diese Schwingungen können dann wieder in elektrische Signale umgewandelt und verzeichnet werden.

Zum Untersuchen verschiedener Materialien kommen verschiedene Verfahren zum Einsatz. Dabei wird ein kurzzeitiger Schallimpuls aussgesendet, welcher von einem Empfänger gegestriert werden kann. Über die Art der Veränderung zwischen ausgehenden und eingehenden Schallsignal lassen sich Schlüsse über das Innere einer Probe ziehen. Je nach Verfahren lassen sich unterschiedlich genaue Informationen über die Probe sammeln. Zum einen das \textit{Durschallungs-Verfahren}, bei dem Sender und Empfänger gegenüber platziert werden und in dessen Mitte sich die zu untersuchende Probe befindet. So lassen sich Aussagen darüber treffen, ob der Schall beim Durchtreten der Probe auf ein Hindernis gestoßen ist. Dabei kann jedoch keine Aussage über den Ort des Hindernisses getroffen werden. 
Bei einem anderen Verfahren, dem \textit{Impuls-Echo-Verfahren}, ist der Sender zugleich der Empfänger. Dabei werden die Schallwellen beim durchtreten der Probe, von potentiellen Hindernissen reflektiert und auf diese Weise regestriert. Bei diesem Verfahren lassen sich Aussagen über den Ort der Fehlstelle treffen, da mit Kenntnis der Laufzeit $t$ und der Schallgeschwindigkeit im Material $c$ die Lage des Hindernisses über
\begin{equation}
    s=\frac{1}{2}ct
\end{equation}
bestimmt werden kann.




%In knapper Form sind die physikalischen Grundlagen des Versuches, des Messverfahrens, sowie sämtliche für die Auswertung erforderlichen Gleichungen darzustellen. (Keine Herleitung)

%(eventuell die Aufgaben)

%Der Versuchsaufbau: Beschreibung des Versuchs und der Funktionsweise (mit Skizze/Bild/Foto)
