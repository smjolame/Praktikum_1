\section{Durchführung}
\label{sec:Durchführung}

\subsection{Versuchsaufbau}
Für die Durchführung des Versuchs werden ein Ultraschallechoskop, Ultraschallsonden verschiedener Frequenzen und ein Rechner zur Aufnahme bzw. Analyse der Daten verwendet. Das Ultraschallechoskop kann je nach verwendeten Verfahren eingestellt werden. Zum minimieren von Verlusten durch Luftabsorption wird bidestiliertes Wasser als Kontaktmittel verwendet.

\subsection{Versuchsdurchführung}
Zuerst wird die Schallgeschwindigkeit innerhalb eines Acrylzylinders gemessen. Dazu wird der Acrylzylinder von oben mit einer $\SI{2}{\mega\hertz}$-Sonde gekoppelt. Zwischen Sonde und Zylinder wird bidestiliertes Wasser gefügt. Über das Impuls-Echo-Verfahren wird ein A-Scan durchgeführt. Darüber werden die Laufzeiten und Amplituden des ersten und zweiten reflektierten Impuls bestimmt. Die Länge des Zylinders wird mit Hilfe einer Schieblehre bestimmt. Dieser Vorgang wird für drei weiter Zylinder verschiedener Längen wiederholt. Dabei wird bei einem Messdurchlauf ein Zylinder durch die Zusammensetzung zwei kleinerer Zylinder realisiert. 

Zuletzt wird mit den gleichen Zylindern die Schallgeschwindigkeit mit dem Durchschallungs-Verfahren bestimmt. Der Zylinder wird dafür an den beiden Stirnseiten mit den jeweiligen Sonden gekoppelt. Wieder wird mit einem A-Scan die Laufzeit gemessen, die der Schall zum durchqueren der Zylinder benötigt. Auch das wird mit den vier verschiedenen Zylinderlängen wiederholt.


%Was wurde gemessen bzw. welche Größen wurden variiert?