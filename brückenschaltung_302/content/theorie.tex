\section{Theorie}
\label{sec:Theorie}

\subsection{Allgemeine Funktionsweise}

Brückenschaltungen werden im Allgemeinen hauptsächlich zur Messung unbekannter Größen 
gebraucht, die sich eideutig auf ihren Widerstand reduzieren lassen. Darunter fallen für
die folgende Versuchsreihe Ohm'sche Widerstände, Kapazitäten von Kondenstatoren und 
Induktivitäten von Spulen. Der grundsätzliche Aufbau [ref abb1] funktioniert dabei, wie folgt:
Zwischen den Stellen $C$ und $D$ wird eine Spannung abgegriffen, die von den in der Masche 
enthaltenen Widerständen abhängig ist. Anhand dessen lässt sich unter der Bedingung, dass 
nur einer der in dem Fall ohmschen Widerstände unbekannt ist, dieser über die Beziehung
von Spannung und Widerständen bestimmen. 
[abb1]

Um dies zu können, müssen die zwei Kirchhoff'schen Gesetze in Betracht gezogen werden:

1.  Die Summe der zufließenden Ströme inn einem Knotenpunkt eines Stromkreises entspricht
    der Summe der abfließenden Ströme.

2.  Die Summe aller Spannungen innerhalb einer Masche eines geschlossenen Stromkreises
    ist null.

Auf [ref abb1] bezogen folgt daraus:

\begin{equation}
    U = \frac{{R_2}{R_3}-{R_1}{R_4}}{({R_3}+{R_4})({R_1}{R_2})}U_s 
\end{equation}

Das bedeutet, dass die resultierende Spannung $U$ null wird, wenn 
\begin{equation}
    R_2 R_3 - R_1 R_4 = 0
\end{equation}

Wenn daher nur einer der Widerstände unbekannt ist, und sich aus beiden Summen mindestens einer
der bekannten Widerstände variieren lässt, ist damit für den Fall $U$ = 0 der unbekannte Widerstand
auszurechnen.

\subsection{komplexe Widerstände}

Bei Wechselstrom haben Kondenstatoren sowie Spulen jeweils komplexe Widerstände, bzw. Impedanzen.

Die Impedanz ist allgemein beschrieben durch

\begin{equation}
    Z = R + iX
\end{equation}

, wbei der Realteil $R$ der Wirkwiderstand, analog zum ohm'schen Widerstand und  der Imaginärteil
X der Blindwiderstand ist. 

Speziell gilt für den Widerstand eines Kondensators der Kapazität $C$ \begin{equation}
    Z_C = \frac{-i}{\omega C}
\end{equation} für den einer Spule mit Induktivität $L$ \begin{equation}
    Z_L = i\omega L
\end{equation} und für einen ohm'schen Widerstand \begin{equation}
    Z_R = R
\end{equation}

\subsection{Spezielle Brückenschaltungen}

\subsubsection{Wheaton'sche Brücke}
[abb2]
Diese Schaltung kann sowohl mit Gleichstrom, als auch mit Wechselstrom betrieben werden, da sie ausschließlich
ohm'sche Widerstände enthält. 
Aus [eq2] folgt für $R_x$:
\begin{equation}
    R_x = R_2 \frac{R_3}{R_4}
\end{equation}

$\frac{R_3}{R_4}$ ist dabei als Potentiometer anzusehen, mit dem das Verhältnis zwischen $R_3$ und $R_4$ 
variiert werden kann. 

\subsubsection{Kapazitätsmessbrücke}
[abb3]






