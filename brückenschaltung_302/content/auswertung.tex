\section{Auswertung}
\label{sec:Auswertung}


Im folgenden werden die einzelnen Versuche ausgewertet. Dabei werden öfters der Mittelwert und der Standardfehler (SEM) des
unbekannten Bauteils \enquote{Wert x} bestimmt. Die jeweilige Berechnung des Mittelwertes und des Standardfehlers wird mithilfe 
der Progamme \enquote{Numpy} \cite{numpy} und \enquote{Scipy} \cite{scipy} durchgeführt. Dabei werden die progamminternen Funktionen 
\enquote{numpy.mean} für die Mittelwertberechnung und von scipy.stats \enquote{sem} für den Standardfehler des Mittelwerts verwendet.

\subsection{Wheaton'sche Brücke}
\begin{table}
    \centering
    \caption{Wert 12 }
    \label{tab:1}
    \sisetup{table-format=}
    \begin{tabular}{c| r c c S[table-format=3.2]}
        \toprule
{Messung} & {$R_2 \:/\: \si{\ohm} $} & {$R_3 \:/\: \si{\ohm}  $} & {$R_4 \:/\: \si{\ohm}  $} & {$R_x \:/\: \si{\ohm} $}\\
        \midrule
 1 & 664 & 334 & 666 & 212.26\\
 2 & 1000 & 268 & 732  &  300.39\\
 3 & 332 & 452 & 547 & 262.08\\

      \bottomrule
      \\
    & \multicolumn{4}{l} {Mittelwert: $ R_x=(\num{258.24+-25.51})\si{\ohm}$}\\
    \end{tabular}
\end{table}



\begin{table}
    \centering
    \caption{Wert 13}
    \label{tab:2}
    \sisetup{table-format=}
    \begin{tabular}{c| r c c c}
        \toprule
       {Messung} &  {$R_2 \:/\: \si{\ohm} $} & {$R_3 \:/\: \si{\ohm}  $} & {$R_4 \:/\: \si{\ohm}  $} & {$R_x \:/\: \si{\ohm} $}\\
        \midrule
 1 & 332 & 390 & 610 & 332.99\\
 2 & 1000 & 231 & 769 & 366.12\\
 3 & 664 & 283 & 717 & 274.34\\

      \bottomrule
            \\
    & \multicolumn{4}{l} {Mittelwert: $ R_x=(\num{324.49+-26.83})\si{\ohm}$}\\
    \end{tabular}
\end{table}

Mit Hilfe der Gleichung \eqref{eqn:wheaton} und den Daten der Tabellen \ref{tab:1} und \ref{tab:2} lassen sich 
die jeweils unbekannten Widerstände \enquote{Wert 12} und \enquote{Wert 13} bestimmen, welche schon am Boden der jeweiligen
Tabelle aufgeführt sind.


\subsection{Kapazitätsmessbrücke}

\begin{table}
    \centering
    \caption{Wert 9}
    \label{tab:3}
    \sisetup{table-format=}
    \begin{tabular}{c| r c c c c S[table-format=4.2]}
        \toprule
       {Messung} &  {$R_2 \:/\: \si{\ohm} $} & {$R_3 \:/\: \si{\ohm}  $} & {$R_4 \:/\: \si{\ohm}  $} & {$C_2 \:/\: \si{\nano\farad}  $} & {$R_x \:/\: \si{\ohm} $} &  {$C_x \:/\: \si{\nano\farad}$}\\
        \midrule
 1 & 664 & 455 & 545 & 399 & 554.35 & 477.92\\
 2 & 1000 & 435 & 565 & 992 & 769.91 & 1288.46\\
 3 & 332 & 655 & 345 & 992 & 630.32 & 522.51\\

      \bottomrule
            \\
    & \multicolumn{6}{l} {Mittelwert: $ R_x=(\num{651.53+-63.12})\si{\ohm}$}\\
    & \multicolumn{6}{l} {Mittelwert: $ C_x=(\num{762.96+-263.06})\si{\nano\farad}$}\\
    \end{tabular}
\end{table}

Die unbekannten Werte $C_x$ und $R_x$ des realen Kondensators \enquote{Wert 9} können über die Gleichungen \eqref{eqn:kapa1} und \eqref{eqn:kapa2}
berechnet werden. Die Werte können der Tabelle \ref{tab:3} entnommen werden. Die gemittelten Werte des realen Kondensators
sind wieder am Boden der Tabelle aufgeführt.

\subsection{Induktivitätsmessbrücke}

\begin{table}
    \centering
    \caption{Wert 16}
    \label{tab:4}
    \sisetup{table-format=}
    \begin{tabular}{c| r c c S[table-format=2.2] c}
        \toprule
       {Messung} &  {$R_2 \:/\: \si{\ohm} $} & {$R_3 \:/\: \si{\ohm}  $} & {$R_4 \:/\: \si{\ohm}  $} & {$R_x \:/\: \si{\ohm} $} &  {$L_x \:/\: \si{\milli\henry}$}\\
        \midrule
 1 & 332 & 60 & 940 & 21.19 & 1.28\\
 2 & 1000 & 60 & 940 & 63.82 & 1.28\\
 3 & 500 & 60 & 940 & 31.92 & 1.28\\
 4 & 0 & 86 & 914 & 0.00 & 1.89\\

      \bottomrule
            \\
    & \multicolumn{5}{l} {Mittelwert: $ R_x=(\num{29.23+-13.30})\si{\ohm}$}\\
     & \multicolumn{5}{l} {Mittelwert: $ L_x=(\num{1.44+-0.15})\si{\milli\henry}$}\\
    \end{tabular}
\end{table}



\begin{table}
    \centering
    \caption{Wert 19}
    \label{tab:5}
    \sisetup{table-format=}
    \begin{tabular}{c| r r c S[table-format=3.2] c}
        \toprule
       {Messung} &  {$R_2 \:/\: \si{\ohm} $} & {$R_3 \:/\: \si{\ohm}  $} & {$R_4 \:/\: \si{\ohm}  $} & {$R_x \:/\: \si{\ohm} $} &  {$L_x \:/\: \si{\milli\henry}$}\\
        \midrule
 1 & 1000 & 100 & 900 & 111.11 & 2.23\\
 2 & 500 & 60 & 940 & 31.91 & 1.28\\
 3 & 0 & 50 & 950 & 0.00 & 1.06\\

      \bottomrule
            \\
    & \multicolumn{5}{l} {Mittelwert: $ R_x=(\num{47.68+-33.03})\si{\ohm}$}\\
     & \multicolumn{5}{l} {Mittelwert: $ L_x=(\num{1.52+-0.36})\si{\milli\henry}$}\\
    \end{tabular}
\end{table}

Die zu bestimmende Induktivität der realen Spulen \enquote{Wert 16} und \enquote{Wert 19} werden mit Hilfe der Gleichung
\eqref{eqn:induk} berechnet. Dabei ist, wie schon in dem Abschnitt \ref{sec:Induktivitätsmessbrücke} erwähnt, die Induktivität der Spule $L_2=20.1\si{\milli\henry}$.
Der ohmsche Widerstand lässt sich analog zu den vorherigen Versuchen wieder mit der Gleichung
\eqref{eqn:wheaton} bestimmen. Auch hier sind die jeweiligen Werte und Ergebnisse den Tabellen \ref{tab:4} und \ref{tab:6}
zu entnehmen.

\subsection{Maxwell-Brücke}

\begin{table}
    \centering
    \caption{Wert 19}
    \label{tab:6}
    \sisetup{table-format=}
    \begin{tabular}{c| r r c c c c}
        \toprule
       {Messung} &  {$R_2 \:/\: \si{\ohm} $} & {$R_3 \:/\: \si{\ohm}  $} & {$R_4 \:/\: \si{\ohm}  $}& {$C_4 \:/\: \si{\nano\farad}  $} & {$R_x \:/\: \si{\ohm} $} &  {$L_x \:/\: \si{\milli\henry}$}\\
        \midrule

 1 & 500 & 125 & 875 & 597 & 71.43 & 37.31\\
 2 & 1000 & 78 & 922 & 399 & 84.60 & 31.12\\
 3 & 332 & 229 & 771 & 399 & 98.61 & 30.33\\


      \bottomrule
            \\
    & \multicolumn{6}{l} {Mittelwert: $ R_x=(\num{84.88+-7.85})\si{\ohm}$}\\
     & \multicolumn{6}{l} {Mittelwert: $ L_x=(\num{32.92+-2.21})\si{\milli\henry}$}\\
    \end{tabular}
\end{table}

Die Tabelle \ref{tab:6} zeigt die gemessenen Daten, aus denen erneut die Induktivität der Spule \enquote{Wert 19}
berechnet werden kann. Diesmal wird dafür, da es sich jetzt um eine Maxwell-Brücke handelt, die Gleichung \eqref{eqn:maxwell} verwendet.
Das Ergebnis ist erneut am Boden der Tabelle aufgeführt.

\subsection{Wien-Robinson-Brücke}

\begin{table}
    \centering
    \caption{Messdaten der Wien-Robinson-Brücke}
    \label{tab:7}
    \begin{tabular}{r| S[table-format=1.2]}
        \toprule
        {$f \:/\: \si{\hertz} $} & {$U_{Br} \:/\: \si{\volt} $}\\
        \midrule
20 & 1.28\\
30 & 1.24\\
40 & 1.21\\
50 & 1.14\\
100 & 0.74\\
150 & 0.40\\
200 & 0.16\\
250 & 0.08\\
300 & 0.24\\
350 & 0.34\\
400 & 0.44\\
500 & 0.60\\
1000 & 0.96\\
1500 & 1.04\\
2000 & 1.08\\
2500 & 1.10\\
3000 & 1.12\\
3500 & 1.13\\
4000 & 1.14\\
5000 & 1.14\\
10000 & 1.08\\
15000 & 1.04\\
20000 & 0.98\\
25000 & 0.96\\
30000 & 0.90\\


      \bottomrule
    \end{tabular}
\end{table}

\begin{figure}
    \centering
    \includegraphics[width=\textwidth]{build/e.pdf}
    \caption{Plot zur Wien-Robinson-Brücke}
    \label{fig:wien}
\end{figure}


Bei dem Versuch wird die Frequenzabhängigkeit einer Wien-Robinson-Brücke ermittelt. Dabei sollen die theoretischen und gemessenen Werte und Ergebnisse verglichen werden.
Die aus dem Versuch gewonnenen Daten können der Tabelle \ref{tab:7} entnommen werden. So lässt sich die Minimalfrequenz
$f_0=250\si{\hertz}$ direkt ablesen. Die Minimalfrequenz ist die Frequenz, bei der die Brückenspannung minimal wird. Die Speisspannung 
$U_S$ ist dabei bei jeder Messung $U_S=2.5 \si{\volt}$. Die anderen Daten sind in dem Abschnitt \ref{sec:wien} aufgeführt.
Es wird das Spannungsverhältnis $U_v=\frac{U_{Br}}{U_s}$ eingeführt. Dieses Spannungsverhältnis ist im Plot \ref{fig:wien}
gegen das Frequenzverhältnis $\Omega=\frac{f}{f_0}$ aufgetragen. Bei dem Berechnen der theoretischen Kurve muss dabei beachetet werden,
dass es sich in \ref{sec:wien} aufgeführten Frequenzen um Kreisfrequenzen handelt. Damit die Saklierung im Plot \ref{fig:wien} passend bleibt,
muss also die Minimalfrequenz der Theorie durch 
\begin{equation}
    f_{0}^{t}=\frac{1}{2\pi RC}
\end{equation} 
berechnet werden. So ergibt sich $f_{0}^{t}=241.14\si{\hertz}$. Der theoretische Wert für das Frequenzverhältnis ergibt sich dann aus $\Omega^{t}=\frac{f}{f_{0}^{t}}$.
Mit Gleichung \eqref{eqn:u} kann das theoretische Spannungsverhältnis $U_{v}^{t}$ in Abhängigkeit von $\Omega^{t}$ in der Grafik
\ref{fig:wien} aufgetragen werden.



%Messwerte: Alle gemessenen physikalischen Größen sind übersichtlich darzustellen.
%
%Auswertung:
%Berechnung der geforderten Endergebnisse
%mit allen Zwischenrechnungen und Fehlerformeln &  sodass die Rechnung nachvollziehbar ist.
%Eine kurze Erläuterung der Rechnungen (z.B. verwendete Programme)
%Graphische Darstellung der Ergebnisse