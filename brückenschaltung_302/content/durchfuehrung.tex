\section{Durchführung}
\label{sec:Durchführung}

Alle Versuche werden mit einer Speisespannung $U_S$ von 2,5 \si{\volt} und einer Frequenz von 500 \si{\hertz}
durchgeführt. Die Brückenspannungen $U_Br$ wird mit einem Oszillographen abgelesen. Das Verhältins von 
$R_3$ und $R_4$ ist im Folgenden immer \begin{equation}
    R_3 = 1000\si{\ohm} - R_4
\end{equation}

\subsection{Wheaton'sche Brücke}

Für einen festen Wert $R_2$ wird $\frac{R_3}{R_4}$ über einen Regler variiert, bis für $U_Br$ auf dem 
Oszillographen ein Minimum abzulesen ist. Dies wird für drei unterschiedliche, bekannte $R_2$ 
durchgeführt und jeweils $R_3$ und $R_4$ notiert. 

\subsection{Kapazitätsmessbrücke}

Hierbei ist neben $R_2$ ein bekannter Kondensator der Kapazität $C_2$ verbaut. 
Wieder wwerden nach Variieren von $\frac{R_3}{R_4}$ bei einem Spannungsminimun der
Brückenspannung deren Werte notiert. Es werden dafür drei unterschiedliche Konstellationen
von $R_2$ und $C_2$ verwendet. 

\subsection{Induktivitätsmessbrücke}

Es wird der Versuch komplett analog zur Kapazitätsmessbrücke durchgeführt, mit dem Unterschied, 
dass nur eine einzige Spule bekannter Induktivität $L_2$ im verfügbaren Inventar enthalten ist,
wodurch drei unterschiedliche Konstellationen von $L_2$ und $R_2$ nur durch drei verschiedene 
$R_2$ realisiert werden.

\subsection{Maxwell_Brücke}

Die drei Konstellationen werden durch unterschiedliche $C_4$ und $R_2$ realisiert, der Rest ist 
analog zu den vorherigen Versuchen.

\subsection{Wien-Robinson-Brücke}

In diesem Fall sind alle Bauelemente bekannt und es wird nur die Frequenz der Speisespannung
zwischen 20\si{hertz} und 30000\si{hertz} variiert. Die Schritte, mit denen die Frequenz erhöht
wird, werden dabei mit zunehmender Frequenz größer, wie in der Auswertung [ref tab] zu sehen ist.
Für jeden Wert wird bei konstanter Speisespannung $U_S$ die Brückenspannung $U_Br$ notiert.


