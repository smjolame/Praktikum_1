\section{Durchführung}
\label{sec:Durchführung}

Alle Versuche werden mit einer Speisespannung $U_{\text{S}}$ von 2,5 \si{\volt} und bis auf Versuch \ref{sec:Induktivitätsmessbrücke} mit einer Frequenz von 500 \si{\hertz}
durchgeführt. Bei dem Versuch \ref{sec:Induktivitätsmessbrücke} wird eine Frequenz von 500 \si{\kilo\hertz} verwendet. Die Brückenspannungen $U_{\text{Br}}$ wird mit einem Oszillographen abgelesen. Das Verhältins von 
$R_3$ und $R_4$ ist im Folgenden immer \begin{equation}
    R_3 = 1000\si{\ohm} - R_4
\end{equation}

\subsection{Wheaton'sche Brücke}

Für einen festen Wert $R_2$ wird $\frac{R_3}{R_4}$ über einen Regler variiert, bis für $U_{\text{Br}}$ auf dem 
Oszillographen ein Minimum abzulesen ist. Dies wird für drei unterschiedliche, bekannte $R_2$ 
durchgeführt und jeweils $R_3$ und $R_4$ notiert. 

\subsection{Kapazitätsmessbrücke}

Hierbei ist neben $R_2$ ein bekannter Kondensator der Kapazität $C_2$ verbaut. 
Wieder werden nach Variieren von $\frac{R_3}{R_4}$ bei einem Spannungsminimun der
Brückenspannung deren Werte notiert. Es werden dafür drei unterschiedliche Konstellationen
von $R_2$ und $C_2$ verwendet. 

\subsection{Induktivitätsmessbrücke}
\label{sec:Induktivitätsmessbrücke}
Es wird der Versuch analog zur Kapazitätsmessbrücke durchgeführt, mit dem Unterschied, 
dass nur eine einzige Spule bekannter Induktivität $L_2=20.1\si{\milli\henry}$ im verfügbaren Inventar enthalten ist,
wodurch drei unterschiedliche Konstellationen von $L_2$ und $R_2$ nur durch drei verschiedene 
$R_2$ realisiert werden.

\subsection{Maxwell-Brücke}

Die drei Konstellationen werden durch unterschiedliche $C_4$ und $R_2$ realisiert, der Rest ist 
analog zu den vorherigen Versuchen.

\subsection{Wien-Robinson-Brücke}
\label{sec:wien}
In diesem Fall sind alle Bauelemente bekannt und es wird nur die Frequenz der Speisespannung
zwischen 20\si{\hertz} und 30000\si{\hertz} variiert. Die Schritte, mit denen die Frequenz erhöht
wird, werden dabei mit zunehmender Frequenz größer, wie in der Auswertung [ref tab] zu sehen ist.
Für jeden Wert wird bei konstanter Speisespannung $U_{\text{S}}$ die Brückenspannung $U_{\text{Br}}$ notiert. 
Dabei sind die in [abb 7] dargestelleten Werte $R'=500\si{\ohm}$, $C=\SI{994e-9}{\farad}$ und $R=664\si{\ohm}$.


